\section{Неоднородные линейные уравнения с постоянными коэффициентами, характеристические функции}

    Рассматривать будем уравнения следующего вида
    \[
        y^{(n)} + p_1 y^{(n - 1)} + \dots + p_{n - 1} y' + p_n y = f(x), ~ p_k - \const ~ \forall k = \overline{1, n}.
    \]
    В общем случае решение можно строить с помощью метода вариации произвольных постоянных, или с помощью метода обратных операторов. Но существует частный случай функций правой части. Она имеют следующий вид
    \[
        f(x) = \sum^m_{j = 1} P^k_j(x) e^{\lambda_j x}, ~ \lambda_j \in \mathscr{C}, ~ P^k_j(x) = A_{j0} x^k + A_{j1} x^{k - 1} + \dots + A_{jk}.
    \]
    Сумма произведений некоторых полиномиальных функций на экспоненциально-тригонометрические -- экспоненты, синусы или косинусы. Такие функции называются характеристическими функциями. Примеры:
    \[
        x^2 \sin{2x}, ~ \bpares{x^3 - 1} e^{2x} \cos{x}, ~ e^{-2x}, ~ \sin{5x}.
    \]
    Примеры функций, не являющихся характеристическими:
    \[
        \dfrac{\sin{2x}}{x}, ~ \cos{x^2}, ~ e^{-x} \tan{x}, ~ \dfrac{x^4 e^x}{\cos{2x}}, ~ \sin{x} \ln{x}. 
    \]
    Для характеристических функций частное решение неоднородного уравнения можно искать по образу правой части в следующем виде
    \[
        y = \sum^m_{j = 1} Q^k_j(x) x^{r_j} e^{\lambda_j x}, ~ \lambda_j \in \mathscr{C}, ~ Q^k_j(x) = B_{j0} x^k + B_{j1} x^{k - 1} + \dots + B_{jk}.
    \]
    Здесь $ r_j $ -- кратность корня $ \lambda_j $ из решений однородного уравнения. $ Q^k_j(x) $ -- полином с неопределенными коэффициентами. Для каждого слагаемого будем расписывать характеристику в следующей форме
    \[
        \begin{split}
            \Big[ ~ &k_j ~ \Big] ~ \lambda_j \\
                    & + r_j
        \end{split}
    \]
    Здесь $ k_j $ -- степень полинома в данном слагаемом, $ \lambda_j $ -- собственное значение экспоненциальной функции (характеристический корень функции), $ r_j $ -- кратность характеристического корня среди корней однородного уравнения. Характеристика отвечает на три вопроса:
    \begin{enumerate}
        \item полином какой степени присутствует в характеристической функции (в квадратной скобке);
        \item какой характеристический корень описывается в характеристической функции (справа от квадратной скобки);
        \item сколько раз среди корней однородного уравнения присутствует данный характеристический корень (подпись с плюсом к квадратной скобке, относящаяся к полиномиальной части).
    \end{enumerate}
    Не стоит забывать, что для тригонометрических форм комплексные собственные значения идут парами. В случае, если характеристические функции имеют один и тот же характеристический корень, их можно объединить, выбрав полином более старшей степени.

    \subsection{Примеры}

        \[
            L[y] = x^2 \sin{2x} + e^x \cos{x} - 1 + xe^{-x} + x^2, ~ \lambda_{1, 2} = -1, ~ \lambda_{3, 4} = \pm i, ~ \lambda_{5, 6} = \pm 2i.
        \]
        На основе корней характеристического уравнения построим общее однородное решение
        \[
            y_\text{оо} = C_1 e^{-x} + C_2 xe^{-x} + C_3 \cos{x} + C_4 \sin{x} + C_5 \cos{2x} + C_6 \sin{2x}.
        \]
        Теперь для каждого слагаемого правой части распишем характеристику
        \[
            \begin{matrix}
                &[ ~ 2 ~ ] ~ \pm 2i &[ ~ 0 ~ ] ~ 1 \pm i &[ ~ 0 ~ ] ~ 0 &[ ~ 1 ~ ] ~ -1 &[ ~ 2 ~ ] ~ 0 \\
                &+1 &+0 &+0 &+2 &+0 \\
                f(x) = &x^2 \sin{2x} + &e^x \cos{x} - &1 + &xe^{-x} + &x^2.
            \end{matrix}
        \]
        В данном выражении третью и пятую характеристики можно объединить, так как они отвечают одному и тому же характеристическому корню. Тогда на основе данных характеристик строим частное решение неоднородного уравнения с неопределенными коэффициентами
        \[
            \begin{split}
                y &= \bpares{A_1 x^2 + A_2 x + A_3} x \cos{2x} + \bpares{B_1 x^2 + B_2 x + B_3} x \sin{2x} + \\ &+ D_1 e^x \cos{x} + E_1 e^x \sin{x} + \bpares{F_1 x^2 + F_2 x + F_3} + \bpares{G_2 x + G_3} x^2 e^{-x}.
            \end{split}
        \]
        А общее решение строится путем линейной комбинации общего однородного и частного неоднородного решений.

        Рассмотрим теперь пример, в котором необходимо найти конкретное частное решение.
        \[
            y'' - 2y' + y = 2e^x + x - 2.
        \]
        Его общее однородное решение имеет вид
        \[
            y_\text{оо} = C_1 e^{x} + C_2 xe^{x}
        \]
        Распишем характеристики для правой части
        \[
            \begin{matrix}
                &[ ~ 0 ~ ] ~ 1 &[ ~ 1 ~ ] ~ 0 \\
                &+2 &+0 \\
                f(x) = &2e^x + &x - 2.
            \end{matrix}
        \]
        Запишем общий вид частного решения с неопределенными коэффициентами согласно данным характеристикам
        \[
            y_\text{чн} = Ax^2 e^x + Bx + D.
        \]
        Найдем коэффициенты с помощью подстановки решения в уравнение. Для этого найдем производные
        \[
            \begin{split}
                &y'_\text{чн} = A\bpares{x^2 + 2x}e^x + B; \\
                &y''_\text{чн} = A\bpares{x^2 + 4x + 2} e^x.
            \end{split}
        \]
        Подставим в уравнение и упростим
        \[
            \begin{split}
                &A\bpares{x^2 + 4x + 2} e^x - 2 \Big( A\bpares{x^2 + 2x}e^x + B \Big) + Ax^2 e^x + \bpares{Bx + D} = 2e^x + x - 2; \\
                &2Ae^x + Bx + D - 2B = 2e^x + x - 2.
            \end{split}
        \]
        Приравняем коэффициенты левой части к коэффициентам правой части у соответствующих функций. Тогда
        \[
            \begin{cases}
                A = 1, \\
                B = 1, \\
                D = 0.
            \end{cases}
        \]
        Тогда частное решение неоднородного уравнения принимает вид
        \[
            y_\text{чн} = x^2 e^x + x.
        \]
        Общее решение имеет вид
        \[
            y = C_1 e^x + C_2 xe^x + x^2 e^x + x.
        \]