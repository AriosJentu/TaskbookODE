\section{Метод вариации произвольных постоянных для линейных уравнений высших порядков}
    Рассматривать будем уравнения следующего вида
    \[
        a_0(x) y^{(n)} + a_1(x) y^{(n - 1)} + \dots + a_{n - 1}(x) y' + a_n(x) y = f(x).
    \]
    Предположим, что нам известна фундаментальная система решений соответствующего однородного уравнения
    \[
        \varPhi = \{ y_1, ~\dots, y_n \}
    \]
    и соответствующее общее однородное решение
    \[
        y_\text{оо} = C_1 y_1 + \dots + C_n y_n.
    \]
    Тогда общее решение неоднородного уравнения можно построить с помощью метода вариации произвольных постоянных. Для этого разложим исходное равнение на уже известную систему уравнений первого порядка с помощью замены
    \[
        \begin{split}
            &u_1 = y, ~ u_2 = y', ~ \dots, ~ u_n = y^{(n - 1)}; \\
            &\begin{cases}
                u'_1 = u_2, \\
                u'_2 = u_3, \\
                \dots
                u'_n = -p_1(x) u_n - p_2(x) u_{n - 1} - \dots - p_n(x) u_1 + q(x).
            \end{cases}
        \end{split}
    \]
    Согласно замене, можно выписать однородное решение для $ u_k $
    \[
        u_k = C_1 y^{(k)}_1 + C_2 y^{(k)}_2 + \dots + C_n y^{(k)}_n.
    \]
    Теперь, согласно методу вариации произвольных постоянных, положим произвольные постоянные функциями от $ x $
    \[
        u_k = C_1(x) y^{(k)}_1 + C_2(x) y^{(k)}_2 + \dots + C_n(x) y^{(k)}_n.
    \]
    Найдем производные
    \[
        u'_k = C'_1(x) y^{(k)}_1 + \dots + C'_n(x) y^{(k)}_n + C_1(x) y^{(k + 1)}_1 + \dots + C_n(x) y^{(k + 1)}_n.
    \]
    Подставляя в систему, общее однородное решение сократится, останется только выражение относительно произвольных постоянных
    \[
        \begin{cases}
            C'_1 y_1 + \dots + C'_n y_n = 0, \\
            C'_1 y'_1 + \dots + C'_n y'_n = 0, \\
            \dots \\
            C'_1 y^{(n - 1)}_1 + \dots + C'_n y^{(n - 1)}_n = q(x).
        \end{cases}
    \]
    Получена система линейных алгебраических уравнений относительно производных от функций $ C_k $. Перепишем ее в матричном виде:
    \[
        \begin{pmatrix}
            y_1 & y_2 & \dots & y_n \\
            y'_1 & y'_2 & \dots & y'_n \\
            \vdots & \vdots & \ddots & \vdots \\
            y^{(n - 1)}_1 & y^{(n - 1)}_2 & \dots & y^{(n - 1)}_n
        \end{pmatrix}
        \begin{pmatrix}
            C'_1 \\ C'_2 \\ \vdots \\ C'_n
        \end{pmatrix}
        =
        \begin{pmatrix}
            0 \\ 0 \\ \vdots \\ q(x)
        \end{pmatrix}.
    \]
    Матрица в левой части является матрицей Вронского, и, так как ФСР состоит из линейно-независимых решений, то ее определитель не равен нулю, а значит матрица обратима. Тогда у системы существует единственное ненулевое решение, которое можно записать в следующем виде
    \[
        \vec{C}' = W^{-1}(x) \cdot \vec{Q}(x), ~ \vec{C}' =
        \begin{pmatrix}
            C'_1 \\ C'_2 \\ \vdots \\ C'_n
        \end{pmatrix}, ~
        \vec{Q} =
        \begin{pmatrix}
            0 \\ 0 \\ \vdots \\ q(x)
        \end{pmatrix}.
    \]
    В результате получена система уравнений первого порядка с разделенными переменными. Производя интегрирование, можно найти значение функций $ C_k $, и затем подставить их в общее неоднородное решение.

    \subsection{Примеры}
        \[
            y'' - 2y' + y = 2e^x -6x, ~ y_1 = e^x.
        \]
        Дано линейное приведенное неоднородное уравнение второго порядка с постоянными коэффициентами. Известно частное решение однородного уравнения, на его основе построим общее решение однородного уравнения
        \[
            y_2 = C_1 e^x \int \dfrac{e^{2 \displaystyle \int dx}}{e^{2x}} ~ dx + C_2 e^x = C_1 xe^x + C_2 e^x.
        \]
        В силу произвольности постоянных, изменим их порядок, тогда однородное решение принимает вид
        \[
            y_\text{оо} = C_1 e^x + C_2 xe^x.
        \]
        С помощью метода вариации произвольных постоянных, положим $ C_k = C_k(x) $
        \[
            y_\text{он} = C_1(x) e^x + C_2(x) xe^x.
        \]
        Сразу запишем матричную систему относительно неизвестных производных $ C_k $
        \[
            \begin{pmatrix}
                e^x & xe^x \\
                e^x & (x + 1)e^x
            \end{pmatrix}
            \begin{pmatrix}
                C'_1 \\ C'_2
            \end{pmatrix} =
            \begin{pmatrix}
                0 \\ 2e^x - 6x
            \end{pmatrix}.
        \]
        Разделим уравнение на $ e^x $
        \[
            \begin{pmatrix}
                1 & x \\
                1 & x + 1
            \end{pmatrix}
            \begin{pmatrix}
                C'_1 \\ C'_2
            \end{pmatrix} =
            \begin{pmatrix}
                0 \\ 2 - 6xe^{- x}
            \end{pmatrix}.
        \]
        Найдем обратную матрицу, запишем матричную систему, разрешенную относительно вектора производных $ C'_k $
        \[
            \begin{pmatrix}
                C'_1 \\ C'_2
            \end{pmatrix}
            = \drecp{x + 1 - x}
            \begin{pmatrix}
                x + 1 & -x \\
                -1    & 1
            \end{pmatrix}
            \begin{pmatrix}
                0 \\ 2 - 6xe^{- x}
            \end{pmatrix}
            =
            \begin{pmatrix}
                6x^2e^{-x} - 2x \\ 2 - 6xe^{- x}.
            \end{pmatrix}
        \]
        Запишем в виде алгебраической системы и проинтегрируем
        \[
            \begin{cases}
                C'_1 = 6x^2e^{-x} - 2x, \\
                C'_2 = 2 - 6xe^{- x}.
            \end{cases} 
            \Longrightarrow
            \begin{cases}
                C_1(x) = -6\bpares{x^2 + 2x + 2}e^{-x} - x^2 + \tilde{C}_1, \\
                C_2(x) = 6(x + 1)e^{-x} + 2x + \tilde{C}_2.
            \end{cases} 
        \]
        Подставим полученные функции в общее неоднородное решение
        \[
            y = C_1 e^x + C_2 xe^x - 6\bpares{x^2 + 2x + 2} - x^2 e^x + 6(x + 1)x + 2x^2 e^x.
        \]
        Упростим
        \[
            y = C_1 e^x + C_2 xe^x + x^2e^x - 6x - 12.
        \]
