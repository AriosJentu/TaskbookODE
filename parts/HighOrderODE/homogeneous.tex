\section{Однородные уравнения высших порядков}

    Одной из классификаций нелинейных уравнений высших порядков является понятие однородных уравнений. Введем понятие однородной функции своих аргументов. Рассмотрим общую функцию:
    \[
        F \bpares{x_1, x_2, \dots, x_n}.
    \]

    Однородной функцией называется функция, в которой при домножении аргументов на коэффициент $ k $, его можно вынести как сомножитель в некоторой степени $ m $
    \[
        F \bpares{kx_1, kx_2, \dots, kx_n} = k^m F \bpares{x_1, x_2, \dots, x_n}.
    \]
    Здесь $ k, m $ -- вещественные числа.

    \begin{enumerate}
        \item Однородным дифференциальным уравнением по степеням $ y $ будем называть уравнение вида
            \[
                F \Big( x, y, y', y'', \dots, y^{(n)} \Big) = 0
            \]
            которое задано с помощью однородной функции только относительно $ y $ и её производных
            \[
                F \Big( x, ky, ky', ky'', \dots, ky^{(n)} \Big) = k^m F \Big( x, y, y', y'', \dots, y^{(n)} \Big).
            \]
            Здесь $ k, m $ будем называть коэффициентом и показателем однородности соответственно. Такие уравнения в большинстве случаев имеют следующий вид:
            \[
                \sum^p_{j = 0} f_j(x) y^{a_{1j}} \bpares{y'}^{a_{2j}} \dots \Big( y^{(n)} \Big)^{a_{nj}} = 0.
            \]
            У таких уравнений для каждого $ j $-го слагаемого можно вычислить показатель однородности:
            \[
                k_j = \sum^n_{i = 1} a_{ij}.
            \]
            Если у каждого слагаемого показатель однородности одинаков, то порядок уравнения можно понизить с помощью замены
            \[
                y' = uy.
            \]

        \item Однородным уравнением относительно $ x, y $ и их производных (уравнением с показательной однородностью, обобщенным однородным уравнением) будем называть уравнение, представленное в виде полной однородной функции -- функция, однородная по всем своим аргументам
            \[
                F \Big( kx, ky, ky', ky'', \dots, ky^{(n)} \Big) = k^m F \Big( x, y, y', y'', \dots, y^{(n)} \Big).
            \]
            В общем случае такие функции представляются в следующем виде
            \[
                \sum^p_{j = 0} b_j x^{a_{0j}} y^{a_{1j}} \bpares{y'}^{a_{2j}} \dots \Big( y^{(n)} \Big)^{a_{nj}} = 0.
            \]
            Данный случай аналогичен классическим уравнениям первого порядка с показательной однородностью. Решение будем строить в виде степенной функции:
            \[
                y = zx^m, ~ y = z^m.
            \]
            Степень будем определять через соответствующую характеристическую систему, строящуюся по следующим правилам:
            \begin{itemize}
                \item операции сложения, вычитания и равенства заменяются на равенства;
                \item операции умножения заменяются на сложение, деление на вычитание;
                \item каждый $ x^p $ заменяется на $ p $;
                \item каждый $ y^p $ заменяется на $ mp $, где $ m $ -- показатель однородности, пока неизвестен;
                \item каждый $ \bpares{y^{(k)}}^p $ заменяется на $ (m - k)p $.
            \end{itemize}
            Если из такой системы удается найти показатель однородности $ m $, то в таком случае можно применять следующие замены:
            \[
                y = uz^m, ~ u = u(x); ~ y = u^m, u = u(x); ~
                \begin{cases}
                    x = e^t, \\
                    y = ue^{mt}.
                \end{cases};
                ~ u = u(t).
            \]
            Первая и вторая замена приводят к уравнениям, содержащим аргумент $ x $, и лучше их использовать только в случаях, если в уравнении отсутствует переменная $ x $ в явном виде. Третья же замена в таких уравнениях полностью исключает переменную $ x $ и $ t $ из уравнения в явном виде, если такая присутствует.

        \item Уравнения вида в которых левая часть является точной производной некоторой функции
            \[
                F \Big( x, y, y', y'', \dots, y^{(n)} \Big) = G' \Big( x, y, y', y'', \dots, y^{(n - 1)} \Big) 
            \]
            называются уравнениями, допускающими полное интегрирование (вполне интегрируемые уравнения). Их решение строится путем полного интегрирования.

    \end{enumerate}

    \subsection{Примеры}

        \[
            xyy'' - x{y'}^2 = yy'.
        \]
        В данном уравнении $ y $ со своими производными в каждом слагаемом суммарно входит во второй степени (показатель однородности равен двум). Тогда проведем замену, найдем производные, и подставим в уравнение
        \[
            \begin{split}
                &y' = uy, ~ u = u(x); \\
                &y'' = u'y + uy' = u'y + u^2y = y\bpares{u' + u^2}; \\
                &xy^2 \bpares{u' + u^2} -xu^2y^2 = y^2u ~ \Bigg| ~ /y^2.
            \end{split}
        \]
        Раскрывая скобки, получаем простое уравнение:
        \[
            \begin{split}
                &xu' = u; \\
                &u = \dfrac{y'}{y} = C_1 x; \\
                &\ln{y} = C_1 x^2 + C_2.
            \end{split}
        \]

        \[
            x^4y'' + 4y \bpares{x^2 - y + xy'} = x^2 y' \bpares{3x + y'} + x^4.
        \]
        Раскрывая скобки, и обращая внимание на то, что данное уравнение является уравнением, представленным в виде полиномов относительно функций $ x, y $ и их производных, запишем систему для показателя однородности:
        \[
            m + 2 = m + 2 = 2m = 2m = m + 2 = 2m + 4 \Longrightarrow m = 2.
        \]
        Из данной системы можно определить значение $ m $, значит перед нами стоит уравнение с показательной однородностью. Проведем замену:
        \[
            \begin{cases}
                x = e^t, \\
                y = ue^{2t}.
            \end{cases};
        \]
        Найдем все производные функции $ y $:
        \[
            \begin{split}
                &y' = \difft{y}{x} = \dfrac{\difft{y}{t}}{\difft{x}{t}} = \dfrac{\dot{u}e^{2t} + 2u e^{2t}}{e^t} = (\dot{u} + 2u)e^t; \\
                &y'' = \difft{y'}{x} = \dfrac{\difft{y'}{t}}{\difft{x}{t}} = \dfrac{(\ddot{u} + 3\dot{u} + 2u)e^t}{e^t} = \ddot{u} + 3\dot{u} + 2u.
            \end{split}
        \]
        Подставим полученные выражения в исходное уравнение, вынесем из каждого слагаемого общий множитель
        \[
            e^{4t} (\ddot{u} + 3\dot{u} + 2u) + 4e^{4t} \bpares{u - u^2 + u(\dot{u} + 2u)} = e^{4t} (\dot{u} + 2u) (\dot{u} + 2u + 3) + e^{4t}.
        \]
        Упростим и проинтегрируем
        \[
            \begin{split}
                &\ddot{u} = \dot{u}^2 + 1; \\
                &\dot{u} = \tan{t + C_1}; \\
                &u = \ln{\sec{(t + C_1)}} + C_2 \\
                &y = x^2 \ln{\sec{(\ln{x} + C_1)}} + C_2 x^2.
            \end{split}
        \]

        \[
            yy'' + {y'}^2 + 2y^2{y'}^2 = \dfrac{yy'}{x}.
        \]
        Разделим уравнение на $ yy' $
        \[
            \dfrac{y''}{y'} + \dfrac{y'}{y} + 2yy' = \drecp{x}.
        \]
        Теперь каждое слагаемое представляет из себя полную производную. Проинтегрируем
        \[
            \ln{yy'} + y^2 = \ln{x} + C_1.
        \]
        Потенцируем и интегрируем снова
        \[
            \begin{split}
                &yy'e^{y^2} = C_1 x; \\
                &e^{y^2} = C_1 x^2 + C_2.
            \end{split}
        \]
