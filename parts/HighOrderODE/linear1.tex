\section{Линейные уравнения с переменными коэффициентами}

    Рассматривать будем уравнения следующего вида
    \[
        a_0(x) y^{(n)} + a_1(x) y^{(n - 1)} + \dots + a_{n - 1}(x) y' + a_n(x) y = f(x).
    \]
    Аналогично уравнениям первого порядка, классификация уравнения может быть следующей:
    \begin{itemize}
        \item порядок уравнения -- $ n $;
        \item $ a_0(x) = 1 $ -- приведённое(иначе -- неприведённое);
        \item $ f(x) = 0 $ -- однородное(иначе -- неоднородное);
        \item $ a_k(x) = \const $ для всех $ k $ -- с постоянными коэффициентами(иначе -- с переменными).
    \end{itemize}
    Рассматривать будем пока однородные уравнения.

    Любая линейная комбинация частных решений однородного линейного уравнения также будет считаться частным решением
    \[
        y_\text{ч} = k_1 y_\text{ч1} + \dots + k_m y_\text{ч$ m $}, ~ k_i - \const ~ \forall i = \overline{1, m}, ~ m \leq n.
    \]
    Фундаментальной системой решений будет называть множество всех возможных линейно-независимых частных решений однородного уравнения:
    \[
        \varPhi = \{ y_1, \dots, y_n \}.
    \]
    Система функций называется линейно-независимой, если определитель Вронского над заданной системой функций не равен нулю. Определителем Вронского системы функций называется определитель матрицы следующего вида:
    \[
        \det{\bpares{W(y_1, \dots, y_n)}} =
        \begin{vmatrix}
            y_1 & y_2 & \dots & y_n \\
            y'_1 & y'_2 & \dots & y'_n \\
            \vdots & \vdots & \ddots & \vdots \\
            y^{(n - 1)}_1 & y^{(n - 1)}_2 & \dots & y^{(n - 1)}_n \\
        \end{vmatrix}
    \]
    Рассмотрим два примера. Первый:
    \[
        \begin{split}
            &\varPhi = \{ 1, x, e^x \}, \\
            &\det{\bpares{W(1, x, e^x)}} =
            \begin{vmatrix}
                1 & x & e^x \\
                0 & 1 & e^x \\
                0 & 0 & e^x \\
            \end{vmatrix}
            = e^x \neq 0.
        \end{split}
    \]
    Второй:
    \[
        \begin{split}
            &\varPhi = \{ x, x^2, 2x^2 - 3x \}, \\
            &\det{\bpares{W(x, x^2, 2x^2 - 3x)}} =
            \begin{vmatrix}
                x & x^2 & 2x^2 - 3x \\
                1 & 2x  & 4x - 3 \\
                0 & 2   & 4 \\
            \end{vmatrix}
            = x(8x - 8x + 6) - \bpares{4x^2 - 4x^2 + 6x} = 0.
        \end{split}
    \]

    Далее будем рассматривать приведенное однородное уравнение
    \[
        y^{(n)} + p_1(x) y^{(n - 1)} + \dots + p_{n - 1} y' + p_n(x) y = 0.
    \]
    Данное уравнение можно свести к системе линейных уравнений первого порядка, вводя следующую замену
    \[
        \begin{split}
            &y = y_1, ~ y' = y^2, ~ \dots, ~ y^{(n - 1)} = y_n \Longrightarrow y^{(n)} = y'_n; \\
            &\begin{cases}
                y'_1 = y_2, \\
                y'_2 = y_3, \\
                \dots \\
                y'_n = -p_1(x) y_n - p_2(x) y_{n - 1} - \dots - p_n(x) y_1.
            \end{cases}
        \end{split}
    \]
    Последнее уравнение этой системы назовем соответствующим линейным уравнением первого порядка для исходного. Зная все возможные $ n - 1 $ решений $ \{ y_1, \dots, y_{n - 1} \} $, решение для $ y_n $ можно получить как решение линейного уравнения первого порядка
    \[
        y_n = Ce^{-\displaystyle\int p_1(x) ~ dx}.
    \]
    При этом, $ y_n $ содержит в себе уже известные $ n - 1 $ линейно-независимых частных решений уравнений выше, а значит определитель Вронского для такой системы не равен нулю. Тогда введем зависимость между определителем Вронского и общим решением соответствующего линейного уравнения первого порядка
    \[
        \det{\bpares{W(y_1, \dots, y_n)}} = \det{W(x)} = y_n = Ce^{-\displaystyle\int p_1(x) ~ dx}.
    \]
    Представленное выражение называется формулой Лиувилля--Остроградского. На ее основе можно вывести построение общего решения однородного уравнения при известных $ n - 1 $ частных решений. Рассмотрим на примере уравнения второго порядка
    \[
        y'' + p_1(x) y' + p_2(x) y = 0.
    \]
    Положим $ y_1 $ -- одно известное частное решение однородного уравнения. Тогда определитель Вронского принимает следующий вид
    \[
        \begin{vmatrix}
            y_1 & y_2 \\
            y'_1 & y'_2 \\
        \end{vmatrix}
        = Ce^{-\displaystyle\int p_1(x) ~ dx}.
    \]
    Раскрывая определитель, получим
    \[
        y_1 y'_2 - y'_1 y_2 = Ce^{-\displaystyle\int p_1(x) ~ dx}.
    \]
    Зная одно частное решение, поделим на $ y^2_1 $, домножим на интегрирующий множитель и запишем уравнение в форме полной производной
    \[
        \pares{\dfrac{y_2}{y_1}}' = \dfrac{C e^{-\displaystyle\int p_1(x) ~ dx}}{y^2_1}.
    \]
    Интегрируя полученное уравнение, полагая $ C = C_1 $ и упрощая, получим линейную комбинацию двух частных решений однородного уравнения, которые являются линейно-независимыми. Эта комбинация является общим решением однородного уравнения первого порядка, построенного на известном частном решении
    \[
        y_2 = y_1 \pares{C_1 \int \dfrac{e^{-\displaystyle\int p_1(x) ~ dx}}{y^2_1} ~ dx + C_2}.
    \]
    Частные решения можно подбирать в виде произвольных функций, например, в виде полиномов, экспоненциальных или тригонометрических функций
    \[
        \begin{split}
            &y_\text{ч} = \alpha_0 x^n + \dots \alpha_{n - 1} x + \alpha_n; \\
            &y_\text{ч} = e^{\alpha x}; \\
            &y_\text{ч} = \cos{\alpha x}; \\
            &y_\text{ч} = \sin{\alpha x}.
        \end{split}
    \]

    Зная частное решение однородного уравнения, можно понизить порядок уравнения с помощью следующей замены:
    \[
        y = y_1 z, ~ z = z(x).
    \]
    Здесь $ y_1 $ -- частное решение однородного уравнения, $ z $ -- новая искомая функция.


    \subsection{Примеры}

        \[
            \bpares{1 - x^2}y'' -2xy' + 2y = 0.
        \]
        Классифицируем уравнение -- линейное неприведенное однородное уравнение второго порядка с переменными коэффициентами. Уравнение представлено в виде произведения степенных функций, тогда и частное решение будем искать в виде степенной функции
        \[
            y_1 = x^n, ~ y'_1 = nx^{n - 1}, ~ y''_1 = n(n - 1)x^{n - 2}.
        \]
        Подставим в уравнение и найдем подходящее значение $ n $
        \[
            \bpares{1 - x^2} n(n - 1) x^{n - 2} - 2nx^n + 2x^n = 0 \Longrightarrow x^n\bpares{-n(n - 2) - 2n + 2} = 0 \Longrightarrow n = 1, ~ n = -2.
        \]
        Построим полином первой степени, найдем для него соответствующие коэффициенты
        \[
            \begin{split}
                &y_1 = axb, ~ y' = a, ~ y'' = 0; \\
                &-2ax + 2ax + 2b = 0 \Longrightarrow y_1 = Cx.
            \end{split}
        \]
        Найдём общее решение с помощью формулы Лиувилля--Остроградского. Найдём определитель Вронского
        \[
            \det{W(x)} = Ce^{\displaystyle \int \dfrac{2x}{1 - x^2} ~ dx} = C_1 e^{-\ln{\bpares{1 - x^2}}} = \dfrac{C_1}{1 - x^2}.
        \]
        Распишем определитель Вронского, и построим полный интеграл линейного уравнения первого порядка
        \[
            \begin{split}
                &\begin{vmatrix}
                    x & y \\
                    1 & y'
                \end{vmatrix}
                = \dfrac{C_1}{1 - x^2}; \\
                &xy' - y = \dfrac{C_1}{1 - x^2} \Longrightarrow \pares{\dfrac{y}{x}}' = \drecp{x^2} \cdot \dfrac{C_1}{1 - x^2} \Longrightarrow y = x\pares{\int \drecp{x^2} \cdot \dfrac{C_1}{1 - x^2} ~ dx} + C_2.
            \end{split}
        \]
        Вычислим полученный интеграл
        \[
            \begin{split}
                \int \drecp{x^2} \cdot \dfrac{C_1}{1 - x^2} ~ dx &= \star
                \begin{vmatrix}
                    t = \drecp{x} \\
                    dt = -\dfrac{dx}{x^2}
                \end{vmatrix}
                \star = \int \drecp{\drecp{t^2} - 1} ~ dt = \int \dfrac{t^2}{t^2 - 1} ~ dt = \\ &= t - \artanh{t} + C = \drecp{x} - \artanh{\drecp{x}} + C.
            \end{split}
        \]
        Запишем общее решение
        \[
            y = C_1 \pares{1 - \artanh{\drecp{x}}} + C_2 x.
        \]

        Рассмотрим на этом же примере понижение порядка, если известно частное решение $ y = x $. Проведём замену
        \[
            y = zx, ~ y' = z'x + z, ~ y'' = z''x + 2z'.
        \]
        Подставим в уравнение и упростим
        \[
            \begin{split}
                &\bpares{1 - x^2}\bpares{z''x + 2z'} - 2x\bpares{z'x + z} + 2zx = 0; \\
                &x \bpares{1 - x^2} z'' = 2z' \bpares{2x^2 - 1}.
            \end{split}
        \]
        Получили уравнение, допускающее понижение порядка. Но мы здесь разделим переменные, зная, что
        \[
            z'' = \difft{z'}{x},
        \]
        получим
        \[
            \dfrac{dz'}{z'} = \dfrac{4x^2 - 2}{x \bpares{1 - x^2}} dx = \pares{\dfrac{2}{x\bpares{1 - x^2}} - \dfrac{4}{x}} dx.
        \]
        Проинтегрируем первое слагаемое в правой части
        \[
            \begin{split}
                \int \dfrac{2}{\bpares{1 - x^2}} &= \star
                \begin{vmatrix}
                    u = x^2 \\
                    du = 2x dx
                \end{vmatrix}
                \star = \int \dfrac{1 - u + u}{u(1 - u)} ~ du = \\ &= \int \pares{\drecp{u} + \drecp{1 - u}} ~ du = 2\ln{x} - \ln{\bpares{1 - x^2}} + C_1.
            \end{split}
        \]
        Проинтегрируем всё уравнение, зная найденный интеграл, затем спотенцируем
        \[
            \begin{split}
                &\ln{z'} = 2\ln{x} - \ln{\bpares{1 - x^2}} - 4\ln{x} + C_1; \\
                & z' = \dfrac{C_1}{x^2 \bpares{1 - x^2}}.
            \end{split}
        \]
        Было получено уравнение с разделяющимися переменными, которое можно проинтегрировать. Такой интеграл уже был вычислен ранее.
