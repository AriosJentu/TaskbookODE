\section{Линейные уравнения с постоянными коэффициентами. Характеристические уравнения}
    Рассматривать будем уравнения следующего вида
    \[
        y^{(n)} + p_1 y^{(n - 1)} + \dots + p_{n - 1} y' + p_n y = 0, ~ p_k - \const ~ \forall k = \overline{1, n}.
    \]
    Введем понятие дифференциального оператора
    \[
        D[y] = \difft{y}{x}.
    \]
    Оператор $ D $ -- дифференциальный оператор первого порядка, действующий на функцию $ y $, возвращающий в результате производную функции $ y $. Применим данный оператор последовательно дважды на функцию $ y $ -- применение оператора дифференцирования на самого же себя
    \[
        D \big[ D[y] \big] = D^2 [y] \difft{}{x} \pares{\difft{y}{x}} = \diffn{y}{x}{2}.
    \]
    Таким образом можно определить производную функции $ y $ порядка $ n $
    \[
        D^n[y] = \diffn{y}{x}{n}.
    \]
    Соответственно, если $ n = 0 $, то оператор является единичным
    \[
        D^0[y] = I[y] = y.
    \]
    Оператор $ D $ удовлетворяет условию линейности -- оператор от произвольной линейной комбинации является линейной комбинацией применения оператора
    \[
        D \big[ C_1 y_1 + \dots + C_n y_n \big] = C_1 D[y_1] + \dots C_n D[y_n].
    \]
    Таким образом, рассматриваемое уравнение можно записать в операторной форме
    \[
        \bpares{D^n + p_1 D^{n - 1} + \dots + p_{n - 1} D + p_n I} [y] = 0.
    \]
    Представленная линейная комбинация операторов является ничем иным, как полиномом степени $ n $ относительно оператора $ D $. Любая линейная комбинация операторов дифференцирования называется линейным дифференциальным оператором, и обозначается следующим образом
    \[
        L[y] = \bpares{D^n + p_1 D^{n - 1} + \dots + p_{n - 1} D + p_n I} [y].
    \]
    Тогда исходное уравнение принимает следующий вид
    \[
        L[y] = 0.
    \]
    Задача сводится к нахождению обратного оператора к оператору $ L $ для получения общего решения. Метод решения называется метод обратных операторов.

    Будем рассматривать оператор $ L $ как линейную комбинацию дифференциальных операторов $ D $. Представленная комбинация является полиномом порядка $ n $ относительно <<неизвестного>> $ D $, и может быть разложена как последовательное применение операторов
    \[
        D^n + p_1 D^{n - 1} + \dots + p_{n - 1} D + p_n I = \bpares{D - \lambda_1 I} \bpares{D - \lambda_2 I} \dots \bpares{D - \lambda_n I}.
    \]
    Здесь числа $ \lambda_k $ называются собственными значениями линейного дифференциального оператора, и являются корнями соответствующего характеристического уравнения
    \[
        \lambda^n + p_1 \lambda^{n - 1} + \dots + p_{n - 1} \lambda + p_n = 0.
    \]
    Для построения общего решения достаточно найти общий вид обратного оператора. Положим, что применение оператора первого порядка на функцию $ y $ даст некоторую функцию $ u $ (нам известно выражение $ u $ через $ y $). Тогда найдем выражение самого $ y $ через $ u $
    \[
        \bpares{D - \lambda_k I}[y] = u.
    \]
    Представленное выражение можно расписать как линейное приведенное неоднородное уравнение первого порядка с постоянными коэффициентами
    \[
        y' - \lambda_k y = u(x) \Longrightarrow y = e^{\lambda_k x} \pares{ C_k + \int\limits^x_{x_0} u(x) e^{-\lambda_k x} ~ dx }.
    \]
    Обозначим интеграл за оператор $ B $
    \[
        B_{\lambda_k} [x] = e^{\lambda_k x} \int\limits^x_{x_0} u(x) e^{-\lambda_k x} ~ dx.
    \]
    Тогда $ y $ можно представить через оператор $ B $ в следующем виде
    \[
        y = C_k e^{\lambda_k x} + B_{\lambda_k} [u].
    \]

    Пусть все собственные значения
    \[
        \lambda = \lambda_k, ~ \lambda_i \neq \lambda_j ~ \forall ~ i \neq j.
    \]
    Тогда решение имеет вид
    \[
        y = C_1 e^{\lambda_1 x} + C_2 e^{\lambda_2 x} + \dots + C_n e^{\lambda_n x}.
    \]
    Если есть кратные собственные значения
    \[
        \lambda_i = \lambda_j ~ \forall ~ i, j = \overline{1, n}, 
    \]
    то кратный корень умножается на $ x $ относительно предыдущего
    \[
        y = C_1 e^{\lambda_1 x} + C_2 x e^{\lambda_2 x} + \dots + C_n x^{n - 1} e^{\lambda_n x}.
    \]
    В случае комплексных корней решение строится следующим образом
    \[
        \begin{split}
            &\lambda_{1, 2} = \alpha \pm \beta i, \\
            &y = C_1 e^{\alpha x} \cos{\beta x} + C_2 e^{\alpha x} \sin{\beta x}.
        \end{split}
    \]
    Для комплексно-сопряженных корней введем <<договоренность>> -- за положительную комплексную часть (за знак плюс) будет отвечать вещественное решение, содержащее косинус, за отрицательную часть (за знак минус) будет отвечать вещественное решение, содержащее синус.

    Уравнением Эйлера-Лагранжа (уравнением Эйлера) называется уравнение следующего вида
    \[
        x^n y^{(n)} + p_1 x^{n - 1} y^{(n - 1)} + \dots + p_{n - 1} x y' + p_n y = 0, ~ p_k - \const ~ \forall k = \overline{1, n}.
    \]
    Его можно свести к линейному уравнению с постоянными коэффициентами с помощью замены
    \[
        x = e^t, ~ y = y(t).
    \]
    Соответствующее характеристическое уравнение имеет следующий вид
    \[
        \lambda(\lambda - 1) \dots \bpares{\lambda - (n - 1)} + p_1 \lambda(\lambda - 1) \dots \bpares{\lambda - (n - 2)} + \dots + p_{n - 1} \lambda + p_n = 0.
    \]
    Общее решение аналогично уравнению с постоянными коэффициентами, но при этом все экспоненциально-полиномиальные формы переходят в следующий вид
    \[
        e^{\lambda_k x} \rightarrow x^{\lambda_k}; ~ x^n \rightarrow \ln^n{x}; ~ \sin{\beta_k x}, ~ \cos{\beta_k x} \rightarrow \sin{\beta_k \ln{x}}, ~ \cos{\beta_k \ln{x}}.
    \]

    \subsection{Примеры}

        \[
            y'' - 3y' + 2y = 0.
        \]
        Запишем уравнение в операторной форме
        \[
            \bpares{D^2 - 3D + 2I}[y] = 0.
        \]
        Зная собственные значения оператора, разложим его на последовательное применение операторов
        \[
            \bpares{D - I} \bpares{D - 2I} [y] = 0.
        \]
        В силу последовательности применения, сделаем следующую замену
        \[
            u = \bpares{D - 2I}[y].
        \]
        Получим следующее уравнение
        \[
            \bpares{D - I}[u] = 0.
        \]
        Построим его решение
        \[
            u = C_1 e^x + e^x \int\limits^x_{x_0} 0 \cdot e^x ~ dx = C_1 e^x.
        \]
        Сделаем обратную замену
        \[
            \bpares{D - 2I}[y] = C_1 e^x.
        \]
        И также, согласно обратному оператору, запишем решение
        \[
            y = C_2 e^{2x} + e^{2x} \int\limits^x_{x_0} C_1 e^x e^{-2x} ~ dx = C_1 e^x + C_2 e^{2x}.
        \]

        \[
            y'' + 4y' + 4y = 0.
        \]
        Проведем те же самые действия
        \[
            \begin{split}
                &\bpares{D^2 + 4d + 4I}[y] = 0, ~ \bpares{D + 2I} \bpares{D + 2I} [y] = 0; \\
                &u = \bpares{D + 2I}[y] \Longrightarrow \bpares{D + 2I} [u] = 0; \\
                &u = C_1 e^{-2x} + e^{-2x} \int \limits^x_{x_0} 0 \cdot e^{2x} ~ dx = C_1 e^{-2x}.
            \end{split}
        \]
        Возвращаемся к обратной замене и снова решаем уравнение первого порядка
        \[
            \begin{split}
                &\bpares{D + 2I}[y] = C_1 e^{-2x}; \\
                &y = C_2 e^{-2x} + e^{-2x} \int C_1 e^{-2x} e^{2x} ~ dx = C_1 x e^{-2x} + C_2 e^{-2x}.
            \end{split}
        \]
        В результате решения были получены кратные корни, в силу чего при построении общего решения вторым интегралом, решение $ e^{-2x} $ было умножено на $ x $.

        \[
            y'' - 2y' + 5y = 0.
        \]
        Построим соответствующее характеристическое уравнение
        \[
            \lambda^2 - 2\lambda + 5 = 0 \Longrightarrow \lambda_{1, 2}  = 1 \pm 2i.
        \]
        Как видим, собственные значения оператора являются комплексными, но при этом разными (положительная и отрицательная ветки соответственно). Построим решение согласно случаю разных корней
        \[
            y = C_1 e^{(1 + 2i)x} + C_2 e^{(1 - 2i)x}.
        \]
        Вынесем $ e^x $ и разложим комплексные экспоненты по формулам Эйлера
        \[
            y = e^x \bpares{C_1 (\cos{2x} + i\sin{2x}) + C_2 (\cos{2x} - i\sin{2x})}.
        \]
        Выделим соответствующие постоянные перед тригонометрическими выражениями
        \[
            y = e^x \bpares{(C_1 + C_2) \cos{2x} + i(C_1 - C_2) \sin{2x}}.
        \]
        И совершим <<магию>> произвольных постоянных. В решении должно быть две произвольные постоянные, при этом постоянные перед тригонометрическими функциями линейно независимы, соответственно, возможно сделать следующее переобозначение
        \[
            \tilde{C}_1 = C_1 + C_2, ~ \tilde{C}_2 = i(C_1 - C_2).
        \]
        Тогда общее решение принимает следующий вид
        \[
            y = C_1 e^x \cos{2x} + C_2 e^x \sin{2x}.
        \]

        Рассмотрим следующий пример линейного оператора. Задача -- построить общее решение при известных собственных значениях линейного дифференциального оператора
        \[
            L[y] = 0; ~ \lambda_{1, 2} = 2, ~ \lambda_{3, 4} = -2 \pm 3i, ~ \lambda_{5, 6} = \pm i, ~ \lambda_{7, 8} = -2 \pm 3i, ~ \lambda_{9, 10} = 0, ~ \lambda_{11, 12} = 2.
        \]
        Общее решение примет следующий вид
        \[
            \begin{split}
                y &= C_1 e^2x + C_2 xe^x + C_3 e^{-2x} \cos{3x} + C_4 e^{-2x} \sin{3x} + C_5 \cos{x} + C_6 \sin{x} + \\ &+ C_7 xe^{-2x} \cos{3x} + C_8 xe^{-2x} \sin{3x} + C_9 + C_{10} x + C_{11} x^2 e^{2x} + C_{12} x^3 e^{2x}.
            \end{split}
        \]

        \[
            x^2y'' + 3xy' + y = 0.
        \]
        Строим соответствующее характеристическое уравнение
        \[
            \lambda (\lambda - 1)  + 3 \lambda + 1 = 0 \Longrightarrow \lambda_{1, 2} = -1.
        \]
        Тогда общее решение принимает следующий вид
        \[
            y = C_1 x^{-1} + C_2 x^{-1} \ln{x} = \dfrac{C_1}{x} + \dfrac{C_2 \ln{x}}{x}.
        \]
