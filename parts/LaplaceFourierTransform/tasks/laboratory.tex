\section{Лабораторная работа}

	\subsection{Постановка задачи}

		Реализовать алгоритм построения интегрального преобразования Фурье для произвольно-заданных функций одной переменной. Для реализации потребуются следующие параметры:
		\begin{itemize}

			\item \( f(t) \) -- определение функции,
			\item \( a, b \) -- область интегрирования функции,
			\item \( n_1 \) -- количество разбиений области интегрирования (также можно использовать шаг $h_1$),
			\item \( n_2 \) -- количество разбиений частотного диапазона (также можно использовать шаг $h_2$),
			\item \( m \) -- максимальное значение частоты.

		\end{itemize}

		Задача сводится к построению спектрального разложения одномерного сигнала $f(t)$ на частоты составляющих его волн.

		Реализация проводится с помощью численных методов рассчета интегралов. Потребуется построить график функции $f(t)$, и ее вещественные и комплексные спектральные разложения ($\FourierRe{f}$ и $\FourierIm{f}$). Графики разложений строятся в диапазоне $\omega \in [0, m]$. Оси графиков разложений представляют из себя по горизонтали -- частотный диапазон, по вертикали -- амплитуда. 

		Решение оформить в среде \LaTeX.

	% \pagebreak
	\subsection{Пример}

		Рассмотрим функцию $f(t) = \sin{t}$ на диапазоне $[a, b] = [0, \pi]$, в остальном диапазоне полагая $f(t) = 0$. Для аналитического решения, функция примет вид: $f(t) = \sin{t} \cdot \Pi_{0, \pi}(t)$. Положим $\wh = 2\pi \omega$.
		Тогда преобразования Фурье примут вид:
		\[ \FourierRe{\sin{t}} = \int\limits_{-\infty}^{\infty} \sin{t} \cdot \Pi_{0, \pi}(t) \cdot \cos{\wh t} ~ dt = \int\limits_{0}^{\pi} \sin{t} \cos{\wh t} ~ dt = \frac{\cos{\pi \wh} + 1}{1 - \wh^2}. \] 
		\[ \FourierIm{\sin{t}} = \int\limits_{-\infty}^{\infty} \sin{t} \cdot \Pi_{0, \pi}(t) \cdot \sin{\wh t} ~ dt = \int\limits_{0}^{\pi} \sin{t} \sin{\wh t} ~ dt = \frac{\sin{\pi \wh}}{1 - \wh^2}, \] 

		В таком случае спектральный график в аналитической форме будет иметь следующий вид:
		\begin{figure}[H]
			\centering
			\includegraphics[width=0.5\textwidth]{additional/Lab/desplot.pdf}
			\caption{Спектральный график волн синус- и косинус-преобразований (красный -- синус, вещественное преобразование, синий -- косинус, комплексное преобразование)}
		\end{figure}

		В процессе реализации численного алгоритма, максимальное значение частоты выберем $m = 2$. Таким образом, численно найденный спектральный график имеет вид:
		\begin{figure}[H]
			\centering
			\includegraphics[width=0.7\textwidth]{additional/Lab/Waves.pdf}
			\caption{Численно рассчитанный спектральный график волн синус- и косинус-преобразований (оранжевый -- синус, синий -- косинус)}
		\end{figure}