\section{Обратное преобразование Лапласа}

	Обратное преобразование Лапласа $\InvLaplaceS{F(s)}{t}$ -- интегральное преобразование из функции комплексной переменной $s \in \mathbb{C}$ в функцию вещественной переменной $t \in \mathbb{R}$. Для решения задач, связанных с поиском обратного преобразования, рекомендуется пользоваться таблицей оригиналов и изображений. Обратное преобразование Лапласа удовлетворяет следующему соотношению:
	\[ \Laplace{f(t)} = F(s) ~ \implies ~ \InvLaplace{F(s)} = f(t). \]

	\subsection{Примеры}

		Рассмотрим следующий пример. Положим 
		\[ F(s) = \int\limits_{s}^{\infty} \frac{1}{\sigma^2} ~ d \sigma. \] 
		Проинтегрировав функцию $F(s)$, получим, что 
		\[ F(s) = \frac{1}{s}, \] 
		а ее обратное преобразвание будет $f(t) = 1$. 

		\vspace{20pt}

		Так же, аналогичное решение можно получить, пользуясь свойством обратного преобразования. Найдем производную от образа для того, чтобы избавиться от оператора интегрирования: 
		\[ F'(s) = -\frac{1}{s^2}. \] 
		Тогда, по свойству преобразования производной, получим 
		\[ f(t) = -\frac{1}{t} \cdot \InvLaplace{-\frac{1}{s^2}} = \frac{t}{t} = 1. \]

		\vspace{20pt}

		Аналогично, найдем решение уравнения, рассматриваемого ранее:
		\[ y' + \delta(t-1) \cdot y = 1, ~ y(1) = 0, ~ y = y(t), \]
		при известном решении в образах:
		\[ Y(s) = \frac{1}{s^2}. \]
		Применяя обратное преобразование, получим, что
		\[ y(\tau) = \tau, \]
		или
		\[ y(t) = t - 1. \]
		Дифференцируя и подставляя в уравнение, можно убедиться, что полученная функция является решением для любых значений, кроме $t = 1$. При $t = 1$ возможен разрыв за счет присутствия в уравнении функции $\delta(t-1)$.
		Для того, чтобы определить, является ли данная функция решением в окрестности $t = 1$, рассмотрим следующий предел: 
		\[ \lim_{t \to 1} ~ y \cdot \delta(t-1) = \lim_{t \to 1} ~ (t-1) \cdot \delta(t-1). \]
		Нетрудно убедиться, что этот предел в точности равен $0$. Соответственно, при $t = 1$, удовлетворяется как уравнение, так и начальное условие. А значит, функция $y = t - 1$ является решением данного уравнения.

	% \pagebreak