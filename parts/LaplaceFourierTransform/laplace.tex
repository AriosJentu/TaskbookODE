\section{Преобразование Лапласа}

	Преобразование Лапласа $\LaplaceS{f(t)}{s}$ -- интегральное преобразование из функции вещественной переменной $t \in \mathbb{R}$ в функцию комплексной переменной $s \in \mathbb{C}$, $s = \alpha + i \omega$, $\alpha, \omega \in \mathbb{R}$, описывающееся следующим выражением:
	\[ \LaplaceS{f(t)}{s} = F(s) = \int\limits_{0}^{\infty} e^{-st} \cdot f(t) ~ dt. \]
	Для упрощения, в дальнейшем будем записывать $\Laplace{f}$, подразумевая $\LaplaceS{f(t)}{s}$. Такое преобразование позволяет сводить задачи дифференциальных и интегральных уравнений к алгебраическим задачам. Здесь $f(t)$ -- функция, называемая <<оригиналом>>, а $F(s)$ -- функция, называемая <<образом>>, или результатом применения преобразования Лапласа.

	Преобразование Лапласа является линейным, соответственно, удовлетворяет следующим условиям:
	\begin{enumerate}
		\item \( \Laplace{f + g} = \Laplace{f} + \Laplace{g} \);
		\item \( \Laplace{C \cdot f} = C \cdot \Laplace{f}, ~ \forall C - \const \);
	\end{enumerate}

	\subsection{Примеры}

		Рассмотрим пример $f(t) = 1 - u(\pi-t)$. Построим преобразование Лапласа по определению:
		\[ 
			\Laplace{f} 
				= \int\limits_{0}^{\infty} e^{-st} \cdot \bpares{1 - u(\pi-t)} ~ dt 
				= \int\limits_{0}^{\infty} e^{-st} ~ dt - \int\limits_{0}^{\infty} e^{-st} \cdot u(\pi-t) ~ dt 
				= \circledast 
		\]
		Первый интеграл тривиальный, в свою очередь, второй интеграл под действием функции $u(\pi-t)$ изменит пределы интегрирования на $t \in (0, \pi)$ в силу поведения функции $u(\pi-t)$ на диапазоне $(0, \infty)$:
		\[ \circledast 
				= -\frac{1}{s} e^{-st} \at_{t = 0}^{t \to \infty} - \int\limits_{0}^{\pi} e^{-st} ~ dt 
				= \circledast 
		\]
		Для сходимости интеграла, положим $s > 0$, тогда получим:
		\[ \circledast 
				= \frac{1}{s} + \frac{1}{s} e^{-st} \at_{0}^{\pi} 
				= \frac{1}{s} \pares{1 + e^{-\pi s} - 1} 
				= \frac{1}{s} e^{-\pi s}, s > 0. 
		\]

		\vspace{20pt}

		Построим решение в образах следующей задачи Коши для линейного дифференциального уравнения путем преобразования Лапласа:
		\[ y' + \delta(t-1) \cdot y = 1, ~ y(1) = 0, ~ y = y(t). \]
		Первым этапом приведем начальные условия к нормальному виду. Для этого сделаем замену $t = \tau + 1$. Тогда, при $t = 1$, $\tau = 0$. Соответственно, $y = y(t) = y(\tau+1) = y(\tau)$, $y'(t) = y'(\tau)$. Получим следующую задачу Коши:
		\[ y' + \delta(\tau) \cdot y = 1, ~ y(0) = 0, ~ y = y(\tau). \]
		Применим преобразование Лапласа на эту задачу:
		\[ \Laplace{y'} + \Laplace{\delta(\tau) \cdot y} = \Laplace{1}, ~ y(0) = 0. \]
		Получим:
		\[ \Bigl[ s \cdot Y(s) - y(0) \Bigr] + y(0) = \frac{1}{s}, \]
		Упростим, выразим $Y(s)$. Таким образом, решение в образах принимает такой вид:
		\[ Y(s) = \frac{1}{s^2}. \]
	
	% \pagebreak	