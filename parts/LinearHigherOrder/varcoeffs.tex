\section{Линейные уравнения с переменными коэффициентами}

	Рассматривать будем уравнения следующего вида:
	\[ a_0(x) \cdot y^{(n)} + a_1(x) \cdot y^{(n-1)} + \dots + a_{n-1}(x) \cdot y' + a_n(x) \cdot y = f(x). \]
	Уравнения, содержащие искомую функцию с ее производными в виде линейной комбинации с некоторыми известными функциями независимой переменной называются линейными уравнениями.

    Классификация таких уравнений, описанная в разделе линейных уравнений первого порядка, справедлива и для уравнений высших порядков.

    В данном разделе рассматривать будем однородные уравнения. Любая линейная комбинация частных решений однородного уравнения также является решением:
    \[ y_p = k_1 \cdot y_{p_1} + k_2 \cdot y_{p_2} + \dots + k_m \cdot y_{p_m}, ~ k_i - \const ~ \forall i = \overline{1, m}, ~ m \leq n. \]
    Фундаментальной системой решений (ФСР) будет называть множество всех возможных линейно-независимых частных решений однородного уравнения:
    \[ \Phi = \{ y_{p_1}, \dots, y_{p_n} \}. \]
    Система функций называется линейно-независимой, если определитель Вронского над заданной системой функций не равен нулю. Определителем Вронского системы функций называется определитель матрицы следующего вида:
    \[ 
    	\det{\bpares{W(y_{p_1}, \dots, y_{p_n})}} =
        \begin{vmatrix}
            y_{p_1} & y_{p_2} & \dots & y_{p_n} \\
            y'_{p_1} & y'_{p_2} & \dots & y'_{p_n} \\
            \vdots & \vdots & \ddots & \vdots \\
            y^{(n - 1)}_{p_1} & y^{(n - 1)}_{p_2} & \dots & y^{(n - 1)}_{p_n} \\
        \end{vmatrix}.
    \]
    Так-же, согласно теореме Абеля, определитель Вронского можно найти как решение соответствующего линейного уравнения первого порядка:
    \[ a_0(x) \cdot \difft{W}{x} + a_1(x) \cdot W = 0, \]
	где $W$ -- определитель Вронского рассматриваемого уравнения. Тогда для линейных уравнений высших порядков справедливо следующее соотношение:
	\[ W(y_{p_1}, \dots, y_{p_n}) = C \cdot \exp\pares{-\int \frac{a_1(x)}{a_0(x)} ~ dx}. \]
	Такая формула называется формулой Абеля, или формулой Лиувилля-Остроградского.

	В большинстве случаев для решения таких уравнений, некоторые частные решения приходится подбирать вручную. В классических задачах встречаются обычно следующие формы:
	\begin{enumerate}
		\item В виде полиномов:
			\[ y_p = x^n + \alpha_1 \cdot x^{n-1} + \dots + \alpha_{n-1} \cdot x + \alpha_n. \]
			Обычно при подборе решения в виде полинома, первым шагом находят самую старшую степень $n$:
			\[ y_p = x^n, \]
			и как только степень найдена, решение можно подбирать в общем виде уже для конкретной степени. В силу того, что подбирается частное решение однородного уравнения, коэффициент при старшей степени можно приравнять к $1$.

		\item В экспоненциально-тригонометрической форме:
			\[ y_p = e^{a x}, ~ a \in \mathbb{C}: ~ a = \alpha + i \beta. \]
			Некоторые частные случаи -- \( y_p = e^{\alpha x}, ~ y_p = \cos{\beta x}, ~ y_p = \sin{\beta x}, ~ y_p = e^{\alpha x} \cdot \cos{\beta x} \) или $y_p = e^{\alpha x} \cdot \sin{\beta x}$.

		\item По структуре уравнения:
			\[ y_p = P(x), \]
			где $P$ -- некоторая произвольная функция. Ее можно подобрать, к примеру, из вида уравнения. Если в уравнении среди коэффициентов присутствуют обратные функции (к примеру -- $\ln{x}$ или $\arcsin{x}$), то возможно в частном решении также будет присутствовать данная функция с тем же аргументом.
	\end{enumerate}

	Зная хотя бы одно частное решение однородного уравнения, порядок исходного можно понизить с помощью следующей замены:
	\[ y = y_{p_1}(x) \cdot u, \]
	где $y_{p_1}$ -- частное решение однородного уравнения, $u$ -- новая искомая функция.

	\subsection{Примеры}

		\begin{enumerate}
			\item Рассмотрим следующий пример:
				\[ xy'' \cdot \pares{x + 1} + \pares{x - 1} \cdot y' = y. \]
				Классифицируем уравнение: линейное неприведенное однородное уравнение второго порядка с переменными коэффициентами. Так как коэффициенты в уравнении представляют собой некоторые степенные функции, частное решение будем искать также в виде степенной функции:
				\[ y_{p_1} = x^n. \]
				Дифференцируем:
				\[ y_{p_1}' = n \cdot x^{n-1}, ~ y_{p_1}'' = n \pares{n-1} \cdot x^{n-2}. \]
				Подставим в уравнение:
				\[ n \pares{n-1} \cdot x^{n-1} \cdot \pares{x + 1} + n \cdot x^{n-1} \cdot \pares{x - 1} - x^n = 0. \]
				Самая старшая степень $x$ в данном уравнении -- $n$. Выпишем коэффициенты при самых старших степенях $x$:
				\[ n \pares{n - 1} + n - 1 = 0. \]
				Упростим и найдем значение $n$:
				\[ n^2 - 1 = 0 \implies n = \pm 1. \]
				Попробуем построить решение в виде полинома первой степени (при $n = 1$):
				\[ y_{p_1} = x + a. \]
				Найдем коэффициент $a$. Для этого продифференцируем:
				\[ y_{p_1}' = 1, ~ y_{p_1}'' = 0, \]
				и подставим в исходное уравнение:
				\[ x - 1 + x + a = 0 \implies a = -1. \]
				Тогда частное решение уравнения можно записать в следующем виде:
				\[ y_{p_1} = x - 1. \]
				Найти общее общее решение теперь можно двумя способами:
				\begin{enumerate}
					\item С помощью формулы Лиувилля-Остроградского. Зная одно частное решение, построим определитель Вронского:
						\[ W(y_{p_1}, y_{p_2}) = \begin{vmatrix}
							x - 1 & y_{p_2} \\
							1 & y_{p_2}'
						\end{vmatrix} = (x - 1) \cdot y_{p_2}' - y_{p_2}. \]
						Тогда, согласно формуле:
						\[ W(y_{p_1}, y_{p_2}) = C \cdot \exp\pares{-\int \frac{a_1(x)}{a_0(x)} ~ dx}, \]
						подставляя известные значения, получим:
						\[ (x - 1) \cdot y_{p_2}' - y_{p_2} = C \cdot \exp\pares{- \int \frac{x - 1}{x \pares{x + 1}} ~ dx}, \]
						или, упрощая:
						\[ (x - 1) \cdot y_{p_2}' - y_{p_2} = \frac{Cx}{\pares{x + 1}^2}. \]
						Тогда общее решение такого уравнения имеет вид:
						\[ y_{p_2} = C_1 \pares{x - 1} - \frac{C}{2 \pares{x + 1}}. \]
						В силу произвольности $C$, упростим:
						\[ y_{p_2} = C_1 \pares{x - 1} + \frac{C_2}{x + 1} \]
						Полученное выражение представляет собой линейную комбинацию двух частных решений однородного уравнения. А так как для однородных линейных уравнений второго порядка ФСР состоит из двух линейно-независимых частных решений, то их линейная комбинация образует общее решение:
						\[ y = C_1 \pares{x - 1} + \frac{C_2}{x + 1}. \]

					\item С помощью понижения порядка в линейном однородном уравнении:
						\[ y = y_{p_1}(x) \cdot u = \pares{x - 1} \cdot u, ~ u = u(x). \]
						Полагая $u$ -- новой искомой функцией, найдем соответствующие производные:
						\[ y' = \pares{x - 1} \cdot u' + u, ~ y'' = \pares{x - 1} \cdot u'' + 2u'. \]
						Подставим в исходное уравнение:
						\[ x \cdot \pares{x^2 - 1} \cdot u'' + \pares{x - 1}^2 \cdot u' + 2 x \cdot \pares{x + 1} \cdot u' + \pares{x - 1} \cdot u = \pares{x - 1} \cdot u. \]
						Упростим:
						\[ x \pares{x^2 - 1} \cdot u'' + \pares{3x^2 + 1} \cdot u' = 0. \]
						Получили уравнение, приводимое к уравнению с разделяющимися переменными путем понижения порядка заменой $v = u', ~ v = v(x)$. Решение этого уравнения:
						\[ u = \frac{C_1}{x^2 - 1} + C_2. \]
						Возвращаясь к исходной замене, получим общее решение:
						\[ y = \frac{C_1}{x + 1} + C_2 \pares{x - 1}. \] 

				\end{enumerate}
				Можно заметить, что частное решение $y_p = \frac{1}{x - 1}$ вполне может удовлетворять виду степенной функции со степенью $n = -1$.
		
			\item Рассмотрим еще один пример:
				\[ \pares{x - 1} \cdot y'' - \pares{x + 1} \cdot y' + 2y = 0. \]
				Классифицируем уравнение: линейное неприведенное однородное уравнение второго порядка с переменными коэффициентами. Так как коэффициенты в уравнении представляют собой некоторые степенные функции, частное решение будем искать также в виде степенной функции:
				\[ y_p = x^n. \]
				Дифференцируем:
				\[ y' = n \cdot x^{n-1}, ~ y'' = n \pares{n-1} \cdot x^{n-2}. \]
				Подставим в уравнение:
				\[ n\pares{n - 1} \cdot \pares{x - 1} \cdot x^{n-2} - n \cdot \pares{x + 1} \cdot x^{n-1} + 2 x^{n} = 0. \]
				Самая старшая степень $x$ в данном уравнении -- $n$. Выпишем коэффициенты при самых старших степенях $x$:
				\[ -n + 2 = 0 \implies n = 2. \]
				Построим частное решение в виде полинома второй степени:
				\[ y_{p_1} = x^2 + ax + b. \]
				Продифференцируем:
				\[ y_{p_1}' = 2x + a, ~ y_{p_1}'' = 2. \]
				Подставим в исходное уравнение для того, чтобы найти коэффициенты $a$ и $b$:
				\[ 2 \pares{x - 1} - \pares{x + 1} \cdot \pares{2x + a} + 2 \pares{x^2 + ax + b} = 0. \]
				Упрощая, получим:
				\[ -ax + \pares{2b - a - 2} = 0 + 0 \cdot x. \]
				Сопоставляя коэффициенты при степенях $x$, получим, что $a = 0$, $b = 1$. Тогда частное решение имеет вид:
				\[ y_{p_1} = x^2 + 1. \]
				Второе частное решение также подберем. Но так как степени для полинома исчерпаны, будем искать частное решение в виде функции. Ничего в уравнении не подсказывает о том, какая могла бы быть функция, поэтому попробуем по порядку -- экспоненциальную:
				\[ y_{p_2} = e^{ax}, \]
				тогда:
				\[ y_{p_2}' = a \cdot e^{ax}, ~ y_{p_2}'' = a^2 \cdot e^{ax}. \]
				Подставим в уравнение:
				\[ a^2 \cdot \pares{x - 1} \cdot e^{ax} - a \cdot \pares{x + 1} \cdot e^{ax} + 2e^{ax} = 0. \]
				Здесь можно вынести $e^{ax}$ как общий сомножитель, и так как $e^{ax} \neq 0$, сократить:
				\[ a^2 \cdot \pares{x - 1} - a \cdot \pares{x + 1} + 2 = 0. \]
				Выпишем коэффициенты для каждой степени $x$:
				\[ a \cdot \pares{a - 1} \cdot x - a \cdot \pares{a + 1} = -2 + 0 \cdot x. \]
				Тогда $a \cdot \pares{a - 1} = 0$ и $a \cdot \pares{a + 1} = 2$. Здесь $a \neq 0$, тогда, чтобы оба уравнения выполнялись, видно, что подходит $a = 1$. Тогда второе частное решение исходного однородного уравнения имеет вид:
				\[ y_{p_2} = e^{x}. \]
				Проверим, что данные два решения линейно-независимы. Для этого вычислим определитель Вронского, и сравним его с нулем:
				\[ W(y_{p_1}, y_{p_2}) = \begin{vmatrix}
					x^2 + 1 & e^{x} \\ 2x & e^x
				\end{vmatrix} = \pares{x - 1}^2 \cdot e^{x} \not\equiv 0. \]
				Таким образом получили, что данные два решения линейно независимы, а значит составляют ФСР исходного уравнения. Тогда общее решение исходного уравнения можно записать в следующем виде:
				\[ y = C_1 \pares{x^2 + 1} + C_2 e^{x}. \]

		\end{enumerate}

	\pagebreak