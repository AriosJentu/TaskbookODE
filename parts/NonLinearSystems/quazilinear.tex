\section{Линейные и квазилинейные уравнения с частными производными}

	Дифференциальным уравнением с частными производными первого порядка в общем случае называется уравнение вида
	\[ F \pares{x_1, \dots, x_n, u, \dpart{u}{x_1}, \dots, \dpart{u}{x_n}} = 0, ~ u = u\pares{x_1, \dots, x_n}. \]
	В векторной форме, уравнение имеет вид:
	\[ F \pares{\vec{x}, u, \grad{u}} = 0, ~ u = u\pares{\vec{x}}. \] 

	В качестве простейшего примера уравнений с частными производными можно привести следующее:
	\[ \dpart{u}{x} = f(x, y), ~ u = u(x, y). \]
	Данное уравнение называется уравнением, допускающим разделение переменных (или допускающим интегрирование), полагая $y$ -- параметром. Рассматривать его будем как обыкновенное дифференциальное уравнение относительно переменных $u$ и $x$. Решение такого уравнения строится путем интегрирования обеих частей по переменной $x$:
	\[ u = \int f(x, y) ~ \partial x = F(x, y) + C_1(y). \]
	Так как уравнение интегрировалось только по переменной $x$ (положив $y$ параметром), то любая произвольная постоянная будет являться в таком случае параметрической функцией. Её аргументом будут выступать те переменные, которые в процессе интегрирования не участвовали.

	\vspace{15pt}

	Линейным уравнением с частными производными называется уравнение вида:
	\[ \sum_{k = 1}^{n} A_k \pares{\vec{x}} \cdot \dpart{u}{x_k} = B\pares{\vec{x}} \cdot u + f\pares{\vec{x}}. \]
	В векторной форме, полагая $\vec{A} \pares{\vec{x}}$ -- вектор с компонентами $A_k \pares{\vec{x}}, ~ k = \overline{1, n}$ такое уравнение принимает вид:
	\[ \vec{A}\pares{\vec{x}} \cdot \grad{u} = B \pares{\vec{x}} \cdot u + f \pares{\vec{x}}, \]
	где левая часть уравнения представляет собой скалярное произведение двух векторных функций.

	\vspace{15pt}

	Для таких уравнений возможно составить характеристику, описываемую в следующем виде:
	\begin{enumerate}
		\item Если \( B\pares{\vec{x}} \neq 0 \) -- уравнение является полным, иначе -- неполное;
		\item Если \( f\pares{\vec{x}} = 0 \) -- уравнение является однородным, иначе -- неоднородное;
		\item Если \( A_k - \const ~ \forall k = \overline{1, n} \) -- уравнение является уравнением с постоянными коэффициентами. Если \( \exists k \in \overline{1, n} : A_k = A_k\pares{\vec{x}} \neq \const \) -- уравнение является уравнением с переменными коэффициентами;
		\item Если \( \exists k \in \overline{1, n} : A_k = A_k \pares{\vec{x}, u} \) -- уравнение называется квазилинейным.
	\end{enumerate}

	Для линейных уравнений также характерен принцип суперпозиции:
	\[ u = \sum_{k = 1}^{n} C_k u_k + u_{p}, \]
	где \( u_k \) -- частные решения однородного уравнения, а \( u_p \) -- частное решение неоднородного уравнения.

	\vspace{15pt}

	Метод решения будем рассматривать на примере квазилинейных уравнений, так как он является общим для таких типов уравнений. Квазилинейным уравнением с частными производными называется уравнение вида
	\[ \sum_{k = 1}^{n} A_k \pares{\vec{x}, u} \cdot \dpart{u}{x_k} = B \pares{\vec{x}, u}. \]

	Решение такого уравнения строится вдоль его функций-характеристик. Соответствующей симметрической характеристической системой квазилинейного дифференциального уравнения называется система вида следующего вида:
	\[ \frac{dx_1}{A_1} = \frac{dx_2}{A_2} = \dots = \frac{dx_n}{A_n} = \frac{du}{B}. \]
	Решения этой системы удовлетворяют следующему свойству:
	\[ \vp_k \pares{\vec{x}, u} = C_k - \const, ~ k = \overline{1, n}. \] 
	Здесь функции $\vp_k$ являются решениями симметрической системы, и называются функциями-характеристиками уравнения с частными производными. Общее решение линейного и квазилинейного уравнения с частными производными первого порядка будет содержать одну произвольную функцию от известных характеристических решений. Общее решение принимает вид:
	\[ \Phi \pares{\vp_1, \dots, \vp_n} = 0, \]
	где $\Phi$ -- произвольная функция.

	\vspace{15pt}
	
	Функция-характеристика так же носит название <<первый интеграл>> уравнения с частными производными, а совокупность первых интегралов дает систему, называемую <<полным интегралом>>.
	
	\subsection{Примеры}

		В качестве примера рассмотрим следующее уравнение
		\[ 2u^2 \pares{ \dpart{u}{x} - 2x \dpart{u}{y}} = \pares{ 8uxy + z } \dpart{u}{z} + u. \]

		Для начала охарактеризуем это уравнение. Это неприведенное квазилинейное дифференциальное уравнение с частными производными первого порядка. Приведем его к каноническому виду:
		\[ 2u^2 \dpart{u}{x} - 4 u^2 x \dpart{u}{y} - \pares{ 8uxy + z } \dpart{u}{z} = u. \]
		Построим соответствующую симметрическую характеристическую систему для данного уравнения:
		\[ \frac{dx}{2u^2} = \frac{dy}{-4 u^2 x} = \frac{dz}{-8uxy - z} = \frac{du}{u}. \]
		Данную симметрическую систему можно свести к системе из трех уравнений. Выберем возможные интегрируемые комбинации:
		\[ \left\lbrace \begin{split}
			\frac{dx}{2u^2} &= \frac{du}{u}; \\
			\frac{dx}{2u^2} &= \frac{dy}{-4u^2 x}; \\
			\frac{du}{u} &= \frac{u dz + z du - 2y dy}{- 8 u^2 xy - zu + zu + 8 u^2 x y}.
		\end{split} \right. \] 
		Полагаем в третьем уравнении в знаменателе $0$, а решение постоянным вдоль этой характеристики (третье уравнение было построено на основе принципа равенства дробей, соответственно такой вариант вполне действителен). После упрощений, получим следующую систему:
		\[ \left\lbrace \begin{split}
			2u du - dx &= 0; \\
			2x dx + dy &= 0; \\
			u dz + z du - 2y dy &= 0.
		\end{split} \right. \] 
		Интегрируя каждое уравнение этой системы, получим три решения:
		\[ u^2 - x = C_1, \quad x^2 + y = C_2, \quad uz - y^2 = C_3, \]
		из которых следуют три характеристические функции:
		\[ \vp_1 \pares{u, x, y, z} = u^2 - x, \quad \vp_2 \pares{u, x, y, z} = x^2 + y, \quad \vp_3 \pares{u, x, y, z} = uz - y^2. \] 
		Тогда общее решение рассматриваемого уравнения имеет вид:
		\[ \Phi \pares{u^2 - x, x^2 + y, uz - y^2} = 0. \]
		Проверим, действительно ли оно является решением. Для этого, воспользуемся следующей техникой. Найдем соответствующие частные производные полученного решения вдоль переменных $x$, $y$, $z$:
		\[ \left\lbrace \begin{split}
			\dpart{\Phi}{x} 
				&= \sum_{k = 1}^{3} \dpart{\Phi}{\vp_k} \cdot \dpart{\vp_k}{x} 
					= \dpart{\Phi}{\vp_1} \cdot \pares{2u \dpart{u}{x} - 1} 
					+ \dpart{\Phi}{\vp_2} \cdot 2x 
					+ \dpart{\Phi}{\vp_3} \cdot z \dpart{u}{x}
			; \\
			\dpart{\Phi}{y} 
				&= \sum_{k = 1}^{3} \dpart{\Phi}{\vp_k} \cdot \dpart{\vp_k}{y} 
					= \dpart{\Phi}{\vp_1} \cdot 2u \dpart{u}{y} 
					+ \dpart{\Phi}{\vp_2} 
					+ \dpart{\Phi}{\vp_3} \cdot \pares{z \dpart{u}{y} - 2y}
			; \\
			\dpart{\Phi}{z} 
				&= \sum_{k = 1}^{3} \dpart{\Phi}{\vp_k} \cdot \dpart{\vp_k}{z} 
					= \dpart{\Phi}{\vp_1} \cdot 2u \dpart{u}{z} 
					+ \dpart{\Phi}{\vp_3} \cdot \pares{z \dpart{u}{y} + u}
			.
		\end{split} \right. \] 
		
		Из общего решения следует, что
		\[ \dpart{\Phi}{x} = \dpart{\Phi}{y} = \dpart{\Phi}{z} = 0, \]
		тогда выразим из каждого уравнения \( \dpart{u}{x} \), \( \dpart{u}{y} \) и \( \dpart{u}{z} \) соответственно:
		\[
			\dpart{u}{x} = \frac{\dpart{\Phi}{\vp_1} - 2x \dpart{\Phi}{\vp_2}}{2u \dpart{\Phi}{\vp_1} + z \dpart{\Phi}{\vp_3}}, ~
			\dpart{u}{y} = \frac{2y \dpart{\Phi}{\vp_3} - \dpart{\Phi}{\vp_2}}{2u \dpart{\Phi}{\vp_1} + z \dpart{\Phi}{\vp_3}}, ~
			\dpart{u}{z} = \frac{- u \dpart{\Phi}{\vp_3}}{2u \dpart{\Phi}{\vp_1} + z \dpart{\Phi}{\vp_3}}.
		\] 
		Подставим известные выражения в исходное уравнение, домножив на общий знаменатель:
		\[ \begin{split}
			2u^2 &\cdot \pares{
				\dpart{\Phi}{\vp_1} - 2x \dpart{\Phi}{\vp_2}
			} - 4 u^2 x \cdot \pares{
				2y \dpart{\Phi}{\vp_3} - \dpart{\Phi}{\vp_2}
			} + \pares{ 8uxy + z } \cdot 
				u \dpart{\Phi}{\vp_3} 
			= \\
			&= u \cdot \pares{
				2u \dpart{\Phi}{\vp_1} + z \dpart{\Phi}{\vp_3}
			}. 
		\end{split} \]
		Вынесем производные за скобки, получим:
		\[ \dpart{\Phi}{\vp_1} \cdot \pares{2u^2 - 2u^2} + \dpart{\Phi}{\vp_2} \cdot \pares{-4u^2 x + 4u^2 x} + \dpart{\Phi}{\vp_3} \cdot \pares{- 8u^2 x y + 8u^2 x y + uz - uz} = 0. \]
		Равенство выполняется, соответственно представленная функция является решением для любого $\Phi \pares{\vp_1, \vp_2, \vp_3}$, где $\vp_k$ -- функции-характеристики.

	% \pagebreak