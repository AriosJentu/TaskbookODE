\section{Системы уравнений с частными производными первого порядка}

	Рассмотрим системы дифференциальных уравнений относительно одной неизвестной функции, разрешенные относительно производной. Они имеют следующий вид:
	\[ \left\lbrace \begin{split} 
		\dpart{u}{x_1} &= f_1 \pares{\vec{x}, u}, \\
		&\vdots \\
		\dpart{u}{x_n} &= f_n \pares{\vec{x}, u}.
	\end{split} \right. \]

	В векторной форме такие уравнения представляют собой следующий вид:
	\[ \grad{u} = \vec{f}\pares{\vec{x}, u}. \]

	Если решение системы существует, то оно описывает однопараметрическое семейство некоторых $n$-мерных поверхностей в $n+1$-мерном пространстве ($n$ независимых переменных и одна зависимая). Решение такой системы существует и единственно в случае, если система уравнений является совместной. Если система совместна, то она считается вполне интегрируемой. Для проверки условия совместности используется классический критерий интегрируемости системы, основанный на принципе равенства смешанных производных:
	\[ \dpartmix{u}{x_i}{x_j} = \dpartmix{u}{x_j}{x_i} ~ \forall i, j = \overline{1, n}, ~ i \neq j. \] 
	Таким образом критерий интегрируемости исходной системы принимает следующий вид:
	\[ \bracs{ \dpart{f_i}{x_j} = \dpart{f_j}{x_i} } ~ \forall i, j = \overline{1, n}, ~ i \neq j. \]
	Для системы из двух уравнений, критерий интегрируемости состоит из одного уравнения. Для системы из трех уравнений, критерий состоит из трех уравнений, так как содержит в себе все возможные различные пары производных по аргументам.

	Рассмотрим случай, когда критерий интегрируемости выполняется. Тогда решение уравнения можно построить на основе метода соответствий функций. Каждое уравнение рассматривается отдельно, строится его общее решение, затем составляется система решений каждого уравнения:
	\[ \left\lbrace \begin{split}
		F_1 & \pares{\vec{x}, u, C_1\pares{\vec{x} \setminus \bracs{x_1}}\phn} = 0, \\
		& \vdots \\
		F_n & \pares{\vec{x}, u, C_n\pares{\vec{x} \setminus \bracs{x_n}}\phn} = 0. \\
	\end{split} \right. \]
	Так как данная система решений удовлетворяет одной и той же поверхности, то данные функции должны быть равны. Приводя функции к общему виду, и сопоставляя произвольные функции $C_1, \cdots, C_n$, получим общее решение данной системы:
	\[ F\pares{\vec{x}, u, C} = 0. \]

	В случае, если критерий интегрируемости не выполняется, решения системы может или не существовать, или существовать бесконечное множество параметрических решений (для такого решения критерий интегрируемости может выполняться \textit{искусственно}). Тогда можно воспользоваться обобщенным методом решения систем путем сведения их к квазилинейному уравнению первого порядка. Для этого домножим каждое уравнение на соответствующие им функции $A_k\pares{\vec{x}, u} \neq 0$ (выбираются произвольно), и сложим получившийся результат. Получим:
	\[ \sum_{k=1}^n A_k \pares{\vec{x}, u} \cdot \dpart{u}{x_k} = B\pares{\vec{x}, u}, ~ B\pares{\vec{x}, u} = \sum_{k=1}^n A_k \pares{\vec{x}, u} \cdot f_k \pares{\vec{x}, u}, \]
	или, в векторной форме, полагая \( \vec{A} \pares{\vec{x}, u} \) -- вектор с компонентами $A_k\pares{\vec{x}, u}, ~ k = \overline{1, n}$:
	\[ \vec{A} \pares{\vec{x}, u} \cdot \grad{u} = \vec{A} \pares{\vec{x}, u} \cdot \vec{f} \pares{\vec{x}, u}. \]

	В результате интегрирования этого уравнения, получим некоторую функцию:
	\[ \Phi\pares{\vp_1, \dots, \vp_n} = 0, \]
	где $\varphi_k = \varphi_k \pares{\vec{x}, u}$. Подставим в исходную систему, и найдем вид функции $\Phi$ в явном виде.

	\subsection{Примеры}

		\begin{enumerate}

			\item Рассмотрим следующий пример:
				\[ \dpart{u}{x} = \frac{u + x}{u - x}, ~ \dpart{u}{y} = - \frac{y}{u - x}. \] % u**2 - 2*u*x - x**2 + y**2 = C

				Для начала проверим критерий интегрируемости данной системы. При дифференцировании не стоит забывать, что $u$ так же считается функцией переменных $x, y$. Распишем частные производные:
				\[ \begin{split} 
					\dpartmix{u}{x}{y} &= \dpart{}{y} \pares{\frac{u + x}{u - x}} = \frac{1}{\pares{u - x}^2} \pares{ \pares{u - x} \cdot \dpart{u}{y} - \pares{u + x} \dpart{u}{y} } = \\ 
					&= -\frac{2x}{\pares{u - x}^2} \dpart{u}{y} = \frac{2xy}{\pares{u - x}^3};
				\end{split} \]
				\[ \begin{split} 
					\dpartmix{u}{y}{x} &= \dpart{}{x} \pares{-\frac{y}{u - x}} = \frac{y}{\pares{u - x}^2} \cdot \pares{\dpart{u}{x} - 1} = \\ 
					&= \frac{y}{\pares{u - x}^2} \cdot \pares{ \frac{u + x}{u - x} - 1 } = \frac{2xy}{\pares{u - x}^2}.
				\end{split} \]

				Смешанные производные функции $u$ равны между собой, следовательно критерий интегрируемости выпоняется, а значит существует единственное однопараметрическое решение данной системы. Построим его согласно метода соответствий. Для этого, решим каждое уравнение отдельно. Второе уравнение системы является уравнением, допускающим разделение переменных, полагая $x$ -- параметром:
				\[ \dpart{u}{y} = - \frac{y}{u - x} \implies \int \pares{u - x} ~ \partial u = - \int y \partial y \implies u^2 - 2ux + y^2 = C_2 \pares{x}. \]
				На данном этапе можно продифференцировать результат по $x$, и подставить в первое уравнение, тем самым находя уравнение для функции $C_2 \pares{x}$, а можно построить решение первого уравнения. Сделаем замену в первом уравнении $v = u - x$, тогда $\dpart{u}{x} = \dpart{v}{x} + 1$, и:
				\[ \dpart{v}{x} + 1 = \frac{v + 2x}{v} \implies \dpart{v}{x} = \frac{2x}{v} \implies v^2 = 2x^2 + C_2 \pares{y}. \]
				Возвращаясь к исходной замене, получим:
				\[ u^2 - 2ux + x^2 = 2x^2 + C_1\pares{y}. \]

				Запишем систему решений для построения соответствий:
				\[ \left\lbrace \begin{split}
					u^2 - 2ux &= x^2 + C_1 \pares{y}, \\
					u^2 - 2ux &= -y^2 + C_2 \pares{x}.
				\end{split} \right. \]
				Из этой системы следует, что $C_1 \pares{y} = -y^2$, $C_2 \pares{x} = x^2$, и, соответствие между функциями $C_1 \pares{y} = C_2 \pares{x}$ возможно, когда $C_1 \pares{y} = C_2 \pares{x} = C$. Тогда поверхность, удовлетворяющая уравнению имеет следующий вид:
				\[ u^2 - 2ux = x^2 - y^2 + C. \] 

			\item Рассмотрим другое уравнение:
				\[ \dpart{u}{x} = 2ux, ~ \dpart{u}{y} = 2uy. \]
				Проверим критерий интегрируемости данного уравнения:
				\[ \dpartmix{u}{x}{y} = 2x \dpart{u}{u} = 4uxy, ~ \dpartmix{u}{y}{x} = 2y \dpart{u}{x} = 4uxy. \]
				Смешанные производные равны, а значит система совместна, и существует решение. Но в этот раз для решения воспользуемся методом сведения системы уравнений к квазилинейному уравнению первого порядка. Попробуем получить простое уравнение. Для этого домножим первое уравнение на $y$, второе на $x$, и вычтем одно из другого. Получим:
				\[ y \dpart{u}{x} - x \dpart{u}{y} = 0 \]
				Его общее решение имеет следующий вид:
				\[ u = \Phi\pares{x^2 + y^2}. \]
				Переобозначим за $t(x, y) = x^2 + y^2$, тогда $u = \Phi(t)$. Найдем частные производные:
				\[ \dpart{u}{x} = \difft{\Phi}{t} \cdot \dpart{t}{x} = 2x \cdot \difft{\Phi}{t}, ~ \dpart{u}{y} = \difft{\Phi}{t} \cdot \dpart{t}{y} = 2y \cdot \difft{\Phi}{t}. \]
				Подставим в исходную систему, полагая $u = \Phi(t)$. Получим:
				\[ 2x \cdot \difft{\Phi}{t} = 2x \Phi, ~ 2y \cdot \difft{\Phi}{t} = 2y \Phi. \]
				Данная система уравнений является линейно-зависимой. Из нее следует уравнение вида:
				\[ \difft{\Phi}{t} = \Phi. \]
				Тогда функция $\Phi(t)$ принимает следующий вид:
				\[ \Phi(t) = C e^{t} \quad \forall C, t. \]
				И, подставляя это выражение в решение квазилинейного уравнения, получим решение исходной системы:
				\[ u = C e^{x^2 + y^2}. \]

			% \item Построим решение методом сведения системы к уравнению первого порядка. В данном случае удобно каждое уравнение домножить на одну и ту же функцию $A_k(u, x, y) = u - x$. Вычитая второе из первого, получим следующее уравнение:
				% \[ \pares{u - x} \dpart{u}{x} - \pares{u - x} \dpart{u}{y} = u + x + y. \]
				% Составим соответствующую симметрическую систему:
				% \[ \frac{dx}{u - x} = -\frac{dy}{u - x} = \frac{du}{u + x + y}. \]
				% Выберем комбинацию из первого и второго выражения, получим:
				% \[ x + y = C_1. \]
				% Рассмотрим первое и третье выражение, полагая, что $x + y = C_1$ из первого соотношения:
				% \[ \frac{dx}{u - x} = \frac{du}{u + C_1}. \]
				% Приведем это уравнение к линейному относительно $x$:
				% \[ \difft{x}{u} = \frac{u - x}{u + C_1} \implies 2x \pares{u + C_1} - u^2 = C_2, \]
				% или
				% \[ u^2 - 2ux - 2x^2 - 2xy = C_2. \]
				% Тогда общее решение уравнения принимает вид:
				% \[ \Phi\pares{x + y, u^2 - 2ux - 2x^2 - 2xy} = 0 \]
				% Найдем $\dpart{u}{x}$ и $\dpart{u}{y}$. Для этого найдем все производные функции $\Phi$:
				% \[ \dpart{\Phi}{u} = \dpart{\Phi}{C_1} \dpart{C_1}{u} + \dpart{\Phi}{C_2} \dpart{C_2}{u} = \dpart{\Phi}{C_2} \cdot \pares{}, \]
				% \[ \dpart{\Phi}{x} = \dpart{\Phi}{C_1} \dpart{C_1}{x} + \dpart{\Phi}{C_2} \dpart{C_2}{x} = \dpart{\Phi}{C_1} - \dpart{\Phi}{C_2} \cdot \pares{2u + 4x + 2y}, \]
				% \[ \dpart{\Phi}{y} = \dpart{\Phi}{C_1} \dpart{C_1}{y} + \dpart{\Phi}{C_2} \dpart{C_2}{y} = \dpart{\Phi}{C_1} - 2x \dpart{\Phi}{C_2}. \]
		
		\end{enumerate}

	% \pagebreak