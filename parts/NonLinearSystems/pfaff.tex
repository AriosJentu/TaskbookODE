\section{Уравнения Пфаффа, интегрирующий множитель}

	Рассмотрим обобщенные уравнения в полных дифференциалах. Введем для них критерии интегрируемости и рассмотрим способы их интегрирования.

	Пусть \( \vec{x} = \begin{pmatrix} x_1 \\ \vdots \\ x_n \end{pmatrix} \) -- вектор независимых переменных, \( \vec{P}\pares{\vec{x}} = \begin{pmatrix} P_1\pares{\vec{x}} \\ \vdots \\ P_n\pares{\vec{x}} \end{pmatrix} \) -- вектор известных функций $n$-переменных. Дифференциальным уравнением Пфаффа называется уравнение следующего вида:
	\[ \vec{P}\pares{\vec{x}} \cdot d\vec{x} = 0, \]
	где $\circledast \cdot \circledast$ -- скалярное произведение, \( d\vec{x} = \begin{pmatrix} dx_1 \\ \vdots \\ dx_n \end{pmatrix} \). Раскрывая скалярное произведение, уравнение можно переписать в следующем виде:
	\[ P_1\pares{\vec{x}} ~ dx_1 + \ldots + P_n\pares{\vec{x}} ~ dx_n = 0. \]

	Уравнение Пфаффа может быть вполне-интегрируемым, если выполняется критерий интегрируемости. Положим, что левая часть уравнения Пфаффа является полным дифференциалом некоторой непрерывной со своими первыми производными функции $u\pares{\vec{x}}$, где
	\[ du = \dpart{u}{x_1} ~ dx_1 + \ldots + \dpart{u}{x_n} ~ dx_n. \]
	Тогда, в силу предположения, построим соответствия с исходным уравнение:
	\[ \dpart{u}{x_1} = P_1\pares{\vec{x}}, \dots, \dpart{u}{x_n} = P_n\pares{\vec{x}}. \]
	Для построения критерия интегрируемости, воспользуемся признаком равенства смешанных производных:
	\[ \dpartmix{u}{x_i}{x_j} = \dpartmix{u}{x_j}{x_i} ~ \forall i, j = \overline{1, n}, ~ i \neq j. \]
	Полагая, что \( \displaystyle P_i\pares{\vec{x}} = \dpart{u}{x_i}, ~ P_j\pares{\vec{x}} = \dpart{u}{x_j} \), и объединив все равенства в совокупность условий, получим следующий критерий интегрируемости:
	\[ \bracs{ \dpart{P_i}{x_j} = \dpart{P_j}{x_i} } ~ \forall i, j = \overline{1, n}, ~ i \neq j. \]
	Если в полученном критерии все равенства выполняются, то уравнение Пфаффа является вполне-интегрируемым, и представляет из себя полный дифференциал некоторой функции. Положим, что критерий выполнен. Снова введем функцию $u\pares{\vec{x}}$, и составим систему соответствий:
	\[ \left\lbrace \begin{split} \dpart{u}{x_1} =& P_1\pares{\vec{x}}; \\ &\vdots \\ \dpart{u}{x_n} =& P_n\pares{\vec{x}}. \end{split} \right. \]
	Данная система представляет из себя систему уравнений с частными производную, допускающую разделение переменных, которую можно решить путем интегрирования каждого $k$-го уравнения по переменной $x_k$:
	\[ \left\lbrace 
	\begin{split} 
		u =& \int P_1\pares{\vec{x}} ~ \partial x_1 + \varphi_1 \pares{\vec{x} \setminus \bracs{x_1}}; \\ 
		&\vdots \\ 
		u =& \int P_n\pares{\vec{x}} ~ \partial x_n + \varphi_n \pares{\vec{x} \setminus \bracs{x_n}}. 
	\end{split} 
	\right. \]
	Здесь $\varphi_k \pares{\vec{x} \setminus \bracs{x_k}}, ~ k = \overline{1, n}$ -- произвольные функции $(n-1)$–й переменной. Из этой системы можно получить общий вид функции $u\pares{\vec{x}}$ с помощью метода соответствий: \( u\pares{\vec{x}} = F\pares{\vec{x}} + C_1 \). Но эта функция получена в следствии решения следующего уравнения:
	\[ du = P_1\pares{\vec{x}} ~ dx_1 + \ldots + P_n\pares{\vec{x}} ~ dx_n, \]
	в свою же очередь, исходное уравнение представляет из себя уравнение $du = 0$, из чего $u\pares{\vec{x}} = C_2$. Тогда:
	\[ \left\lbrace \begin{split} u &= F\pares{\vec{x}} + C_1 \\ u &= C_2 \end{split} \right. \implies F\pares{\vec{x}} = C - \text{общее решение}. \]

	\vspace{10pt}
	Но если критерий интегрируемости не выполняется, то решение можно получить также с помощью интегрирующего множителя $\mu = \mu\pares{\vec{x}}$, если он существует. Суть интегрирующего множителя заключается в том, что при домножении всего уравнения на него, выполняется критерий интегрируемости:
	\[ \bracs{ \dpart{\pares{\mu P_i}}{x_j} = \dpart{\pares{\mu P_j}}{x_i} } ~ \forall i, j = \overline{1, n}, ~ i \neq j. \]
	Раскрывая производные, и сводя каждое уравнение к каноническому виду линейного полного однородного уравнения с частными производными первого порядка, получим систему уравнений с частными производными относительно интегрирующего множителя:
	\[ \bracs{ P_j \dpart{\mu}{x_i} - P_i \dpart{\mu}{x_j} = \mu \pares{\dpart{P_i}{x_j} - \dpart{P_j}{x_i}} } ~ \forall i, j = \overline{1, n}, ~ i \neq j. \]

	\vspace{10pt}
	Далее будет вывод критерия существования интегрирующего множителя в трехмерном случае, случаи высших порядков аналогичны, но могут содержать более одного параметра в решении.

	Рассмотрим уравнение
	\[ P_x (x, y, z) ~ dx + P_y (x, y, z) ~ dy + P_z (x, y, z) ~ dz = 0, \]
	для которого не выполняется хотя бы одно равенство в критерии интегрируемости:
	\[ \dpart{P_x}{y} \neq \dpart{P_y}{x}, ~ \dpart{P_x}{z} \neq \dpart{P_z}{x}, ~ \dpart{P_y}{z} \neq \dpart{P_z}{y}. \]
	Положим, что существует интегрирующий множитель $\mu = \mu(x, y, z)$. Составим для интегрирующего множителя систему уравнений с частными производными:
	\[ \left\lbrace \begin{split}
		P_y \dpart{\mu}{x} - P_x \dpart{\mu}{y} &= \mu \pares{\dpart{P_x}{y} - \dpart{P_y}{x}}, \\ 
		P_z \dpart{\mu}{x} - P_x \dpart{\mu}{z} &= \mu \pares{\dpart{P_x}{z} - \dpart{P_z}{x}}, \\ 
		P_z \dpart{\mu}{y} - P_y \dpart{\mu}{z} &= \mu \pares{\dpart{P_y}{z} - \dpart{P_z}{y}}, \\ 
	\end{split} \right. \]
	Сделаем переобозначение: \( \displaystyle A_{xy} := \dpart{P_x}{y} - \dpart{P_y}{x}, ~ A_{xz} := \dpart{P_x}{z} - \dpart{P_z}{x}, ~ A_{yz} := \dpart{P_y}{z} - \dpart{P_z}{y} \). Разделим каждое уравнение системы на $\mu$, сделаем переобозначение $\eta = \ln{\mu}$, тогда \( \displaystyle \frac{1}{\mu} \dpart{\mu}{x_k} = \dpart{\eta}{x_k} = \eta'_{x_k} \), и запишем систему в матричной форме:
	\[ \begin{pmatrix}
		P_y & - P_x & 0 \\
		P_z & 0 & - P_x \\
		0 & P_z & - P_y
	\end{pmatrix} \cdot \begin{pmatrix}
		\eta'_x \\ \eta'_y \\ \eta'_z
	\end{pmatrix} = \begin{pmatrix}
		A_{xy} \\ A_{xz} \\ A_{yz}
	\end{pmatrix} \]

	В полученной системе определитель матрицы равен нулю, но полученная система является неоднородной. Из нее можно вывести критерий существования интегрирующего множителя. Нетрудно убедиться, что если домножить второе уравнение на $\dfrac{P_y}{P_z}$, третье на $\dfrac{P_x}{P_z}$, и вычесть из второго третье, то в левой части нового уравнения получится в точности левая часть первого уравнения. Тогда при таких операциях, правые части тоже должны быть равны, а значит:
	\[ \frac{P_y}{P_z} A_{xz} - \frac{P_x}{P_z} A_{yz} = A_{xy}. \]
	Домножим на $P_z$, перенесем всё в одну часть, и раскроем замены, получим:
	\[ P_x \pares{\dpart{P_z}{y} - \dpart{P_y}{z}} + P_y \pares{\dpart{P_x}{z} - \dpart{P_z}{x}} + P_z \pares{\dpart{P_y}{x} - \dpart{P_x}{y}} = 0. \]
	Если такое равенство выполняется, то существует интегрирующий множитель для уравнения Пфаффа. Согласно теории решений неоднородных систем с вырожденными матрицами, у представленной системы при выполнении условия выше существует единственное однопараметрическое решение. Исключим одно уравнение из системы, например, первое:
	\[ \begin{pmatrix}
		P_z & 0 & - P_x \\
		0 & P_z & - P_y
	\end{pmatrix} \cdot \begin{pmatrix}
		\eta'_x \\ \eta'_y \\ \eta'_z
	\end{pmatrix} = \begin{pmatrix}
		A_{xz} \\ A_{yz}
	\end{pmatrix} \]
	Теперь положим, например, $\eta'_{z} = q$ -- параметр. Удобно выбирать в системе параметром ту переменную, столбец для которой не содержит нулей. Разрешим систему относительно $\eta'_{x}$ и $\eta'_{y}$:
	\[ \begin{pmatrix}
		P_z & 0 \\
		0 & P_z
	\end{pmatrix} \cdot \begin{pmatrix}
		\eta'_x \\ \eta'_y
	\end{pmatrix} = \begin{pmatrix}
		A_{xz} + P_x q \\ A_{yz} + P_y q
	\end{pmatrix} \]
	Решая эту систему, получим систему параметрических уравнений с частными производными следующего вида:
	\[ \left\lbrace \begin{split} 
		\eta'_x &= \frac{1}{P_z} \pares{A_{xz} + P_x q}, \\
		\eta'_y &= \frac{1}{P_z} \pares{A_{yz} + P_y q}, \\
		\eta'_z &= q.
	\end{split} \right. \]
	Выбирая значение $q = q(x, y, z)$ можно построить различные варианты уравнений для функции $\eta$. Значение $q$ необходимо подобрать такое, при котором полученная система будет совместной. Чтобы получить интегрирующий множитель в явном виде, достаточно после интегрирование полученной системы с выбранным значением $q$ воспользоваться обратной заменой $\eta = \ln{\mu} \implies \mu = e^{\eta}$. Домножая все уравнение на полученный интегрирующий множитель, исходное уравнение будет представлять из себя уравнение в полных дифференциалах.

	Так же возможны другие случаи параметризации этой системы, в зависимости от того, какое уравнение было исключено, и того, какая производная неизвестной функции была параметризована. К примеру, если исключить второе уравнение из системы, и параметризовать $\eta'_y$, то система примет вид:
	 \[ \left\lbrace \begin{split} 
		\eta'_x &= -\frac{1}{P_y} \pares{A_{xy} + P_x q}, \\
		\eta'_y &= q, \\
		\eta'_z &= \frac{1}{P_y} \pares{A_{yz} - P_z q}.
	\end{split} \right. \]

	Аналогично, если исключить третье уравнение из системы, и параметризовать $\eta'_x$, система принимает вид:
	\[ \left\lbrace \begin{split} 
		\eta'_x &= q, \\
		\eta'_y &= \frac{1}{P_x} \pares{A_{xy} - P_y q}, \\
		\eta'_z &= \frac{1}{P_x} \pares{A_{xz} - P_z q}.
	\end{split} \right. \]

	\subsection{Примеры}

		Рассмотрим следующий пример:
		\[ 2 \pares{x ~ dx - dy} \cdot \sec{y} + dx - z ~ dy = \pares{x ~ dy + dz} \tan{y} \] % * cos(y) -> x**2 + x*cos(y) - 2*y - z*sin(y) = C

		Первым шагом сведем это уравнение к классическому виду уравнения Пфаффа:
		\[ \pares{2x \sec{y} + 1} ~ dx - \pares{2 \sec{y} + z + x \tan{y}} ~ dy - \tan{y} ~ dz = 0. \]
		Здесь представлены следующие функции:
		\[ P_x = 2x \sec{y} + 1, \quad P_y = -2 \sec{y} - z - x \tan{y}, \quad P_z = -\tan{y}. \]

		Проверим критерий интегрируемости для данного уравнения. Для этого найдем соовтетствующие частные производные:
		\[ \dpart{P_x}{y} = 2x \sec{y} \tan{y}, ~ \dpart{P_y}{x} = - \tan{y} \implies \dpart{P_x}{y} \neq \dpart{P_y}{x}; \]
		\[ \dpart{P_x}{z} = 0, ~ \dpart{P_z}{x} = 0 \implies \dpart{P_x}{z} = \dpart{P_z}{x}; \]
		\[ \dpart{P_y}{z} = -1, ~ \dpart{P_z}{y} = -\sec^2{y} \implies \dpart{P_y}{z} \neq \dpart{P_z}{y}. \]

		Из полученных частных производных следует, что критерий интегрируемости не выполняется. В таком случае найдем интегрирующий множитель. Проверим критерий существования интегрирующего множителя:
		\[ \begin{split}
			&\pares{2x \sec{y} + 1} \pares{
				1 - \sec^2{y}
			} -
			\pares{2 \sec{y} + z + x \tan{y}} \pares{
				0 - 0
			} + \\ +&
			\tan{y} \pares{
				2x \sec{y} \tan{y} + \tan{y}
			} = 0. 
		\end{split} \]
		Раскрываем скобки, и, пользуясь одной из вариаций основного тригонометрического тождества: $\sec^2{y} - 1 = \tan^2{y}$, получим:
		\[ -2x \sec{y} \tan^2{y} - \tan^2{y} + 2x \sec{y} \tan^2{y} + \tan^2{y} = 0. \]
		Получили верное равенство, следовательно для уравнения существует интегрирующий множитель $\mu = \mu(x, y, z)$. Введем функцию $\eta = \ln{\mu}$, и воспользуемся уже готовым выводом для интегрирующего множителя. В данном случае достаточно простой функцией является $P_z$, воспользуемся выводом именно с ней в знаменателе:
		\[ \left\lbrace \begin{split} 
			\eta'_x &= -\frac{1}{\tan{z}} \bpares{0 + \pares{2x \sec{y} + 1} q}, \\
			\eta'_y &= -\frac{1}{\tan{z}} \bpares{\sec^2{y} - 1 - \pares{2 \sec{y} + z + x \tan{y}} q}, \\
			\eta'_z &= q.
		\end{split} \right. \]
		Удобно выбрать, например, $q = 0$. Тогда система принимает вид:
		\[ \left\lbrace \begin{split} 
			\eta'_x &= 0, \\
			\eta'_y &= -\tan{y}, \\
			\eta'_z &= 0.
		\end{split} \right. \]	
		Из этой системы видно, что интегрирующий множитель зависит только от переменной $y$, и при этом критерий совместности системы выполняется: 
		\[ \dpart{\eta'_x}{y} = \dpart{\eta'_y}{x} = 0, \quad \dpart{\eta'_x}{z} = \dpart{\eta'_z}{x} = 0, \quad \dpart{\eta'_y}{z} = \dpart{\eta'_z}{y} = 0. \]
		Интегрируя систему, получаем интегрирующий множитель в следующем виде:
		\[ \eta = \ln{\cos{y}} + C \implies \mu = C \cos{y}, ~ C \neq 0. \]

		Положим $C = 1$ для удобства. Можно выбрать любое удобное значение $C$, так как уравнение можно на него сократить при условии, что $C \neq 0$, и решение от этого не изменится. Домножим уравнение на полученный интегрирующий множитель, и приведем к классическому виду:
		\[ \pares{2x + \cos{y}} ~ dx - \pares{2 + z \cos{y} + x \sin{y}} ~ dy - \sin{y} ~ dz = 0 \] % * 1 -> x**2 + x*cos(y) - 2*y - z*sin(y) = C
		Здесь:
		\[ P_x = 2x + \cos{y}, \quad P_y = -2 - z \cos{y} - x \sin{y}, \quad P_z = -\sin{y}. \]
		Проверим критерий интегрируемости для полученного уравнения:
		\[ \dpart{P_x}{y} = -\sin{y}, ~ \dpart{P_y}{x} = -\sin{y} \implies \dpart{P_x}{y} = \dpart{P_y}{x}, \]
		\[ \dpart{P_x}{z} = 0, ~ \dpart{P_z}{x} = 0 \implies \dpart{P_x}{z} = \dpart{P_z}{x}. \]
		\[ \dpart{P_y}{z} = - \cos{y}, ~ \dpart{P_z}{y} = - \cos{y} \implies \dpart{P_y}{z} = \dpart{P_z}{y}. \]
		Критерий интегрируемости выполняется, соответственно данное уравнение представляет из себя уравнение в полных дифференциалах. Тогда положим, что существует функция $u(x, y, z)$ такая, что левая часть уравнения представляет из себя полный дифференциал этой функции, а значит справедливо уравнение:
		\[ du = 0 \implies u = C_1. \]
		Зная, что \( \displaystyle du = \dpart{u}{x} ~ dx + \dpart{u}{y} ~ dy + \dpart{u}{z} ~ dz \), получим систему следующих соответствий:
		\[ \left\lbrace \begin{split} 
			\dpart{u}{x} &= 2x + \cos{y}; \\
			\dpart{u}{y} &= - 2 - z \cos{y} - x \sin{y}; \\
			\dpart{u}{z} &= -\sin{y}.
		\end{split} \right. \]
		Представленная система является системой уравнений с частными производными, разрешенная относительно производных. При этом, каждое уравнение представляет из себя уравнение, допускающее разделение переменных. Проинтегрируем уравнения по соотвествующим им переменным, полагая оставшиеся параметрами, не участвующими в интегрировании:
		\[ \left\lbrace \begin{split} 
			u &= x^2 + x \cos{y} + \varphi_x(y, z); \\
			u &= -2y - z \sin{y} + x \cos{y} + \varphi_y(x, z); \\
			u &= - z \sin{y} + \varphi_z(x, y).
		\end{split} \right. \]
		Так как каждое из найденных решений соответствует одному и тому же выражению, то поставим соответствия между функциями $\varphi$:
		\[ \varphi_x(y, z) = - 2y - z \sin{y}, ~ \varphi_y(x, z) = x^2, ~ \varphi_z(x, y) = x^2 + x \cos{y} - 2y. \]
		При этом, $\bar{\varphi}_x(y, z) = \bar{\varphi}_y(x, z) = \bar{\varphi}_z(x, y) = C_2$. Тогда поверхность, удовлетворяющая полученной системе принимает вид:
		\[ u = x^2 + x\cos{y} - z \sin{y} - 2y + C_2, \]
		а решение уравнения Пфаффа $u = C_1$ при известной функции $u$ принимает вид:
		\[ x^2 + x\cos{y} - z \sin{y} - 2y = C, \quad \mu = \cos{y}. \]
		Представленное уравнение является общим решением уравнения Пфаффа.

	% \pagebreak