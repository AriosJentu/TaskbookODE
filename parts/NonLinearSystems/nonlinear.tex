\section{Нелинейные системы обыкновенных дифференциальных уравнений}

	Пусть \( \vec{x} = \begin{pmatrix} x_1 \\ \vdots \\ x_n \end{pmatrix} \) -- вектор неизвестных функций, $\vec{x} = \vec{x} \pares{t}$. Системой нелинейных обыкновенных дифференциальных уравнений первого порядка называется система следующего вида:
	\[ \dot{\vec{x}} = \vec{f}\pares{\vec{x}, t}. \]
	Стационарной системой обыкновенных дифференциальных уравнений первого порядка называется система следующего вида:
	\[ \dot{\vec{x}} = \vec{f}\pares{\vec{x}}. \]
	Симметрической системой $(n-1)$-го равенства обыкновенных дифференциальных уравнений первого порядка называется система следующего вида:
	\[ \frac{dx_1}{f_1\pares{\vec{x}, t}} = \frac{dx_2}{f_2\pares{\vec{x}, t}} = \dots = \frac{dx_n}{f_n\pares{\vec{x}, t}}. \]

	Первым интегралом нелинейной системы называется некоторая функция $\varphi\pares{\vec{x}, t} = C_1$, полученная путем интегрирования любой комбинации уравнений системы ровно один раз. Здесь $C_1$ -- произвольная постоянная. $k$-порядковым интегралом ($k$-м интегралом, общим итегралом) называется некоторая функция $\varphi\pares{\vec{x}, t, C_1, \dots, C_{k-1}} = C_k$, содержащая в себе $(k-1)$-порядковых интегралов. Соответственно, вторым интегралом называется интеграл, содержащий в себе две произвольные постоянные. Полным интегралом системы называется интеграл, полученный путем полного интегрирования системы (содержит в себе $n$-произвольных постоянных). Общим решением системы называется совокупность всех возможных первых и общих интегралов, образующих полный интеграл.

	Для нелинейных систем характерны следующие свойства:
	\begin{enumerate}
		\item Любую стационарную систему из $n$ уравнений можно свести к нестационарной системе $(n-1)$-го уравнения. Для этого достаточно зафиксировать $k$-е уравнение ($k$ -- произвольно, $f_k \pares{\vec{x}} \not\equiv 0$), и затем разделить каждое уравнения на данное. $k$-е уравнение примет вид $1 = 1$, тогда его можно исключить из системы. Таким образом $x_k$ становится новой независимой переменной. \label{nls:e1}

		\item Любую систему из $n$ уравнений можно свести к симметрической системе путем разделения каждого уравнения на их соответствующую правую часть, и дальнейшего приравнивая их между собой. Такую операцию можно производить и в обраную сторону. \label{nls:e2}

	\end{enumerate}

	Рассмотрим свойство \ref{nls:e1}.
	\[ \left\lbrace \begin{split} 
		\dot{x}_1 &= f_1\pares{\vec{x}}, \\
		&\vdots \\
		\dot{x}_k &= f_k\pares{\vec{x}}, \\
		&\vdots \\
		\dot{x}_n &= f_n\pares{\vec{x}}. \\
	\end{split} \right. \]
	Зафиксируем уравнение $\dot{x}_k = f_k\pares{\vec{x}}$, где $f_k\pares{\vec{x}} \not\equiv 0$. Введем вектор \( \vec{y} = \bracs{x_1, \dots, x_{k-1}, x_{k+1}, \dots, x_n} = \vec{x} \setminus \bracs{x_k} \) размерности $n-1$, полагая теперь, что $\vec{y} = \vec{y}\pares{x_k}$. Разделим каждое $m$-е уравнение системы на $k$-е, $m \in \overline{1, n}, ~ m \neq k$:
	\[ \frac{\dot{x}_m}{\dot{x}_k} = \frac{f_m\pares{\vec{x}}}{f_k\pares{\vec{x}}}. \]
	Здесь $\frac{\dot{x}_m}{\dot{x}_k} = \frac{\difft{x_m}{t}}{\difft{x_k}{t}} := \difft{x_m}{x_k} = y_m'$. Переобозначим $\frac{f_m\pares{\vec{x}}}{f_k\pares{\vec{x}}} := g_m\pares{\vec{y}, x_k}$. Тогда система принимает следующий вид:
	\[ \vec{y}' = \vec{g}\pares{\vec{y}, x_k}. \]
	В данной системе отсутствует $k$-е уравнение. Таким образом получена нестационарная система из $(n-1)$-го уравнения.

	Рассмотрим свойство \ref{nls:e2}.
	\[ \left\lbrace \begin{split} 
		\dot{x}_1 &= f_1\pares{\vec{x}, t}, \\
		&\vdots \\
		\dot{x}_n &= f_n\pares{\vec{x}, t}. \\
	\end{split} \right. \]
	Разделяя каждое уравнение на их соответствующие правые части, получим следующую систему:
	\[ \left\lbrace \begin{split} 
		\frac{\dot{x}_1}{f_1\pares{\vec{x}, t}} &= 1, \\
		&\vdots \\
		\frac{\dot{x}_n}{f_n\pares{\vec{x}, t}} &= 1. \\
	\end{split} \right. \]
	Правые части каждого уравнения полученной системы равны, соответственно, приравняем каждое уравнение между собой:
	\[ \frac{\dot{x}_1}{f_1\pares{\vec{x}, t}} = \frac{\vec{x}_2}{f_2\pares{\vec{x}, t}} = \dots = \frac{\dot{x}_n}{f_n\pares{\vec{x}, t}} = 1. \]
	Домножим выражение на $dt$, получим симметрическую систему следующего вида:
	\[ \frac{dx_1}{f_1\pares{\vec{x}, t}} = \frac{dx_2}{f_2\pares{\vec{x}, t}} = \dots = \frac{dx_n}{f_n\pares{\vec{x}, t}} = dt. \]

	Рассмотрим это же свойство в другую сторону. Любая система равенств может быть представлена в виде некоторой системы. Зафиксируем некоторую $k$-ю часть равенства, и приравняем остальные части к нему в виде следующей системы:
	\[ \left\lbrace \begin{split} 
		\frac{dx_1}{f_1\pares{\vec{x}, t}} &= \frac{dx_k}{f_k\pares{\vec{x}, t}}, \\
		&\vdots \\
		\frac{dx_n}{f_n\pares{\vec{x}, t}} &= \frac{dx_k}{f_k\pares{\vec{x}, t}}, \\
		dt &= \frac{dx_k}{f_k\pares{\vec{x}, t}}.
	\end{split} \right. \]
	Разделим каждое уравнение на $x_k$, и умножим на соотвествующий знаменатель левой части каждого уравнения системы:
	\[ \left\lbrace \begin{split} 
		\frac{dx_1}{dx_k} &= \frac{f_1\pares{\vec{x}, t}}{f_k\pares{\vec{x}, t}}, \\
		&\vdots \\
		\frac{dx_n}{dx_k} &= \frac{f_n\pares{\vec{x}, t}}{f_k\pares{\vec{x}, t}}, \\
		\frac{dt}{dx_k} &= \frac{1}{f_k\pares{\vec{x}, t}}.
	\end{split} \right. \]
	Снова производя переобозначения, получим:
	\[ \left\lbrace \begin{split} 
		y'_1 &= g_1\pares{\vec{x}, t}, \\
		&\vdots \\
		y'_n &= g_n\pares{\vec{x}, t}, \\
		\frac{dt}{dx_k} &= g_k\pares{\vec{x}, t},
	\end{split} \right. \]
	где $t = t\pares{x_k}$.

	Рассмотрим методы решения систем нелинейных уравнений. Будем рассматривать стационарные уравнения. Для таких уравнений используется \textit{метод понижения порядка}. Положим, что возможно зафиксировать одно уравнение в системе. Тогда, разделяя каждое уравнение системы на него, согласно свойству (\ref{nls:e1}), полученная система будет состоять из $(n-1)$-го уравнения, тем самым был понижен её порядок.

	В случае систем из двух уравнений, такой метод сводит к классическому скалярному уравнению первого порядка:
	\[ \left\lbrace \begin{split} 
		\dot{x} &= f_1(x, y), \\
		\dot{y} &= f_2(x, y),
	\end{split} \right. \implies \difft{y}{x} = y' = g(x, y). \]

	Другой метод решения называется \textit{методом повышения порядка} для решения систем нестационарных уравнений. Он позволяет сводить системы $1$-го порядка из $n$ уравнений к уравнению $n$-го порядка. Его возможно реализовать только в случае, если имеется возможность выражать переменные из уравнений.

	В случае систем из двух уравнений, полагая, что одну из переменных в системе можно выразить в явном виде, система сводится к уравнению второго порядка:
	\[ \left\lbrace \begin{split}
		y &= \varphi\pares{x, \dot{x}, t}, \\
		\dot{y} &= \psi\pares{x, y, t},
	\end{split} \right. \implies \dpart{\varphi}{\dot{x}} \cdot \ddot{x} + \dpart{\varphi}{x} \cdot \dot{x} + \dpart{\varphi}{t} = \psi\pares{x, \varphi\pares{x, \dot{x}}\phn, t}, \]
	или, в простой форме:
	\[ F\pares{t, x, \dot{x}, \ddot{x}} = 0. \]

	В общем случае такие методы называются \textit{методами исключения переменной}. Если возможно выразить из одного из уравнений нестационарной системы независимую переменную $t$, то после подстановки такого выражения, система сократит количество уравнений до $n-1$. В случае, если возможно выразить зависимую переменную, то при подстановке в другие уравнения, их порядок производной увеличивается. 

	Рассмотрим методы решения симметрических систем. Для системы из $n$ равенств необходимо найти $n$ независимых первых и общих интегралов системы. Для равенств симметрической системы характерен принцип равенства дробей. Положим равенство некоторому значению $\lambda$ следующих дробей:
	\[ \frac{\alpha_1}{\beta_1} = \frac{\alpha_2}{\beta_2} = \dots = \frac{\alpha_n}{\beta_n} = \lambda. \]
	Тогда дробь, построенная в виде линейной комбинацией числителей и такой же линейной комбинацией знаменателей, также равна некоторому числу $\lambda$:
	\[ \frac{k_1 \alpha_1 + k_2 \alpha_2 + \dots + k_n \alpha_n}{k_1 \beta_1 + k_2 \beta_2 + \dots + k_n \beta_n} = \lambda ~ \forall k_i, ~ i \in \overline{1, n}. \]
	Составляя функциональные комбинации таким образом, чтобы получилось интегрируемое уравнение, можно построить, соответственно, первые интегралы системы. Уже известные комбинации можно использовать в дальнейших решениях уравнений. В симметрических системах возможно наличие выражений следующего вида:
	\[ \frac{dx_k}{0}. \]
	При построении системы уравнений, возникнет комбинация вида $dx_k = 0$, или $x_k = C_k$.

	\subsection{Примеры}

		\begin{enumerate}

			\item Рассмотрим следующий пример:
				\[ t\dot{x} + x = 0, ~ t^2 \dot{y} = 2x^2 - ty. \]
				Здесь представлена система двух нестационарных уравнений. Выразить свободно переменные $t, x$ или $y$ для получения более простой системы невозможно. Но в данном примере первое уравнение системы представляет собой вполне интегрируемое уравнение, соответственно, возможно понизить порядок данной системы. Воспользуемся методом подстановки решения. Первое уравнение системы имеет следующее решение:
				\[ xt = C_1. \]
				Подставим это значение во второе уравнение:
				\[ t^2 \dot{y} = 2 \frac{C_1^2}{t^2} - ty. \]
				Получили в результате линейное неоднородное уравнение первого порядка с переменными коэффициентами. Построим второй (полный) интеграл с помощью интегрирующего множителя $\frac{1}{t}$:
				\[ t\dot{y} + y = 2 \frac{C_1^2}{t^3} \implies ty + \frac{C_1^2}{t^2} = C_2. \]
				Запишем решение в виде системы первых интегралов. Подставим во второй интеграл решение первого уравнения. Получим следующую систему:
				\[ xt = C_1, ~ ty + x^2 = C_2. \]
				Полученная система является полным интегралом системы, и также является общим решением.

			\item Рассмотрим другой пример:
				\[ t\dot{x} = \frac{x}{1 - y}, ~ t\dot{y} = \frac{y}{1 - y} \]
				Данная система представляет собой также систему двух нестационарных уравнений. Так как второе уравнение является уравнением с разделяющимися переменными, можно воспользоваться методом подстановки, но в данном примере рассмотрим другой метод. Можно увидеть, что если разделить одно уравнение на другое, результирующее уравнение будет независимо от $t$. При этом, уравнение сведется к обыкновенному дифференциальному уравнению первого порядка. Воспользуемся частным случаем метода исключения переменной -- методом понижения порядка. Разделим второе уравнение на первое:
				\[ \frac{\dot{y}}{\dot{x}} := y' = \frac{y}{x}. \]
				Полученное уравнение является уравнением с разделяющимися переменными, первый интеграл (в данном случае он является также полным), соответственно, имеет следующий вид:
				\[ \frac{x}{y} = C_1. \]
				Построим общее решение данного уравнения. Для этого подставим первый интеграл в виде уравнения $x = C_1 y$ в рассматриваемую систему. Тогда
				\[ \dot{x} = C_1 \dot{y}, \]
				и
				\[ t C_1 \dot{y} = \frac{C_1 y}{1 - y}, ~ t \dot{y} = \frac{y}{1 - y}. \]
				При подстановке первого интеграла в исходную систему, с условием, что $C_1 \neq 0$, получили два зависимых уравнения. Исключая одно из уравнений в силу зависимости, получим одно уравнение с разделяющимися переменными. Построим его решение:
				\[ \frac{\pares{1 - y} ~ dy}{y} = \frac{dt}{t} \implies t e^{y} = C_2 y. \]
				Таким образом получили второе решение, которое является ещё одним первым интегралом системы. Тогда общее решение системы можно записать в виде следующей системы:
				\[ x = C_1 y, ~ t e^{y} = C_2 y. \]
				В некоторых задачах может потребоваться найти решение только в виде полного интеграла $F(x, y) = C$. Для таких случаев подстановка в уравнение и построение второго решения не требуется.

			\item Рассмотрим третий пример:
				\[ \dot{x} = y e^{x}, ~ \dot{y} = 2e^{-2x} - 2y^2 e^{x}. \]
				Представленная система является стационарной, и ее можно решать методом понижения порядка, как в случае примера выше. Для данного примера рассмотрим ещё один вариант метода исключения переменных -- метод повышения порядка. В первом уравнении системы можно выразить $y$ в через переменные $x$ и $\dot{x}$:
				\[ y = \dot{x} \cdot e^{-x}. \]
				Подставим данное равенство во второе уравнение, исключая из уравнения полностью переменную $y$. Тогда
				\[ \dot{y} = \ddot{x} \cdot e^{-x} - \dot{x}^2 \cdot e^{-x}, \]
				и при подстановке, уравнение принимает следующий вид:
				\[ \ddot{x} \cdot e^{-x} - \dot{x}^2 \cdot e^{-x} = 2e^{-2x} -2 \dot{x}^2 \cdot e^{-x}. \]
				Приведем подобные слагаемые, и домножим на $e^{2x}$:
				\[ \ddot{x} \cdot e^{x} + \dot{x}^2 \cdot e^{x} = 2. \]
				Левая часть уравнения представляет собой вторую полную производную функции $e^x$ по переменной $t$, тогда общее решение данного уравнения принимает вид:
				\[ e^{x} = t^2 + C_1 t + C_2. \]
				Дифференцируя данное выражение, выразим $\dot{x}$:
				\[ e^{x} \dot{x} = 2t + C_1 \implies \dot{x} = \pares{C_1 + 2t} e^{-x}. \]
				Подставим данное выражение в $y$, и запишем общий интеграл в виде следующей системы:
				\[ e^{x} = t^2 + C_1 t + C_2, ~ y = \pares{C_1 + 2t} e^{-2x}. \]
				Также можно выразить $x(t)$ и $y(t)$:
				\[ x = \ln\abs{t^2 + C_1 t + C_2}, ~ y = \frac{C_1 + 2t}{\pares{t^2 + C_1 t + C_2}^2} \]
				Данная система представляет собой общее решение исходной системы.

			\item Последний пример представляет собой следующую симметрическую систему трех равенств:
				\[ \frac{dx}{x^2 z} = \frac{dy}{z \pares{z - xy}} = - \frac{dz}{x z^2} = \frac{du}{uxz + x^2}. \] % u z - x = C_0; x z = C_1; x y + z = C_2
				Необходимо построить систему из трех независимых интегрируемых комбинаций. Первую комбинацию выберем в виде равенства между первым и третьим уравнением:
				\[ \frac{dx}{x^2 z} = - \frac{dz}{x z^2} \implies \frac{dx}{x} + \frac{dz}{z} = 0. \]
				Данная комбинация является вполне интегрируемой. Первый интеграл симметрической системы принимает вид:
				\[ xz = C_1. \]
				Для построения второй и третьей комбинации, необходимо задействовать, соответственно, второе и четвертое соотношения. Также возможно подставлять уже известные первые интегралы в систему. Построим вторую комбинацию по следующему принципу: домножим третье соотношение на $z$ (минус ассоциируем со знаменателем, соответственно числитель берется со знаком плюс), четвертое домножим на $u$, сложим по принципу равенства дробей для упрощения знаменателя, и приравняем результат к первому соотношению:
				\[ \frac{z ~ du + u ~ dz}{uxz^2 + x^2 z - uxz^2} = \frac{dx}{x^2 z} \implies z ~ du + u ~ dz = dx. \]
				В результате получена интегрируемая комбинация, и второй интеграл принимает следующий вид:
				\[ uz - x = C_2. \]
				На данный момент решения не было задействовано второе соотношение, которое также необходимо реализовать. Для построения третьей интегрируемой комбинации, воспользуемся известным первым интегралом:
				\[ \frac{dx}{C_1 x} = \frac{dy}{z^2 - C_1 y} = - \frac{dz}{C_1 z} = \frac{du}{C_1 u + x^2}. \] % u z - x = C_0; x z = C_1; x y + z = C_2
				Рассмотрим вторую и третью комбинацию:
				\[ \frac{dy}{z^2 - C_1 y} = - \frac{dz}{C_1 z} \implies \frac{dy}{dz} = \frac{y}{z} - \frac{z}{C_1}. \]
				Для полученного линейного неоднородного уравнения первого порядка с переменными коэффициентами для функции $y(z)$ построим решение с помощью интегрирующего множителя $\frac{1}{z}$:
				\[ \frac{y}{z} = \tilde{C}_3 - \frac{z}{C_1}. \]
				Данное решение является вторым интегралом. Запишем в виде первого интеграла, подставляя известное значение $C_1$:
				\[ xy = \tilde{C}_3 xz - z \implies xy = \tilde{C}_3 C_1 - z \implies xy + z = C_3 \]
				Теперь запишем общее решение данной системы:
				\[ xz = C_1, ~ uz - x = C_2, ~ xy + z = C_3. \]
				
		\end{enumerate}

	% \pagebreak
