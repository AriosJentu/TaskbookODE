\section{Уравнения с разделяющимися переменными}

    Дифференциальным уравнением называется равенство нулю функции, зависимой от неизвестной функции, ее аргументов и всех возможных ее производных:
    \[
        F \left( x_1, \dots, x_n, u, \dpart{u}{x_1}, \dots, \dpart{u}{x_n}, \dots \right) = 0.
    \]
    Порядком дифференциального уравнения называется порядок самой старшей производной в уравнении. Например в уравнении
    \[
        xu^3 - \left( \dfrac{\partial u}{\partial y} \right)^4 = \dfrac{\partial^3 u}{\partial x^3} + y^2 \dfrac{\partial^2 u}{\partial x \partial y}.
    \]
    порядок старшей производной -- третий.
    
    Обыкновенным дифференциальным уравнением называется уравнение, в котором искомая функция зависит только от одной переменной:
    \[
        F \left( x, y, \dfrac{dy}{dx}, \dots, \dfrac{d^ny}{dx^n} \right) = 0.
    \]

    Для определения единственного решения уравнения, для него можно поставить начальные условия. Начальными условиями называются все значение функции $ y $ до $ (n - 1) $-й производной в точке $ x_0 $. Совокупность уравнения и начальных условий называется задачей Коши:
    \[
        \begin{cases}
            F \left( x, y, \dfrac{dy}{dx}, \dots, \dfrac{d^ny}{dx^n} \right) = 0; \\
            y(x_0) = y_0, ~ \dfrac{dy}{dx} \Bigg|_{x_0} = y_0', ~ \dots, ~ \dfrac{d^{n - 1}y}{dx^{n - 1}} \Bigg|_{x_0} = y_0^{n - 1}.
        \end{cases}
    \]
    Решением задачи Коши является функция, одновременно удовлетворяющая уравнению и начальным условиям.
    
    Введение в уравнения с разделяющимися переменными.
    \[
        M_1(x) N_1(y) dx = M_2(x) N_2(y) dy.
    \]
    Метод решения -- разделение переменных и последующее интегрирование.

    \subsection{Примеры}
        \[
            y'\cos^2{x} = \tan{y} + \cot{y}.
        \]
