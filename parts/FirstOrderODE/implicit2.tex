\section{Метод общей параметризации, уравнения Лагранжа и Клеро}

    Теперь будем рассматривать уравнения вида
    \[
        F(x, y, y') = 0.
    \]
    Метод общей параметризации заключается в положении $ y’ = p $, использовании соответствия $ dy = p dx $ и получении параметрического решения. Алгоритм решения разделяется на следующие этапы:
    \begin{enumerate}
        \item полагаем $ y’ = p $;
        \item если возможно, выразим из уравнения $ x $ или $ y $;
        \item находим полный дифференциал уравнения;
        \item применяем соотношения между дифференциалами $ dx $ и $ dy $;
        \item решаем полученное уравнение;
        \item записываем решение в параметрическом виде.
    \end{enumerate}

    Уравнением Лагранжа называется уравнение следующего вида:
    \[
        y = xf(y') + g(y').
    \]
    В общем случае решается методом общей параметризации. Частный случай уравнения Лагранжа -- уравнение Клеро:
    \[
        y = xy' + g(y').
    \]
    Метод общей параметризации сведет уравнение к следующему виду:
    \[
        \begin{split}
            y' = p \Longrightarrow y = xp + g(p); \\
            dy = p dx = p dx + x dp + g(p) dp \Longrightarrow \bpares{x + g(p)} dp = 0.
        \end{split}
    \]
    Из последнего выражения возможны два случая:
    \[
        x = -g(p) \text{ или } p = C.
    \]
    Общее решение уравнения Клеро можно получить из второго случая:
    \[
        y = Cx + g(C).
    \]

    \subsection{Примеры}
        \[
            y = 2xy' + y^2{y'}^3, ~ y' = p \Longrightarrow y = 2xp + y^2 p^3 \Longrightarrow x = \dfrac{y}{2p} - \dfrac{y^2 p^2}{2}.
        \]
        Разрешили представленное уравнение относительно переменной $ x $.
        \[
            dx = \dfrac{dy}{p} = \dfrac{dy}{2p} - \dfrac{y dp}{2p^2} - y p^2 dy - y^2 p dp \Longrightarrow \bpares{2y p^3 + 1} \pares{p dy + y dp} = 0.
        \]
        Отсюда
        \[
            y = -\drecp{2p^3} \text{ или } y = \dfrac{C}{p}.
        \]
        \[
            \begin{cases}
                x = \dfrac{y}{2p} - \dfrac{y^2 p^2}{2}, \\
                y = \dfrac{C}{p}.
            \end{cases}
        \]
