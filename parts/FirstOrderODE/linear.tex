\section{Линейные уравнения, уравнения Бернулли и Риккати}

    Общий вид:
    \[
        a_1(x)y' + a_2(x)y = f(x).
    \]
    В линейном уравнении искомая функция со своими производными входит только с первыми степенями -- линейно. Для линейных уравнений любая линейная комбинация решений также является решением.
    
    Классификация:
    \begin{itemize}
        \item порядок уравнения -- первый;
        \item $ a_1(x) = 1 $ -- приведенное (иначе -- неприведенное);
        \item $ f(x) = 0 $ -- однородное (иначе -- неоднородное);
        \item $ a_1(x), a_2(x) = \const $ одновременно -- с постоянными коэффициентами (иначе -- с переменными).
    \end{itemize}
    Пример полной классификации: линейное неприведенное неоднородное уравнение первого порядка с переменными коэффициентами.

    Методы решения -- метод Бернулли, метод Лагранжа.
    Вывод системы для формулы Бернулли(метод разделения переменных), вывод для метода Лагранжа(метод вариации произвольной постоянной).
    
    Уравнения, приводящиеся к линейным:
    \[
        a_1(x)p'(x, y) + a_2(x)p(x, y) = f(x), ~ u = p(x, y) \Longrightarrow a_1(x)u' + a_2(x)u = f(x).
    \]

    Частный случай -- уравнение Бернулли:
    \[
        a_1(x)y' + a_2(x)y = f(x)y^n, ~ n \neq 0, 1.
    \]

    Уравнение Риккати:
    \[
        y' = p(x) + q(x)y + r(x) y^2.
    \]

    Решается или подбором частного решения и сведения к уравнению Бернулли, или методом повышения порядка путем избавления от коэффициента $ r(x) $:
    \[
        u = r(x)y \Longrightarrow y' = \dfrac{u' - r'(x)u}{r(x)} \Longrightarrow u' = p(x)r(x) + \bpares{q(x) + r'(x)}u + u^2,
    \]
    а затем, после переобозначений
    \[
        p(x)r(x) = P(x), ~ q(x) + r'(x) = Q(x) \Longrightarrow u' = u^2 + Q(x)u + P(x)
    \]
    и замены
    \[
        u = -\dfrac{v'}{v}, ~ u' = \dfrac{v'^2}{v^2} - \dfrac{v''}{v}
    \]
    при подстановке получим приведенное линейное однородное уравнение второго порядка с переменными коэффициентами:
    \[
        \begin{split}
            &\dfrac{v'^2}{v^2} - \dfrac{v''}{v} = \dfrac{v'^2}{v^2} + Q(x)\dfrac{v'}{v} + P(x), \\
            &v'' + Q(x)v' + P(x)v = 0.
        \end{split}
    \]

    \subsection{Примеры}
        \[
            2y' + \drecp{x} \cdot e^{-2y} \tan{x} = -\drecp{x} ~ \Bigg| \cdot e^y, ~ z = e^y \Longrightarrow 2z' + \dfrac{z}{x} = -\dfrac{\tan{x}}{zx},
        \]
        \[
            u = z^2, ~ u' = 2zz' \Longrightarrow u' + \dfrac{u}{x} = -\dfrac{\tan{x}}{x}.
        \]
