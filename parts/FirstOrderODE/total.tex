\section{Уравнения в полных дифференциалах}

    Уравнения вида
    \[
        P(x, y)dx + Q(x, y)dy = 0
    \]
    называются уравнениями в полных дифференциалах, если левая часть уравнения представляет из себя полный дифференциал некоторой функции
    \[
        u = u(x, y) ~ : ~ du = \dfrac{\partial u}{\partial x} dx + \dfrac{\partial u}{\partial y} dy.
    \]
    Критерием уравнения в полных дифференциалах (или критерием Коши, критерием интегрируемости) является критерий равенства смешанных производных.
    \[
        \dfrac{\partial^2 u}{\partial x \partial y} = \dfrac{\partial^2 u}{\partial y \partial x} \Longleftrightarrow \dfrac{\partial P}{\partial y} = \dfrac{\partial Q}{\partial x}.
    \]
    
    Метод решения -- метод соответствий. Если уравнению удовлетворяет одна функция, то решение системы уравнений с частными производными должно быть совместно с точностью до константы:
    \[
        \begin{cases}
            \dpart{u}{x} = P(x, y), \\
            \dpart{u}{y} = Q(x, y).
        \end{cases}
        \Longrightarrow
        \begin{cases}
            u = \displaystyle\int P(x, y) ~ \partial x + C_1(y), \\
            u = \displaystyle\int Q(x, y) ~ \partial y + C_2(x).
        \end{cases}
    \]
    Затем ставим соответствия между функциями $ C_1 $ и $ C_2 $ в этой системе. Так как левая часть уравнения является полным дифференциалом функции $ u $, то интегрируя всё уравнение, получим, что $ u = C $. Зная общий вид функции $ u $, можно исключить $ u $ из системы, получая общее решение уравнения.

    Рассмотрим случай, когда критерий Коши не выполняется. Но предварительно рассмотрим понятие симметрической системы. Симметрической системой называется система такого вида:
    \[
        f_1 = f_2 = \dots = f_n.
    \]
    Для таких систем выполняется свойство, называемое принципом равенства дробей. Рассмотрим на примере:
    \[
        \dfrac{1}{2} = \dfrac{2}{4} = \dfrac{3}{6} = \dfrac{1 + 2 + 3}{2 + 4 + 6} = \dfrac{1 \cdot 2 - 2 \cdot 4 + 3 \cdot 5}{2 \cdot 2 - 4 \cdot 4 + 6 \cdot 5}.
    \]
    Обобщая принцип равенства, получим следующий результат:
    \[
        \dfrac{a_1}{b_1} = \dfrac{a_2}{b_2} = \dots = \dfrac{a_n}{b_n} = \dfrac{k_1 a_1 + k_2 a_2 + \dots + k_n a_n}{k_1 b_1 + k_2 b_2 + \dots + k_n b_n}.
    \]

    Перейдем к интегрирующему множителю. Уравнения вида
    \[
        P(x, y)dx + Q(x, y)dy = 0,
    \]
    для которых критерий Коши не выполняется, т.е.
    \[
        \dpart{P}{y} \neq \dpart{Q}{x}
    \]
    могут быть приведены к уравнениям в полных дифференциалах с помощью интегрирующего множителя. Положим существование такой функции
    \[
        \mu = \mu(x, y),
    \]
    при умножении всего уравнения на которую, оно становится уравнением в полных дифференциалах. Тогда критерий интегрируемости для такого уравнения принимает вид:
    \[
        \dpart{(P\mu)}{y} = \dpart{(Q\mu)}{x}.
    \]
    Раскрывая производные, получим уравнение с частными производными первого порядка относительно неизвестной функции $ \mu(x, y) $:
    \[
        Q\dpart{\mu}{x} - P\dpart{\mu}{y} = \mu \pares{\dpart{P}{y} - \dpart{Q}{x}}.
    \]
    Получить сам интегрирующий множитель можно из следующей системы уравнений
    \[
        \dfrac{dx}{Q} = -\dfrac{dy}{P} = \dfrac{d\mu}{\mu \pares{\dpart{P}{y} - \dpart{Q}{x}}}.
    \]
    В этой системе достаточно найти одну комбинацию, при которой возможно найти $ \mu(x, y) $.

    Топ 10 интегрирующих множителей:

    \subsection{Примеры}
        \[
            2x \sin{y} dx + x^2 \cos{y} dy = e^x dx.
        \]

        \[
            \pares{y + \drecp{y}} dx + 2x dy = xy \tan{x} dx.
        \]
        Сведем уравнение к общему виду
        \[
            \pares{y + \drecp{y} - xy \tan{x}} dx + 2x dy = 0.
        \]
        Нетрудно убедиться, что критерий интегрируемости не выполняется
        \[
            \begin{split}
                &\dpart{P}{y} = 1 - \drecp{y^2} - x \tan{x}, \\
                &\dpart{Q}{x} = 2.
            \end{split}
        \]
        Тогда отыщем интегрирующий множитель. Для этого найдем критерий интегрируемости в разности
        \[
            \dpart{P}{y} - \dpart{Q}{x} = -\pares{1 + \drecp{y^2} + x \tan{x}}.
        \]
        Составим соответствующую симметрическую систему для интегрирующего множителя
        \[
            \dfrac{dx}{2x} = \dfrac{dy}{xy \tan{x} - y \drecp{y}} = \dfrac{d\mu}{-\mu \pares{1 + \drecp{y^2} + x \tan{x}}}.
        \]
        Попробуем из первых двух уравнений системы составить линейную комбинацию таким образом, чтобы получить множитель знаменателя третьего уравнения системы, а в третьем уравнении положим множитель $ y $:
        \[
            \dfrac{y \tan{x} dx - dy}{2xy \tan{x} - \pares{xy\tan{x} - y - \drecp{y}}} = \dfrac{y d\mu}{-\mu \pares{y + \drecp{y} + xy \tan{x}}}.
        \]
        Упростим:
        \[
            \dfrac{y \tan{x} dx - dy}{xy\tan{x} + y + \drecp{y}} = \dfrac{y d\mu}{-\mu \pares{y + \drecp{y} + xy \tan{x}}}.
        \]
        Упростим результат, разделим переменные и получим уравнение, допускающее интегрирование:
        \[
            y \tan{x} dx - dy = -\dfrac{y}{\mu} d\mu \Bigg| ~ /y \Longrightarrow \tan{x} dx - \dfrac{dy}{y} = -\dfrac{\mu}{\mu}.
        \]
        Проинтегрируем уравнение с разделенными переменными
        \[
            \ln{\mu} = \ln{\cos{x}} + \ln{y} + C \Longrightarrow \mu = Cy \cos{x}.
        \]
        Положим константу любым удобным ненулевым значением, и умножим уравнение на этот множитель:
        \[
            \bpares{y^2\cos{x} - xy^2\sin{x} + \cos{x}} dx + \pares{2xy \cos{x}} dy = 0.
        \]
        Проверим критерий интегрируемости:
        \[
            \begin{split}
                &\dpart{P}{y} = 2y \cos{x} - 2xy \sin{x}, \\
                &\dpart{Q}{x} = 2y \cos{x} - 2xy \sin{x}.
            \end{split}
        \]
        Критерий выполняется, таким образом получено уравнение в полных дифференциалах.
