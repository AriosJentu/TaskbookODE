\section{Уравнения, не разрешённые относительно производной}

    Уравнения вида
    \[
        F(x, y, y') = 0,
    \]
    в которых невозможно выразить производную искомой функции так, чтобы получить уравнение вида
    \[
        y' = f(x, y)
    \]
    называются уравнениями, неразрешенными относительно производной.

    Рассмотрим частные случаи.
    \begin{enumerate}
        \item Случай, когда в уравнении присутствует только производная:
            \[
                F(y') = 0.
            \]
            Такие уравнения решаются поиском нулей функции $ F $. Положим $ F(p_k) = 0, ~ p_k – \const $, тогда исходное уравнение сводится к уравнению, разрешенному относительно производной следующего вида
            \[
                y' = p_k \Longrightarrow y = p_k x + C.
            \]
            Разрешая относительно $ p_k $ и подставляя в исходное уравнение, получим общее решение:
            \[
                p_k = \dfrac{y + C}{x} \Longrightarrow F\pares{\dfrac{y + C}{x}} = 0.
            \]

        \item Случай, когда в уравнении отсутствует искомая функция в явном виде:
            \[
                F(x, y') = 0.
            \]
            Если такое уравнение удается разрешить относительно производной, то получаем уравнение с разделяющимися переменными. Если можно разрешить относительно $ x $, то получим
            \[
                x = f(y').
            \]
            Положим $ y’ = p(x) $, выразим $ dy = p dx $, и продифференцируем исходное уравнение:
            \[
                dx = f'(p) dp.
            \]
            Домножая на $ p $ и возвращаясь к замене, получим уравнение относительно переменной $ y $
            \[
                dy = p dx = p f'(p) dp.
            \]
            Интегрируя полученное уравнение, решение запишем в параметрическом виде
            \[
                \begin{cases}
                    x = f(p), \\
                    y = \displaystyle\int p f'(p) ~ dp + C.
                \end{cases}
            \]

        \item Случай, когда в уравнении отсутствует аргумент в явном виде:
            \[
                F(y, y') = 0.
            \]
            Аналогично с предыдущим случаем можно получить уравнение с разделяющимися переменными. Если возможно выразить $ y $ в явном виде, то получим
            \[
                y = f(y'), ~ y' = p.
            \]
            Аналогично вводя параметр, и проводя аналогичные операции, запишем решение в параметрическом виде
            \[
                \begin{split}
                    &dy = p dx = f'(p) dp, \\
                    &dx = \dfrac{f'(p)}{p} dp; \\
                    &\begin{cases}
                        x = \displaystyle\int \dfrac{f'(p)}{p} ~ dp + C, \\
                        y = f(p).
                    \end{cases}
                \end{split}
            \]
    \end{enumerate}

    \subsection{Примеры}
        \[
            {y'}^3 - 1 = 0 \Longrightarrow \pares{\dfrac{y + C}{x}}^3 - 1 = 0.
        \]

        \[
            \begin{split}
                &x = y' + e^{y'}, ~ y' = p \Longrightarrow x = p + e^p, \\
                &dx = (1 + e^p) dp, ~ dy = p dx = \bpares{p + pe^p} dp. \\
                &\begin{cases}
                    x = p + e^p, \\
                    y = \dfrac{p^2}{2} + (p - 1)e^p + C.
                \end{cases}
            \end{split}
        \]


        \[
            \begin{split}
                &y = y' + \ln{y'}, ~ y' = p \Longrightarrow y = p + \ln{p}, \\
                &dy = \pares{1 + \drecp{p}} dp, ~ dx = \dfrac{dy}{p} = \pares{\drecp{p} + \drecp{p^2}} dp. \\
                &\begin{cases}
                    x = \ln{p} - \drecp{p} + C, \\
                    y = p + \ln{p}.
                \end{cases}
            \end{split}
        \]