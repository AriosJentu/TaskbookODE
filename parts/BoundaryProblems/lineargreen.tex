\section{Функции Грина}

	В данном разделе рассмотрим интегральные формулы Грина и функции Грина для решения краевых задач для линейных уравнений второго порядка.

	Введем линейный дифференциальный оператор второго порядка по следующему правилу:
	\[ L[u] = \difft{}{x} \bracks{ p(x) \cdot \difft{u}{x} } - q(x) \cdot u, ~ u = u(x) \]

	Рассмотрим краевую задачу для следующего линейного дифференциального уравнения второго порядка:
	\[ L[u] = f(x), ~ x \in \pares{x_0, x_l} \]
	с однородными краевыми условиями Дирихле:
	\[ u(x_0) = u(x_l) = 0. \]
	Введем такую функцию $v \in C^2[x_0, x_l]$, удовлетворяющую однородным условиям Дирихле:
	\[ v(x_0) = v(x_l) = 0, \]
	затем домножим исходное уравнение на данную функцию, и проинтегрируем по области $D = [x_0, x_l]$:
	\[ \int\limits_{x_0}^{x_l} L[u] \cdot v ~ dx = \int\limits_{x_0}^{x_l} f \cdot v ~ dx. \]
	Раскроем линейный оператор:
	\[ \int\limits_{x_0}^{x_l} \difft{}{x} \bracks{p \cdot \difft{u}{x}} \cdot v ~ dx - \int\limits_{x_0}^{x_l} q \cdot u \cdot v ~ dx = \int\limits_{x_0}^{x_l} f \cdot v ~ dx. \]
	Для первого слагаемого воспользуемся формулой интегрирования по частям:
	\[ \int\limits_{x_0}^{x_l} \difft{}{x} \bracks{p \cdot \difft{u}{x}} \cdot v ~ dx = p \cdot v \cdot \difft{u}{x} \at_{x_0}^{x_l} - \int\limits_{x_0}^{x_l} p \cdot \difft{u}{x} \cdot \difft{v}{x} ~ dx. \]
	Из условия, что функция $v$ удовлетворяет однородным условиям Дирихле, следует:
	\[ \int\limits_{x_0}^{x_l} \difft{}{x} \bracks{p \cdot \difft{u}{x}} \cdot v ~ dx = - \int\limits_{x_0}^{x_l} p \cdot \difft{u}{x} \cdot \difft{v}{x} ~ dx. \]
	Введем скалярное произведение следующим образом:
	\[ \pares{u, v} = \int\limits_{x_0}^{x_l} u(x) \cdot v(x) ~ dx, \]
	тогда исходная задача принимает следующий вид:
	\[ -\pares{p \cdot u', v'} - \pares{q \cdot u, v} = \pares{f, v}, \]
	что является частным случаем первой формулы Грина, а исходная краевая задача сведена к интегральной.

	\vspace{10pt}

	Теперь рассмотрим способы сведения линейных краевых задач с неоднородными краевыми условиями к неоднородным линейным краевым задачам с однородными краевыми условиями. Снова рассмотрим уравнение
	\[ L[y] = f(x), ~ x \in \pares{x_0, x_l} \]
	с неоднородными смешанными краевыми условиями:
	\[ a_{11} y(x_0) + a_{12} y'(x_0) = y_0, ~ a_{21} y(x_l) + a_{22} y'(x_l) = y_l. \]
	Так как представленное уравнение является линейным, то выполняется принцип суперпозиции:
	\[ y = u + v. \]
	Пусть функция $v$ удовлетворяет только неоднородным краевым условиям, тогда функция $u$ должна удовлетворять, соответственно, однородным краевым условиям, и новому уравнению. Построим новую краевую задачу из этих условий:
	\[ L[u] + L[v] = f(x), ~ x \in \pares{x_0, x_l}, \]
	\[ a_{11} u(x_0) + a_{12} u'(x_0) = 0, ~ a_{21} u(x_l) + a_{22} u'(x_l) = 0, \]
	\[ a_{11} v(x_0) + a_{12} v'(x_0) = y_0, ~ a_{21} v(x_l) + a_{22} v'(x_l) = y_l. \]
	Положим, что функцию $v$ можно подобрать согласно краевым условиям, тогда краевая задача сводится к задаче с однородными краевыми условиями относительно новой неизвестной функции $u$.

	Подбирать функцию $v$ можно различными способами. Например, пусть $v$ -- некоторая квадратичная функция с неопределенными коэффициентами: $v = Ax^2 + Bx + C$. Подставляя в краевые условия, можно подобрать значения коэффициентов $A, B, C$:
	\[ \syst{
		&a_{11} \pares{A x_0^2 + B x_0 + C} + a_{12} \pares{2A x_0 + B} = y_0, \\
		&a_{21} \pares{A x_l^2 + B x_l + C} + a_{22} \pares{2A x_l + B} = y_l, \\
	} \] 
	Представленная система является линейной системой алгебраических уравнений из двух уравнений и трех неизвестных. Из этой системы можно получить любой набор коэффициентов $A, B, C$, на основе которых построить такую функцию $v$, которая будет удовлетворять краевым условиям. Так как функция $v$ известна, то и известно значение выражения $L[v]$. Пусть $g(x) = f(x) - L[v]$. Тогда краевая задача принимает следующий вид:
	\[ L[u] = g(x), ~ x \in \pares{x_0, x_l}, \]
	\[ a_{11} u(x_0) + a_{12} u'(x_0) = 0, ~ a_{21} u(x_l) + a_{22} u'(x_l) = 0. \]
	Решая данную задачу, и подставляя известное значение функции $v$, получим решение исходной краевой задачи.

	\vspace{10pt}

	Рассмотрим линейный неоднородный дифференциальный оператор с однородными смешанными краевыми условиями:
	\[ L[u] = f(x), ~ x \in \pares{x_0, x_l}, ~ L[u] = \difft{}{x} \bracks{p \cdot \difft{u}{x}} - q \cdot u, \]
	\[ B_0[u] = a_{11} u(x_0) + a_{12} u'(x_0) = 0, ~ B_l[u] = a_{21} u(x_l) + a_{22} u'(x_l) = 0. \]
	Функцией Грина данной краевой задачи называется такая функция $G(x, t)$, удовлетворяющая следующим условиям:
	\begin{enumerate}
		\item $G(x, t) \in \pares{C^2 \pares{x_0, x_l} \cap C\bracks{x_0, x_l}} \times \pares{C^1 \pares{x_0, x_l} \cap C\bracks{x_0, x_l}}$ -- непрерывна на квадрате, дважды дифференцируема по $x$, и дифференцируема один раз по $t$;
		\item $L_x \bracks{G(x, t)} = \delta\pares{x - t} = \syst{& 0: x \neq t \\ & 1: x = t }$ -- удовлетворяет однородному уравнению, когда $x \neq t$, и оператор равен $1$ при $x = t$;
		\item $B_0\bracks{G(x, t)} = B_l\bracks{G(x, t)} = 0$ -- удовлетворяет однородным краевым условиям по переменной $x$;
		\item $\lim\limits_{\varepsilon \to 0} \dpart{G}{x} \at_{x = t - \varepsilon}^{x = t + \varepsilon} = \frac{1}{p(t)}$ -- производная функции терпит разрыв первого рода со скачком в точке $x = t$.
	\end{enumerate}

	Функция Грина является обратным оператором для исходной краевой задачи. Решение можно записать в следующем виде:
	\[ u(x) = \int\limits_{x_0}^{x_l} G(x, t) \cdot f(t) ~ dt. \]
	Справедливость данного выражения можно доказать следующим образом:
	\[ L[u] = L_x\bracks{\int\limits_{x_0}^{x_l} G(x, t) \cdot f(t) ~ dt} = \int\limits_{x_0}^{x_l} L_x[G(x, t)] \cdot f(t) ~ dt = \int\limits_{x_0}^{x_l} \delta(x - t) \cdot f(t) ~ dt = f(x). \]
	Такая функция удовлетворяет уравнению. Краевые условия выполняются согласно условиям, наложенным на функцию $G$.

	Рассмотрим построение функции Грина. Положим следующее уравнение:
	\[ L[G(x, t)] = 0, ~ x \neq t. \]
	Рассмотрим следующие две области:
	\[ I_0(t) = [x_0, t), ~ I_l(t) = (t, x_l]. \]
	Пусть $u_0, u_l$ -- линейно-независимые частные решения однородного уравнения, при условии, что:
	\[ B_0[u_0] = 0, ~ B_l[u_l] = 0. \]
	Тогда пусть функция Грина принимает следующий вид:
	\[ G(x, t) = \syst{
		& C_1(t) \cdot u_0(x), ~ x \in I_0(t), \\
		& C_2(t) \cdot u_l(x), ~ x \in I_l(t).
	} \]
	Удовлетворим условие непрерывности функции в точке $x = t$:
	\[ \lim\limits_{\varepsilon \to 0} G(t - \varepsilon, t) = \lim\limits_{\varepsilon \to 0} G(t + \varepsilon, t), \]
	из чего следует:
	\[ C_1(t) \cdot u_0(t) = C_2(t) \cdot u_l(t). \]
	Затем удовлетворим условие разрывности производной функции в точке $x = t$:
	\[ \lim\limits_{\varepsilon \to 0} \dpart{G}{x} \at_{x = t - \varepsilon}^{x = t + \varepsilon} = \frac{1}{p(t)}, \]
	из чего следует:
	\[ C_2(t) \cdot u_l'(t) - C_1(t) \cdot u_0'(t) = \frac{1}{p(t)}. \]
	Рассматривая систему, построенную на двух данных условиях, возможно найти неизвестные функции $C_1(t)$ и $C_2(t)$. Тогда возможно определить единственную функцию Грина на основе конкретных функций $u_0$ и $u_l$.

	\subsection{Примеры}

		\begin{enumerate}

			\item Построим интегральную форму следующей краевой задачи:
				\[ y'' + y' - 2y = 10e^{-x} \cos{x}, ~ y(0) = 1, ~ y(\ln{2}) = -1. \]
				Введем функцию $v$, удовлетворяющую однородным условиям Дирихле:
				\[ v(0) = v(\ln{2}) = 0. \]
				Найдем функции $p(x)$ и $q(x)$. Домножим исходное уравнение на некоторой неизвестный интегрирующий множитель $\mu(x)$. Соответствующее уравнение для функций $p$ и $q$ можно найти из следующего соотношения:
				\[ \mu \cdot \pares{y'' + y' - 2y} = \difft{}{x} \bracks{p \cdot y'} - q \cdot y \]
				Раскрываем скобки, получим:
				\[ y'' \mu + y' \mu - 2y \mu = py'' + p'y' - qy, \]
				отсюда:
				\[ p = \mu, ~ p' = \mu, ~ q = 2 \mu. \]
				Найдем $p$:
				\[ p' = p \implies p = C e^{x}, \text{ пусть } C = 1, \]
				тогда
				\[ q = 2 e^{x}. \]
				Сведем исходное уравнение к операторному виду:
				\[ L[y] = \difft{}{x} \bracks{y' e^x} - 2 y e^x = 10 \cos{x}. \]
				Домножим уравнение на функцию $v$, и проинтегрируем:
				\[ \int\limits_{0}^{\ln{2}} \difft{}{x} \bracks{y' e^x} \cdot v ~ dx - 2 \int\limits_{0}^{\ln{2}} y e^{x} \cdot v ~ dx = 10 \int\limits_{0}^{\ln{2}} \cos{x} \cdot v ~ dx. \]
				Воспользуемся формулой интегрирования по частям для первого интеграла, получим:
				\[ y' e^{x} \cdot v \at_{x = 0}^{x = \ln{2}} - \int\limits_{0}^{\ln{2}} y' e^{x} \cdot v' ~ dx - 2 \int\limits_{0}^{\ln{2}} y e^{x} \cdot v ~ dx = 10 \int\limits_{0}^{\ln{2}} \cos{x} \cdot v ~ dx. \]
				Введем скалярное произведение следующим образом:
				\[ \pares{u, v} = \int\limits_{0}^{\ln{2}} u(x) \cdot v(x) ~ dx. \]
				Упростим выражение, получим следующую интегральную форму исходной краевой задачи:
				\[ \pares{y' e^{x}, v'} + 2 \pares{y e^{x}, v} = - 10 \pares{\cos{x}, v}. \]

			\item Построим решение следующей неоднородной краевой задачи для линейного неоднородного уравнения второго порядка с переменными коэффициентами:
				\[ \syst{
					&x \pares{2x - 1} y'' + 2 \pares{x - 1} y' - 2y = 6x \pares{x - 1} + 2, \\ 
					&y(1) + y'(1) = 3, ~ 2y(2) - y'(2) = 3 
				}\]
				Решение будем строить в виде линейной комбинации двух функций:
				\[ y = u + v, \]
				где функция $v$ удовлетворяет неоднородным краевым условиям, а для функции $u$ ставится новая краевая задача. Пусть $v$ -- некоторая линейная функция: $v = Ax + B$. Найдем коэффициенты $A$ и $B$:
				\[ \syst{
					& 2A + B = 3 \\
					& 3A + 2B = 3
				} \implies A = 3, B = -3, ~ v(x) = 3\pares{x - 1}. \]
				% Подставляя значение функции $v$ в исходное уравнение, получим, что $L[v] = 0$.
				Тогда новая краевая задача принимает следующий вид:
				\[ \syst{
					&x \pares{2x - 1} \pares{u + 3x - 3}'' + 2 \pares{x - 1} \pares{u + 3x - 3}' - 2 \pares{u + 3x - 3} = 6x \pares{x - 1} + 2, \\ 
					&u(1) + u'(1) = 0, ~ 2u(2) - u'(2) = 0
				} \]
				Раскроем скобки, упростим выражение:
				\[ \syst{
					&x \pares{2x - 1} u'' + 2 \pares{x - 1} u' - 2u = 6x \pares{x - 1} + 2, \\
					&u(1) + u'(1) = 0, ~ 2u(2) - u'(2) = 0 
				}\]
				Найдем значение функций $p(x)$ и $q(x)$ домножая исходное уравнение на интегрирующий множитель $\mu(x)$:
				\[ \mu \cdot \bpares{x \pares{2x - 1} u'' + 2 \pares{x - 1} u' - 2u} = p u'' + p'u' - q u. \]
				Отсюда следует:
				\[ p = \mu \cdot x \pares{2x - 1}, ~ p' = 2\mu \cdot \pares{x - 1}, ~ q = 2\mu. \]
				Найдем $p$:
				\[ \frac{p'}{p} = \frac{2 \pares{x - 1}}{x \pares{2x - 1}} \implies p(x) = \frac{C x^2}{2x - 1}, ~ \text{пусть } C = 1, \]
				тогда
				\[ \mu(x) = \frac{x}{\pares{2x - 1}^2}, ~ q(x) = \frac{2x}{\pares{2x - 1}^2}, ~ f(x) = \frac{2x \bpares{3x \pares{x - 1} + 1}}{\pares{2x - 1}^2} \]
				Таким образом, исходное уравнение принимает следующий вид:
				\[ \difft{}{x} \bracks{ \frac{x^2}{2x - 1} \cdot u' } - \frac{2x}{\pares{2x - 1}^2} = \frac{2x \bpares{3x \pares{x - 1} + 1}}{\pares{2x - 1}^2} \]
				Найдем однородные решения исходного уравнения:
				\[ x \pares{2x - 1} u'' + 2 \pares{x - 1} u' - 2u = 0. \]
				Подберем частное решение. Пусть $u = x^n$. Подставим в уравнение:
				\[ n\pares{n - 1} \cdot x \pares{2x - 1} x^{n - 2} + 2n \pares{x - 1} \cdot x^{n-1} - 2x^n = 0. \]
				Рассчитаем коэффициенты при старших степенях $x$:
				\[ 2n \pares{n - 1} x^n + 2n x^n - 2x^n = 0 \implies n = \pm 1. \]
				Найдем частное решение в виде полинома степени $n = 1$: $u = x + a$. Подставим в уравнение:
				\[ 2 \pares{x - 1} - 2 \pares{x + a} = 0 \implies a = -1. \]
				Частное решение однородного уравнения: $u = x - 1$. Найдем второе частное решение. Положим $u = z \cdot \pares{x - 1}, ~ z = z(x)$. Подставим в уравнение:
				\[ x\pares{2x^2 - 3x + 1} z'' + 2\pares{3x^2 - 3x + 1} z' = 0. \]
				Разделяя переменные, получим:
				\[ z' = C_1 \cdot \frac{1 - 2x}{x^2 \pares{x-1}^2} \implies z = C_1 \pares{\frac{1}{x-1} - \frac{1}{x}} + C_2. \]
				Подставляя и упрощая, получим общее решение однородного уравнения:
				\[ u = C_1 \pares{x - 1} + \frac{C_2}{x} \]

				Подберем два линейно-независимых частных решения, удовлетворящих однородным краевым условиям. Для этого найдем производную функции $u$:
				\[ u' = C_1 - \frac{C_2}{x^2}. \]
				Система для двух частных решений принимает вид:
				\[ \begin{matrix}
					\syst{&u_1 = C_1 \pares{x - 1} + \frac{C_2}{x}, \\ &u_1(1) + u_1'(1) = 0.} & 
					\syst{&u_2 = C_1 \pares{x - 1} + \frac{C_2}{x}, \\ &2u_2(2) - u_2'(2) = 0.} 
				\end{matrix} \]
				Подберем значения $C_1$ и $C_2$ для функций $u_1$ и $u_2$:
				\[ u_1: C_2 + C_1 - C_2 = 0; \quad u_2: 2C_1 + C_2 - C_1 + \frac{C_2}4 = 0. \]
				Выберем следующие функции:
				\[ u_1 = \frac{1}{x}, ~ u_2 = 5 \pares{x - 1} - \frac{4}{x}. \]
				Проверим их линейную независимость:
				\[ W(x) = \begin{vmatrix} 
					\dfrac{1}{x} & 5 \pares{x - 1} - \dfrac{4}{x} \\
					-\dfrac{1}{x^2} & 5 + \dfrac{4}{x^2}
				\end{vmatrix} = \frac{5}{x^2} \pares{2x - 1} \neq 0. \]

				Построим функцию Грина:
				\[ G(x, t) = \syst{
					& \frac{C_1(t)}{x}, ~ 1 \le x < t \le 2, \\
					& C_2(t) \cdot \pares{5 \pares{x - 1} - \frac{4}{x}}, ~ 1 \le t < x \le 2.
				} \]
				Выпишем условия непрерывности и разрывности функции Грина:
				\[ \syst{
					& \frac{C_1(t)}{t} = C_2(t) \cdot \pares{5 \pares{t - 1} - \frac{4}{t}} \\
					& C_2(t) \cdot \pares{5 + \frac{4}{t^2}} + \frac{C_1(t)}{t^2} = \frac{1}{p(t)} = \frac{2t - 1}{t^2}.
				} \]
				Запишем данную систему в матричной форме относительно неизвестных $C_1(t)$ и $C_2(t)$:
				\[ \begin{pmatrix} 
					\dfrac{1}{t} & -5 \pares{t - 1} + \dfrac{4}{t} \\
					\dfrac{1}{t^2} & 5 + \dfrac{4}{t^2}
				\end{pmatrix} \begin{pmatrix} C_1(t) \\ C_2(t) \end{pmatrix} = \begin{pmatrix} 0 \\ \dfrac{2t - 1}{t^2} \end{pmatrix} \]
				Определитель матрицы коэффициентов при $C_1(t)$, $C_2(t)$ известен, и равен $W(t)$. Тогда
				\[ C_1(t) = \frac{t^2}{5\pares{2t - 1}} \cdot \pares{5 \pares{t - 1} - \frac{4}{t}} \cdot \frac{2t - 1}{t^2} = \frac{1}{5} \pares{5 \pares{t - 1} - \frac{4}{t}}, \] 
				\[ C_2(t) = \frac{t^2}{5\pares{2t - 1}} \cdot \frac{1}{t} \cdot \frac{2t - 1}{t^2} = \frac{1}{5t}. \]
				Тогда функция Грина принимает следующий вид:
				\[ G(x, t) = \syst{
					& \frac{1}{5x} \cdot \pares{5 \pares{t - 1} - \frac{4}{t}}, ~ 1 \le x < t \le 2, \\
					& \frac{1}{5t} \cdot \pares{5 \pares{x - 1} - \frac{4}{x}}, ~ 1 \le t < x \le 2.
				} \] 

				Построим решение новой краевой задачи:
				\[ u(x) = \int\limits_1^2 G(x, t) \cdot f(t) ~ dt = \int\limits_1^x G_2(x, t) \cdot f(t) ~ dt + \int\limits_x^2 G_1(x, t) \cdot f(t) ~ dt = \circledast \]
				Рассмотрим каждый интеграл отдельно:
				\[
					I_1 = \int\limits_1^x G_2(x, t) \cdot f(t) ~ dt = \frac{1}{5} \cdot \pares{5 \pares{x - 1} - \frac{4}{x}} \cdot \int\limits_1^x \frac{1}{t} \cdot \frac{2t \bpares{3t \pares{t - 1} + 1}}{\pares{2t - 1}^2} ~ dt = \star
				\]
				\[ \begin{split} 
					I'_1 &= 2 \int\limits_1^x \frac{3t \pares{t - 1} + 1}{\pares{2t - 1}^2} ~ dt = \frac{1}{2} \int\limits_1^x 3 + \frac{1}{\pares{2t - 1}^2} = \frac{3}{2} t - \frac{1}{4 \pares{2t - 1}} \at_{t = 1}^{t = x} = \\
					&= \frac{3}{2} x - \frac{1}{4 \pares{2x - 1}} - \frac{5}{4} = \frac{1}{4} \cdot \pares{6x - 5 - \frac{1}{2x - 1}};
				\end{split} \]
				\[ 
					\star = \frac{1}{20} \cdot \pares{5 \pares{x - 1} - \frac{4}{x}} \cdot \pares{6x - 5 - \frac{1}{2x - 1}} = \frac{15x^4 - 35 x^3 + 13x^2 + 11x - 4}{5x \pares{2x - 1}};
				\]

				\[ 
					I_2 = \int\limits_x^2 G_1(x, t) \cdot f(t) ~ dt = \frac{1}{5x} \cdot \int\limits_x^2 \pares{5 \pares{t - 1} - \frac{4}{t}} \cdot \frac{2t \bpares{3t \pares{t - 1} + 1}}{\pares{2t - 1}^2} ~ dt = \star
				\]
				\[ \begin{split}
					I'_2 &= \int\limits_x^2 \pares{5 \pares{t - 1} - \frac{4}{t}} \cdot \frac{2t \bpares{3t \pares{t - 1} + 1}}{\pares{2t - 1}^2} ~ dt = \int\limits_x^2 \frac{15}{2} t^2 - \frac{15}{2} t - \frac{43}{8} - \frac{21}{8 \pares{2t - 1}^2} ~ dt = \\
					&= \frac{5}{2} t^3 - \frac{15}{4} t^2 - \frac{43}{8} t + \frac{21}{16\pares{2t - 1}} \at_{t = x}^{t = 2} = -\frac{39}{8} - \frac{5}{2} x^3 + \frac{15}{4} x^2 + \frac{43}{8} x - \frac{21}{16\pares{2x - 1}} = \\
					&= -\frac{5x^4 - 10x^3 - 7x^2 + 16x - 4}{2x - 1};
				\end{split} \]
				\[ 
					\star = -\frac{5x^4 - 10x^3 - 7x^2 + 16x - 4}{5x \pares{2x - 1}};
				\]
				\[ \begin{split}
					\circledast &= \frac{15x^4 - 35 x^3 + 13x^2 + 11x - 4}{5x \pares{2x - 1}} - \frac{5x^4 - 10x^3 - 7x^2 + 16x - 4}{5x \pares{2x - 1}} = \\
					&= \frac{10x^4 - 25 x^3 + 20x^2 - 5x}{5x \pares{2x - 1}} = \frac{2x^3 - 5x^2 + 4x - 1}{2x - 1} = \pares{x-1}^2.
				\end{split} \]
				Тогда общее решение принимает вид:
				\[ y = u + v = \pares{x-1}^2 + 3 \pares{x-1} = x^2 + x - 2. \]

		\end{enumerate}

	\subsubsection*{Замечание}

		На примере видно, что функция Грина является симметричной относительно аргументов $x$ и $t$. В общем случае данная функция может быть представлена в следующем виде:
		\[ G(x, t) = \syst{
			& \frac{u_0(x) \cdot u_l(t)}{W(t) \cdot p(t)}, ~ x_0 \le x < t \le x_l, \\
			& \frac{u_0(t) \cdot u_l(x)}{W(t) \cdot p(t)}, ~ x_0 \le t < x \le x_l,
		} \]
		где $u_0, u_l$ -- частные решения, удовлетворяющие первому и второму краевым условиям, $W(t)$ -- определитель Вронского системы двух частных решений, $p(t)$ -- коэффициент при старшей производной в рассматриваемом линейном дифференциальном операторе. 

	% \pagebreak