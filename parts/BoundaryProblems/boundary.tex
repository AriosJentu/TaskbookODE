\section{Краевые задачи, метод подстановки}

	В данном разделе будем рассматривать некоторые частные случаи краевых задач для обыкновенных дифференциальных уравнений.

	Введем понятие краевой задачи для обыкновенного дифференциального уравнения в общем случае. Рассмотрим уравнение на интервале следующего вида:
	\[ F \pares{x, y, y', \dots, y^{(n)}} = 0, ~ x \in \pares{x_0, x_l}. \]
	Краевыми условиями на функцию $y$ данного уравнения называется совокупность следующих дифференциальных операторов $(n-1)$-го порядка в точках $x_0$ и $x_l$:
	\[ B_i^{(n-1)} \bracks{y}\pares{x_0, x_l} = 0, ~ i = \overline{1, n}. \]
	Частный случай линейных краевых условий имеет следующий вид:
	\[ \sum_{j = 1}^{n} \pares{a_{ij} \cdot y^{(j-1)}\pares{x_0} + b_{ij} \cdot y^{(j-1)} \pares{x_l}} = c_i, i = \overline{1, n}, \]
	где \( a_{ij}, ~ b_{ij}, ~ c_i - \const ~ \forall i, j = \overline{1, n} \).
	
	Задача, содержащаяя в себе уравнение $n$-го порядка и $n$-соответствующих краевых условий называется краевой задачей. Рассматривать будем частные случаи краевых задач для уравнений второго порядка. Их классификацию будем проводить на основе следующих характеристик:
	\begin{enumerate}
		\item Условия вида \( y\pares{x_0} = y_0, ~ y\pares{x_l} = y_l \) называются краевыми условиями первого рода (условия Дирихле);
		\item Условия вида \( y'\pares{x_0} = y_0, ~ y'\pares{x_l} = y_l \) называются краевыми условиями второго рода (условия Неймана);
		\item Условия вида \( a_{11} y\pares{x_0} + a_{12} y'\pares{x_0} = y_0, ~ a_{21} y\pares{x_l} + a_{22} y'\pares{x_l} = y_l \), при условии \( a_{11}^2 + a_{12}^2 \neq 0, ~ a_{21}^2 + a_{22}^2 \neq 0 \) называются краевыми условиями третьего рода (смешанные краевые условия, условия Робена).
	\end{enumerate}
	Краевые условия называются однородными, если $y_0 = y_l = 0$.

	Рассмотрим линейное обыкновенное дифференциальное уравнение второго порядка:
	\[ y'' + p_1 \pares{x} \cdot y' + p_2 \pares{x} \cdot y = f(x), ~ x \in \pares{x_0, x_l}. \]

	Классическим решение краевой задачи Дирихле для данного уравнения будем называть такую функцию $y \pares{x} \in C^{2} \pares{x_0, x_l} \cap C\bracks{x_0, x_l}$, удовлетворяющую одновременно и уравнению и краевым условиям.

	Классическим решение краевой задачи Неймана и смешанной краевой задачи для данного уравнения будем называть такую функцию $y \pares{x} \in C^{2} \pares{x_0, x_l} \cap C^{1} \bracks{x_0, x_l}$, удовлетворяющую одновременно и уравнению и краевым условиям.

	Решение рассматриваемых в разделе можно проводить с помощью метода подстановки. Суть метода заключается в том, чтобы найти сначала общее решение уравнения, а затем, подставляя известные значения из краевыех условий, построить решение, удовлетворяющее краевой задаче.

	\subsection{Примеры}

		\begin{enumerate}
			\item Рассмотрим следующую краевую задачу для нелинейного обыкновенного дифференциального уравнения второго порядка:
				\[ \syst{
					&y'' \pares{x - \cos{y}} + 2y' + y'^2 \sin{y} = 0; \\ 
					&y\pares{0} = \pi, ~ y'\pares{1} + y\pares{1} = y'\pares{1} \cos{y\pares{1}}.
				} \]

				Представленное уравнение является уравнением с полной второй производной:
				\[ \pares{xy - \sin{y}}'' = 0, \]
				отсюда общее решение уравнения принимает вид:
				\[ xy - \sin{y} = C_1 x + C_2. \]
				Подставим первое краевое условие:
				\[ 0 \cdot \pi - \sin{\pi} = C_1 \cdot 0 + C_2 \implies C_2 = 0. \]
				Так как второе краевое условие не является линейным, но при этом в нем присутствует значение производной в точке, продифференцируем решение:
				\[ xy' + y - y' \cos{y} = C_1. \]
				Подставим $x = 1$ в это выражение, получим:
				\[ y' \pares{1} + y \pares{1} - y' \pares{1} \cos{y\pares{1}} = C_1. \]
				Данное выражение в точности повторит второе краевое условие, если $C_1 = 0$. Тогда общее решение краевой задачи принимает следующий вид:
				\[ xy = \sin{y}. \]

			\item Рассмотрим неоднородную смешанную краевую задачу для линейного неоднородного обыкновенного дифференциального уравнения второго порядка:
				\[ \syst{
					&y'' \pares{x^2 + 1} - 2xy' + 2y = 2; \\
					&y \pares{0} + y' \pares{0} = 0, ~ y' \pares{1} - y \pares{1} = 1.
				} \]
				Для однородного уравнения первое частное решение нетрудно подобрать: $y_{h_1} = x$. Построим общее однородное решение с помощью понижения порядка в линейном уравнении:
				\[ y = ux, ~ u = u \pares{x} \implies u'' x \cdot \pares{x^2 + 1} + 2u' = 0 \implies u' = C_1 \cdot \pares{1 + \frac{1}{x^2}} \implies u = C_1 \pares{x - \frac{1}{x}} + C_2. \]
				и общее однородное решение принимает вид:
				\[ y = C_1 \pares{x^2 - 1} + C_2 x. \]
				По виду уравнения нетрудно подобрать частное неоднородное решение:
				\[ y_{p} = 1. \]
				Тогда общее решение исходного уравнения принимает вид:
				\[ y = C_1 \pares{x^2 - 1} + C_2 x + 1. \]
				Для подстановки краевых условий, найдем производную от решения:
				\[ y' = 2 C_1 x + C_2. \]
				Согласно методу подстановки, составим систему уравнений для нахождения значений произвольных постоянных:
				\[ \syst{
					& \pares{- C_1 + 1} + \pares{C_2} = 0, \\
					& \pares{2C_1 + C_2} - \pares{C_2 + 1} = 1;  
				} \implies \syst{
					& C_1 - C_2 = 1, \\ &C_1 = 1;
				} \]
				из чего следует $C_1 = 1, C_2 = 0$. Тогда общее решение краевой задачи имеет следующий вид:
				\[ y = x^2. \]
				
		\end{enumerate}

	% \pagebreak
