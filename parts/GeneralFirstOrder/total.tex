\section{Уравнения в полных дифференциалах, интегрирующий множитель}
	
	Рассматривать будем уравнения следующего вида:
	\[ P(x, y) ~ dx  + Q(x, y) ~ dy = 0. \]
	Такие уравнения называются уравнениями в полных дифференциалах, если левая часть уравнения представляет собой полный дифференциал некоторой функции $u = u(x, y)$:
	\[ du = \dpart{u}{x} ~ dx + \dpart{u}{y} ~ dy. \]
	Для определения того, что изначальное уравнение является уравнением в полных дифференциалах, необходимо воспользоваться критерием Коши (критерием интегрируемости). Он заключается в том, что смешанные производные искомой функции равны на одном и том же интервале, если она непрерывна:
	\[ \dpartmix{u}{x}{y} = \dpartmix{u}{y}{x}. \]
	Полагая, что исходное уравнение представляет собой полный дифференциал некоторой функции $u$, согласно соответствиям $P(x, y) = \dpart{u}{x}$, $Q(x, y) = \dpart{u}{y}$, критерий интегрируемости будет выглядеть следующим образом:
	\[ \dpartmix{u}{y}{x} := \dpart{P}{y} = \dpart{Q}{x} =: \dpartmix{u}{x}{y}. \]

	В случае, если критерий интегрируемости выполняется, то представленное уравнение является уравнением в полных дифференциалах. Введем функцию $u$, для которой левая часть рассматриваемого уравнения является полным дифференциалом. Тогда получается, что уравнение можно переписать в виде следующей системы:
	\[ \left\lbrace \begin{split} du &= P(x, y) ~ dx + Q(x, y) ~ dy, \\ du &= 0 \end{split} \right. \]
	Сразу рассмотрим второе уравнение -- дифференциал некоторой функции равен нулю, соответственно сама функция является некоторой постоянной:
	\[ u(x, y) = C. \]
	Теперь необходимо решить первое уравнение в этой системе, чтобы определить вид самой функции $u(x, y)$. Для этого воспользуемся методом соответствий:
	\[ \dpart{u}{x} = P(x, y), ~ \dpart{u}{y} = Q(x, y). \]
	Если уравнению удовлетворяет одна и та же функция, то решение представленной системы уравнений с частными производными должно совпадать с точностью до постоянной. Данные уравнения представляют собой классические интегрируемые уравнения. Для их решения достаточно их проинтегрировать по той переменной, по которой изначально была найдена производная, при этом полагая оставшиеся переменные -- постоянными. Введем следующее обозначение частного интеграла от функции нескольких переменных:
	\[ \int f(x_1, x_2, \cdots, x_n) ~ \partial x_j = F(x_1, x_2, \cdots, x_n) + G(x_1, x_2, \cdots, x_{j-1}, x_{j+1}, \cdots, x_n), \]
	при этом:
	\[\dpart{}{x_j} \pares{F + G} = f. \]
	Здесь $F$ -- частный интеграл функции $f$ с точностью до функции $G$, которая не зависит от переменной интегрирования $x_j$.

	Интегрируя систему соответствий, получим:
	\[ \left\lbrace \begin{split} 
		\dpart{u}{x} &= P(x, y), \\
		\dpart{u}{y} &= Q(x, y),
	\end{split} \right. \implies \left\lbrace \begin{split} 
		u &= \int P(x, y) ~ \partial x + G_1(y), \\
		u &= \int Q(x, y) ~ \partial y + G_2(x)
	\end{split} \right. \]
	Частные интегралы функций $P(x, y)$ и $Q(x, y)$ в общем случае имеют следующую структуру:
	\[ \int P(x, y) ~ \partial x = F(x, y) + \varphi(x), ~ \int Q(x, y) ~ \partial y = F(x, y) + \psi(y). \]
	Здесь $F(x, y)$ -- общая часть, одновременно получаемая в ходе интегрирования обоих функций, это достигается за счет факта, что смешанные производные такой функции, согласно критерию, равны. Функции $\varphi(x)$ и $\psi(y)$ -- это частные интегралы от тех частей функций $P(x, y)$ и $Q(x, y)$, которые зависят только от переменной интегрирования. Тогда получаем систему для функции $u(x, y)$:
	\[ \left\lbrace \begin{split} u(x, y) &= F(x, y) + \varphi(x) + G_1(y) \\ u(x, y) &= F(x, y) + \psi(y) + G_2(x) \end{split} \right. \]
	Из факта того, что функция $u$ имеет одну и ту же структуру, согласно методу соответствий, все произвольные функции должны быть сопоставлены между известными, а значит:
	\[ G_1(y) \equiv \psi(y), ~ G_2(x) \equiv \varphi(x), \]
	и тогда функция $u$ принимает следующий вид:
	\[ u(x, y) = F(x, y) + \varphi(x) + \psi(y). \]

	Таким образом, система решения будет записана в следующем виде:
	\[ \left\lbrace \begin{split} u(x, y) &= F(x, y) + \varphi(x) + \psi(y), \\ u(x, y) &= C. \end{split} \right. \]
	В силу того, что функция $u(x, y)$ была введена как вспомогательная, и в изначальном уравнении отсутствовала, то выпишем общее решение согласно системе выше:
	\[ F(x, y) + \varphi(x) + \psi(y) = C. \]

	Если же критерий интегрируемости не выполняется, то решение невозможно построить согласно методу, описанному выше. В таком случае будем полагать, что исходное уравнение можно свести к уравнению в полных дифференциалах, если домножить его на некоторую функцию $\mu(x, y)$, называемую интегрирующим множителем:
	\[ \mu(x, y) \cdot P(x, y) ~ dx + \mu(x, y) \cdot Q(x, y) ~ dy = 0. \]
	Эта функция выбирается таким образом, чтобы критерий интегрируемости выполнялся:
	\[ \dpart{}{y} \bpares{\mu(x, y) \cdot P(x, y)} \equiv \dpart{}{x} \bpares{\mu(x, y) \cdot Q(x, y)}. \]
	Раскрывая производные, получим уравнение с частными производными первого порядка относительно неизвестной функции $ \mu(x, y) $:
	\[ Q\dpart{\mu}{x} - P\dpart{\mu}{y} = \mu \pares{\dpart{P}{y} - \dpart{Q}{x}}. \]
	Для упрощения записи, обозначим $R(x, y) = \dpart{P}{y} - \dpart{Q}{x}$. Подробнее решение таких уравнений рассмотренно в разделах ниже, на текущий момент будем полагать, что функцию $\mu(x, y)$ можно подобрать. К примеру, можно предположить, что функция $\mu(x, y)$ представляет собой функцию одной переменной. Рассмотрим случай только для переменной $x$, для переменной $y$ ситуация аналогична:
	\[ \mu(x, y) = \eta(x) \]
	Тогда:
	\[ \dpart{\mu}{x} = \eta'(x), ~ \dpart{\mu}{y} = 0, \]
	и
	\[ Q(x, y) \cdot \eta'(x) = R(x, y) \cdot \eta(x) \implies \frac{\eta'(x)}{\eta(x)} = \frac{R(x, y)}{Q(x, y)}. \]
	Если дробь $\frac{R(x, y)}{Q(x, y)}$ после упрощений представляет собой функцию только переменной $x$, то в таком случае интегрирующий множитель существует, и его можно представить в виде функции только переменной $x$ путем интегрирования полученного выше уравнения.

	\subsection{Примеры}

		\begin{enumerate}

			\item Рассмотрим следующий пример:
				\[ 2x \sin{y} ~ dx + x^2 \cos{y} ~ dy = e^{x} ~ dx. \]
				Приведем его к виду уравнения в полных дифференциалах:
				\[ \pares{2x \sin{y} - e^{x}} ~ dx + x^2 \cos{y} ~ dy = 0. \]
				Здесь $P(x, y) = 2x \sin{y} - e^{x}$, $Q(x, y) = x^2 \cos{y}$. Проверим критерий интегрируемости, для этого найдем соответствующие частные производные:
				\[ \dpart{P}{y} = 2x \cos{y}, ~ \dpart{Q}{x} = 2x \cos{y}. \]
				Как видим, критерий интегрируемости выполняется, так как соответствующие частные производные равны, а значит левая часть уравнения представляет собой полный дифференциал некоторой функции $u = u(x, y)$. Тогда, для функции $u$ введем следующую систему:
				\[ \left\lbrace \begin{split}
					du &= \pares{2x \sin{y} - e^{x}} ~ dx + x^2 \cos{y} ~ dy, \\
					du &= 0.
				\end{split} \right. \]
				Второе уравнение системы сразу дает решение $u(x, y) = C$, а с первым разберемся с помощью метода соответствий. Выпишем систему соответствий для производных
				\[ \left\lbrace \begin{split} 
					\dpart{u}{x} &= 2x \sin{y} - e^{x}, \\ 
					\dpart{u}{y} &= x^2 \cos{y}.
				\end{split} \right. \]
				Решим каждое уравнение путем частного интегрирования:
				\[ \left\lbrace \begin{split} 
					u &= x^2 \sin{y} - e^{x} + G_1(y), \\ 
					u &= x^2 \sin{y} + G_2(x).
				\end{split} \right. \]
				Здесь общая часть записана через выражение $x^2 \sin{y}$. Функции $G_2(x)$ соответствует функция $-e^{x}$ -- $G_2(x) \equiv -e^{-x}$, а для функции $G_1(y)$ соответствий нет -- $G_1(y) \equiv 0$. Тогда решению соответствует следующая система:
				\[ \left\lbrace \begin{split} u &= x^2 \sin{y} - e^{x}, \\ u &= C, \end{split} \right. \]
				а общее решение принимает вид:
				\[ x^2 \sin{y} - e^x = C. \]
				
			\item Рассмотрим следующее уравнение:
				\[ 2xy ~ dx + \pares{x^2 + 1} ~ dy = x^2 y \tan{y} ~ dy. \]
				Приведем его к виду уравнения в полных дифференциалах:
				\[ 2xy ~ dx + \pares{x^2 + 1 - x^2 y \tan{y}} ~ dy = 0. \]
				Здесь $P(x, y) = 2xy$, $Q(x, y) = x^2 + 1 - x^2 y \tan{y}$. Проверим критерий интегрируемости:
				\[ \dpart{P}{y} = 2x, ~ \dpart{Q}{x} = 2x - 2xy \tan{y}. \]
				Критерий интегрируемости не выполняется, соответственно данное уравнение не является уравнением в полных дифференциалах. Найдем интегрирующий множитель $\mu$. Вычислим предварительно функцию $R(x, y)$: 
				\[ R(x, y) = \dpart{P}{y} - \dpart{Q}{x} = 2xy \tan{y}. \]
				Положим, что $\mu(x, y) = \eta(x)$. Тогда соответствующая система для нахождения интегрирующего множителя примет вид:
				\[ \pares{x^2 + 1 - x^2 y \tan{y}} \cdot \eta'(x) = 2xy \tan{y} \cdot \eta(x). \]
				Нетрудно заметить, что чистую функцию переменной $x$ при разделении выражений на $Q(x, y)$ в данном случае получить невозможно, поэтому попробуем найти множитель в виде функции переменной $y$ -- $\mu(x, y) = \eta(y)$. Проводя те же операции, получим:
				\[ -2xy \cdot \eta'(y) = 2xy \tan{y} \cdot \eta(y). \]
				Упрощая выражение и разделяя переменные, получим:
				\[ \frac{\eta'(y)}{\eta(y)} = -\tan{y}. \]
				В данном случае получена интегрируемая конструкция. Проинтегрируем, получим:
				\[ \ln{\eta(y)} = \ln{\cos{y}} + \tilde{C}. \]
				Для интегрирующего множителя можно выбрать любое удобное значение $\tilde{C}$, в данном случае выберем $\tilde{C} = 0$. Спотенцируем, получим:
				\[ \eta(y) = \cos{y}. \]
				Домножим уравнение на данную функцию:
				\[ 2xy \cos{y} ~ dx + \pares{\pares{x^2 + 1}^{\vphantom{2}} \cdot \cos{y} - x^2 y \sin{y}} ~ dy = 0. \]
				Здесь $P(x, y) = 2xy \cos{y}$, $Q(x, y) = \pares{x^2 + 1} \cos{y} - x^2 y \sin{y}$. Проверим критерий интегрируемости:
				\[ \dpart{P}{y} = 2x \cos{y} - 2xy \sin{y}, ~ \dpart{Q}{x} = 2x \cos{y} - 2xy \sin{y}. \]
				Как можно видеть, критерий интегрируемости выполняется, значит представленное уравнение является уравнением в полных дифференциалах. Введем функцию $u = u(x, y)$, для которой сразу выпишем систему соответствий производных:
				\[ \left\lbrace \begin{split} \dpart{u}{x} &= 2xy \cos{y}, \\ \dpart{u}{y} &= \pares{x^2 + 1}^{\vphantom{2}} \cdot \cos{y} - x^2 y \sin{y}. \end{split} \right. \]
				Проинтегрируем уравнения:
				\[ \left\lbrace \begin{split} 
					u &= x^2 y \cos{y} + G_1(y), \\
					u &= \pares{x^2 + 1} \sin{y} + x^2 \pares{y \cos{y} - \sin{y}} + G_2(x). 
				\end{split} \right.\]
				Упростим второе равенство, получим $u = x^2 y \cos{y} + \sin{y} + G_2(x)$. Теперь, согласно методу соответствий, общая часть здесь $x^2 y \cos{y}$, функции $G_1(y)$ соответствует функция $\sin{y}$ -- $G_1(y) \equiv \sin{y}$, а функции $G_2(x)$ нет соответствий -- $G_2(x) \equiv 0$. Тогда общее решение принимает вид:
				\[ x^2 y \cos{y} + \sin{y} = C. \]
		\end{enumerate}

	\pagebreak