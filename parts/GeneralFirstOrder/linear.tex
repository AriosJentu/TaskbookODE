\section{Линейные уравнения}

	Рассматривать будем уравнения следующего вида:
	\[ a_0(x) \cdot y' + a_1(x) \cdot y = f(x). \]
	Уравнения, содержащие искомую функцию с ее производными в виде линейной комбинации с некоторыми известными функциями независимой переменной называются линейными уравнениями.

	Для линейных уравнений характерно свойство линейности для решений. Если известно $n$ частных решений $y_i(x)$ линейного однородного уравнения, то линейная комбинация таких частных решений также является однородным решением:
	\[ y_h(x) = \sum_{i = 1}^n C_i \cdot y_i(x). \]
	Общее решение линейного уравнения представляет собой сумму двух решений -- общего однородного и частного неоднородного:
	\[ y(x) = y_h(x) + y_{nh}(x). \]

	Определим классификацию таких уравнений:
	\begin{itemize}
		\item \textit{Порядок уравнения}. Определяется порядком старшей производной в уравнении. Рассматривать будем уравнения первого порядка;
		\item \textit{Приведенность}. Определяется коэффициентом при старшей производной. Если он равен $1$, то уравнение считается приведенным, иначе -- неприведенное;
		\item \textit{Однородность}. Определяется равенством нулю выражения, не зависимого от искомой функции. Если $f(x) = 0$ -- уравнение однородное, иначе -- неоднородное;
		\item \textit{Тип коэффициентов}. Уравнение может быть с \textit{постоянными} или с \textit{переменными} коэффициентами. В случае, если $a_i(x)$ -- постоянные для всех $i$ одновременно, то уравнение является уравнением с постоянными коэффициентами. Если существует хотя бы одно $i$, для которого $a_i(x)$ не постоянна, то уравнение является уравнением с переменными коэффициентами.
	\end{itemize}

	Стоит упомянуть, что линейные однородные уравнения первого порядка представляют собой уравнения с разделяющимися переменными. Здесь же рассмотрим только неоднородные случаи.

	Для линейных уравнений существует несколько методов решения. В данном разделе будем рассматривать метод вариации произвольной постоянной. Суть метода заключается в нахождении соответствующего однородного решения, а затем формирования из него общего решения путем представления произвольной постоянной как новой искомой функции, удовлетворяющей исходному неоднородному уравнению.

	Рассмотрим приведенное линейное неоднородное уравнение первого порядка:
	\[ y' + p(x) \cdot y = f(x). \]
	Прийти к такому виду из общего можно путем разделения исходного вида уравнения на функцию $a_0(x)$. Для нахождения однородного решения, рассмотрим соответствующее однородное линейное уравнение путем исключения из уравнения правой части:
	\[ y_h' + p(x) \cdot y_h = 0. \]
	Данное уравнение представляет собой уравнение с разделяющимися переменными. Его решение:
	\[ y_h = C \cdot \exp\pares{- \int p(x) ~ dx} = C \cdot F(x). \]
	Для нахождения общего решения исходного уравнения, согласно методу произвольной постоянной, положим $C$ -- новой функцией, которую будем искать такой, чтобы такое решение удовлетворяло исходному уравнению:
	\[ y = C(x) \cdot F(x), ~ y' = C'(x) \cdot F(x) + C(x) \cdot F'(x). \]
	Подставляя данное выражение в исходное уравнения, при том, что $F'(x) = -p(x) \cdot F(x)$, получим:
	\[ C'(x) \cdot F(x) - p(x) \cdot C(x) \cdot F(x) + p(x) \cdot C(x) \cdot F(x) = f(x). \]
	Упростим:
	\[ C'(x) \cdot F(x) = f(x). \]
	Тогда
	\[ C(x) = \int -\frac{f(x)}{F(x)} ~ dx = \frac{G(x)}{F(x)} + \tilde{C}. \]
	Для удобства результат интегрирования был записан в виде дроби. Тогда общее решение можно представить в следующем виде:
	\[ y = C \cdot F(x) + G(x), \]
	где
	\[ F(x) = \exp\pares{-\int p(x) ~ dx}, ~ G(x) = F(x) \cdot \int \frac{f(x)}{F(x)} ~ dx.\]

	\subsection{Примеры}

		Рассмотрим следующий пример:
		\[ y' \cdot e^x + y = 3e^{-x}. \]
		Классифицируем данное уравнение: это линейное неоднородное неприведенное уравнение первого порядка с переменными коэффициентами. Выпишем приведенное уравнение, разделив исходное на $e^x$:
		\[ y' + y \cdot e^{-x} = 3e^{-2x}. \]
		Согласно методу вариации произвольной постоянной, первым шагом найдем решение соответствующего однородного уравнения. Для этого исключим из уравнения слагаемые, не содержащие $y$:
		\[ y'_h + y_h \cdot e^{-x} = 0. \]
		Полученное уравнение является уравнением с разделяющимися переменными. Найдем $y_h$:
		\[ y_h = C \cdot e^{e^{-x}}. \]
		Теперь необходимо найти общее решение. Согласно методу, положим $C$ -- новой искомой функцией, удовлетворяющей исходному неоднородному уравнению:
		\[ y = C(x) \cdot e^{e^{-x}}. \]
		Найдем производную:
		\[ y' = C'(x) \cdot e^{e^{-x}} - C(x) \cdot e^{e^{-x}} \cdot e^{-x}. \]
		Подставим эти выражения в исходное приведенное уравнение:
		\[ \bracks{C'(x) \cdot e^{e^{-x}} - C(x) \cdot e^{e^{-x}} \cdot e^{-x}} + \bracks{C(x) \cdot e^{e^{-x}}} \cdot e^{-x} = 3e^{-2x}. \]
		Как можно видеть, выражения, содержащие только $C(x)$ взаимно исключаются, так как являются однородным решением. Остается:
		\[ C'(x) \cdot e^{e^{-x}} = 3e^{-2x}. \]
		Получили уравнение с разделяющимися переменными для неизвестной $C(x)$. Разделим переменные:
		\[ C'(x) = 3e^{-2x} \cdot e^{-e^{-x}}. \]
		Интегрируем, сделав в интеграле замену $u = -e^{-x}, ~ du = e^{-x} dx$:
		\[ C(x) = \int 3e^{-2x} \cdot e^{-e^{-x}} ~ dx = -\int 3u e^u ~ du = 3e^u - 3u e^{u} + \tilde{C} = 3e^{-e^{-x}} + 3e^{-x} \cdot e^{-e^{-x}} + \tilde{C}. \]
		Подставим полученное значение $C(x)$ в общее решение:
		\[ y = \bracks{3e^{-e^{-x}} + 3e^{-x} \cdot e^{-e^{-x}} + \tilde{C}} \cdot e^{e^{-x}} = \tilde{C} \cdot e^{e^{-x}} + 3e^{-x} + 3. \]
		Данное выражение является общим решением исходного уравнения. 

	\pagebreak