\section{Однородные уравнения}
	
	Рассматривать будем уравнения следующего вида:
	\[ F \pares{x, y, y'} = 0. \]
	Положим, что функция $F$ может быть представлена в следующем виде:
	\[ F \pares{x, y, y'} = G \bpares{x, v\pares{x, y}, v'\pares{x, y}}, \]
	где $v \pares{x, y}$ -- некоторая известная сложная функция, $v' \pares{x, y}$ -- ее производная. С помощью замены $u = v \pares{x, y}$, где $u$ -- новая искомая функция, уравнение примет более упрощенный вид:
	\[ G \pares{x, u, u'} = 0. \]

	Теперь введем понятие однородной функции. Рассмотрим некоторую произвольную функцию многих переменных:
	\[ F\pares{x_1, x_2, \dots, x_n}. \]
	Такая функция будет однородной, если справедливо следующее соотношение:
	\[ F\pares{k \cdot x_1, k \cdot x_2, \dots, k \cdot x_n} = k^m \cdot F\pares{x_1, x_2, \dots, x_n}. \]
	То есть, для однородных функций справедливо следующее утверждение: если домножить каждый аргумент функции на некоторый множитель $k$, то его возможно вынести как множитель из самой функции с некоторой степенью $m$. Здесь $k$, $m$ -- вещественные числа.

	Однородной заменой считается замена вида $y = ux$, где $u$ -- новая искомая функция, зависимая от переменной $x$. Уравнения, содержащие такую замену, называются однородными уравнениями. С помощью такой замены уравнение можно свести к более простому виду, к примеру -- уравнению с разделяющимися переменными.

	Уравнения, приводящиеся к однородным имеют следующий вид:
	\[ y' = \frac{a_1 x + b_1 y + c_1}{a_2 x + b_2 y + c_2}, \quad a_i, b_i, c_i ~ - ~ \const ~ \forall i = \overline{1, 2}. \]
	Метод их решения зависит от значений $a_i, b_i, c_i$. Стоит обратить внимание, что числитель и знаменатель представляют собой выражения для уравнения прямых. Обозначим их как $l_i: a_i x + b_i y + c_i = 0$. Рассмотрим два случая:

	\begin{enumerate}
		\item Если $l_1 \parallel l_2$:

			В таком случае $a_1 = k \cdot a_2$ и $b_1 = k \cdot b_2$, где $k$ -- коэффициент масштабирования. Такое уравнение можно свести к уравнению с разделяющимися переменными с помощью замены $u = a_i x + b_i y + d$, где $i$ выбирается произвольно из значений $\overline{1, 2}$, а $d$ -- любое удобное число для подстановки.

		\item Если $l_1 \nparallel l_2$:

			В таком случае уравнение можно свести к однородному с помощью параллельного переноса всего уравнения в точку пересечения прямых $\pares{x_0, y_0}$:
			\[ \left\lbrace \begin{split} u = x - x_0 \\ v = y - y_0 \end{split} \right. \]
			Здесь $u$ -- новый аргумент, $v$ -- новая функция, зависимая от переменной $u$. Тогда:
			\[ \difft{y}{x} = \difft{v}{u} = v'_u. \]
			После такой замены, уравнение примет вид:
			\[ v'_u = \frac{a_1 u + b_1 v}{a_2 u + b_2 v}, \]
			которое можно свести к уравнению с разделяющимися переменными с помощью замены $v = w \cdot u$, где $w$ -- новая искомая функция, зависимая от $u$.

	\end{enumerate}

	Обобщенные однородные уравнения -- уравнения, замена для которых строится в виде $y = u \cdot x^m$, где $u$ -- новая функция, зависимая от переменной $x$, а $m$ -- показатель однородности. Определить его можно путем подстановки замены в уравнение. В случае, если уравнение имеет вид суммы степенных выражений:
	\[ \sum_{k \ge 0}^n a_k \cdot x^{p_k} \cdot y^{q_k} \cdot y'^{r_k} = 0, \quad a_k, p_k, q_k, r_k ~ - ~ \const ~ \forall k, \]
	то коэффициент однородности можно найти с помощью соответствующего характеристического уравнения. Оно строится по следующим правилам:
	\begin{itemize}
		\item Операции сложения, вычитания и равенства заменяются на равенства;
		\item Операция умножения заменяется на сложение, деление на вычитание;
		\item Каждый $x^p$ заменяется на $p$;
		\item Каждый $y^q$ заменяется на $m \cdot q$, где $m$ -- показатель однородности, на момент составления его значение неизвестно;
		\item Каждый $y'^{r}$ заменяется на $(m - 1) \cdot r$.
	\end{itemize}
	Если из соответствующей системы удается найти значение $m$, в таком случае можно воспользоваться заменой $y = u \cdot x^m$, или $y = u^m$, где $u$ -- новая искомая функция, зависимая от переменной $x$.

	\subsection{Примеры}

		Рассмотрим следующий пример:
		\[ xy' = \tan{xy} - y. \]
		Стоит обратить внимание на то, что внутри функции $\tan$ находится сложная функция. Заменим ее на новую переменную:
		\[ u = xy, ~ u' = xy' + y, ~ u = u(x), ~ y = y(x). \]
		Тогда уравнение принимает следующий вид:
		\[ u' = \tan{u}. \]
		Это уравнение с разделяющимися переменными, и имеет следущее решение:
		\[ \sin{u} = C \cdot e^{x}. \]
		Вернемся к исходным переменным:
		\[ \sin{xy} = C \cdot e^{x}, \]
		что является общим решением данного уравнения.

		Рассмотрим другой пример:
		\[ y' = \frac{1 + x + 2y}{3 + 2x + 4y}. \]
		Как можно видеть, прямые, полученные из числителя и знаменателя -- параллельны. Такой вывод можно получить, построив определитель из коэффициентов при $x$ и $y$ в уравнениях:
		\[ \begin{vmatrix} 1 & 2 \\ 2 & 4 \end{vmatrix} = 0. \]
		Тогда положим следующую замену:
		\[ u = 1 + x + 2y, ~ u' = 1 + 2y' \implies y' = \frac{u' - 1}{2}, ~ u = u(x), ~ y = y(x). \]
		Подставим в уравнение, получим:
		\[ \frac{u' - 1}{2} = \frac{u}{2u + 1}. \]
		Упростим выражение:
		\[ u' = \frac{4u + 1}{2u + 1}. \]
		Полученное уравнение является уравнением с разделяющимися переменными. Разделим:
		\[ \frac{2u + 1}{4u + 1} ~ du = dx. \]
		Переменные разделены -- проинтегрируем:
		\[ \frac{1}{2} u + \frac{1}{8} \ln \abs{4u + 1} = x + C. \]
		Упростим выражение:
		\[ 4u + \ln\abs{4u + 1} = 8x + C \implies \pares{4u + 1} \cdot e^{4u} = C \cdot e^{8x}. \]
		Вернемся к исходной замене $u = 1 + x + 2y$:
		\[ \pares{4x + 8y + 5} \cdot e^{4x + 8y + 4} = C \cdot e^{8x}. \]
		Упростив, получим:
		\[ \pares{4x + 8y + 5} \cdot e^{8y - 4x + 4} = C, \]
		что является общим решением исходного уравнения.

		Рассмотрим еще один пример:
		\[ \pares{2x - y + 4} \cdot y' = x - 2y + 5. \]
		Сведем его к классическому виду:
		\[ y' = \frac{x - 2y + 5}{2x - y + 4}. \]
		Проверим, являются ли соответствующие прямые параллельными:
		\[ \begin{vmatrix} 1 & -2 \\ 2 & -1 \end{vmatrix} = 3 \neq 0. \]
		Прямые не являются параллельными, соответственно, найдем их точку пересечения:
		\[ \left\lbrace \begin{split} x - 2y &= -5 \\ 2x - y &= -4 \end{split} \right. \implies \left\lbrace \begin{split} x_0 &= -1 \\ y_0 &= 2 \end{split} \right. \]
		Теперь введем замену в уравнение:
		\[ \left\lbrace \begin{split} u &= x + 1 \\ v &= y - 2 \end{split} \right. \implies \left\lbrace \begin{split} x &= u - 1 \\ y &= v + 2 \end{split} \right. \implies y' = v'_u. \]
		Подставим замену в уравнение:
		\[ v'_u = \frac{(u - 1) - 2(v + 2) + 5}{2 (u - 1) - (v + 2) + 4} \implies v'_u = \frac{u - 2v}{2u - v}. \]
		Получили однородное уравнение. Сделаем замену:
		\[ v = u \cdot w, ~ v'_u = u \cdot w'_u + w, ~ w = w(u). \]
		Подставим:
		\[ uw'_u + w = \frac{u - 2uw}{2u - uw} \implies uw'_u = \frac{1 - 2w}{2 - w} - w \]
		Упростим:
		\[ uw'_u = \frac{1 - 4w + w^2}{2 - w}. \]
		Разделим переменные:
		\[ \frac{2 - w}{w^2 - 4w + 1} ~ dw = \frac{du}{u}. \]
		Интегрируя, получаем:
		\[ \ln\abs{w^2 - 4w + 1} = C - 2 \ln{u}. \]
		Упростим:
		\[ w^2 - 4w + 1 = \frac{C}{u^2}. \]
		Теперь вернемся к исходной замене, домножив на $u^2$:
		\[ \pares{y - 2}^2 - 4 \pares{y - 2} \pares{x + 1} + \pares{x + 1}^2 = C. \]
		Раскроем скобки, перенесем все свободные постоянные в $C$:
		\[ x^2 - 4xy + y^2 + 10x - 8 y = C. \]
		Полученное выражение является общим решением исходного уравнения.

		Рассмотрим последнее уравнение:
		\[ y' - \frac{y}{x} = y^2 + \recip{x^2}. \]
		Данное выражение представляет собой сумму степенных выражений. Попробуем найти коэффициент однородности. Для этого составим соответствующее характеристическое уравнение:
		\[ m - 1 = m - 1 = 2m = -2. \]
		Из данной системы следует, что $m = -1$.
		Тогда замена:
		\[ y = ux^{-1} \implies u = xy, ~ u' = xy' + y \implies y' = \frac{u'}{x} - \frac{u}{x^2}, ~ u = u(x), ~ y = y(x). \]
		Подставим замену в уравнение:
		\[ \frac{u'}{x} - \frac{u}{x^2} - \frac{u}{x^2} = \frac{u^2}{x^2} + \recip{x^2}. \]
		Упростим:
		\[ xu' = u^2 + 2u + 1. \]
		Получили уравнение с разделяющимися переменными. Его решение:
		\[ \recip{u + 1} = C - \ln{x}. \]
		Возвращаясь к исходным переменным:
		\[ xy + 1 = \recip{C - \ln{x}}. \]
		Таким образом получено общее решение исходного уравнения.


