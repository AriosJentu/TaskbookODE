\section{Уравнения, приводящиеся к линейным}

	Рассматривать будем уравнения следующего вида:
	\[ a_1(x) \cdot p'(x, y) + a_2(x) \cdot p(x, y) = f(x). \]
	Здесь $p(x, y)$ -- некоторое известное выражение, зависимое от неизвестной функции.
	Такие уравнения можно привести к линейным с помощью замены:
	\[ u = p(x, y), ~ u' = p'(x, y), ~ u = u(x), ~ y = y(x). \]
	При подстановке в исходное уравнение получим линейное уравнение, в котором в роли искомой функции выступает функция $u$:
	\[ a_1(x) \cdot u' + a_2(x) \cdot u = f(x). \] 
	Получив решение данного уравнения, и воспользовавшись обратной заменой, получим общее решение исходного:
	\[ p(x, y) = u(x), \]
	где $u(x)$ -- решение соответствующего линейного уравнения.

	Частным случаем приводящихся к линейным уравнений является уравнение Бернулли:
	\[ a_1(x) \cdot y' + a_2(x) \cdot y = f(x) \cdot y^{n}. \]
	В отличии от классического линейного уравнения, нелинейность достигается за счет того, что функция в правой части, отвечающая за неоднородность, стоит в произведении со степенью искомой функцией $y^n$.

	Построение линейного уравнения из уравнения Бернулли заключается в разделении всего уравнения на функцию $y^n$, и введении замены $u = y^{1-n}, ~ u = u(x), ~ y = y(x)$. Получим линейное уравнение:
	\[ \tilde{a}_1(x) \cdot u' + \tilde{a}_2(x) \cdot u = \tilde{f}(x). \]

	В случае, если в уравнении Бернулли $n = 1$, то исходное уравнение является уравнением с разделяющимися переменными. В случае, если $n = 0$, то исходное уравнение является обычным линейным уравнением. Для таких случаев замены производить не нужно.

	Другой случай уравнений, приводящихся к линейным -- уравнения Рикатти. Они имеют следующий вид:
	\[ y' = p(x) + q(x) \cdot y + r(x) \cdot y^2. \]
	Данные уравнения не являются линейными, так как присутствует выражение, содержащее $y^2$. При этом они не являются уравнением Бернулли в силу наличия выражения $p(x)$, не зависимого от искомой функции. Но его можно свести или к уравнению Бернулли, или к линейному уравнению второго порядка.

	Для того, чтобы свести данное уравнение к уравнению Бернулли, необходимо найти какое-либо частное решение данного уравнения $y_p$, и сделать замену: $y = u + y_p$, $u = u(x)$.
	Дифференцируем, и в силу того, что $y_p$ -- частное решение исходного уравнения, получим:
	\[ y' = u' + y_p' = u' + p(x) + q(x) \cdot y_p + r(x) \cdot y_p^2. \]
	Подставляя в исходное уравнение, получим:
	\[ u' + p(x) + q(x) \cdot y_p + r(x) \cdot y_p^2 = p(x) + q(x) \cdot \pares{u + y_p} + r(x) \cdot \pares{u + y_p}^2. \]
	Упростим:
	\[ u' = \bpares{q(x) + 2r(x) \cdot y_p} \cdot u + r(x) \cdot u^2. \]
	Получили уравнение Бернулли для новой искомой функции $u$.

	Для того, чтобы свести уравнение Рикатти к линейному уравнению второго порядка, нужно выполнить следующий алгоритм:
	\begin{enumerate}
		\item Избавиться от коэффициента при $y^2$, тем самым сведя уравнение к каноничному квадратному уравнению:
		\[ u = r(x) \cdot y, ~ y = \frac{u}{r(x)} \implies y' = \frac{u'}{r(x)} - \frac{r'(x) \cdot u}{r^2(x)}, ~ u = u(x), ~ y = y(x). \]
		Получим:
		\[ \frac{u'}{r(x)} - \frac{r'(x) \cdot u}{r^2(x)} = p(x) + \frac{q(x) \cdot u}{r(x)} + \frac{u^2}{r(x)}. \]
		Домножим уравнение на $r(x)$, и сделаем следующие переобозначения:
		\[ Q(x) = p(x) \cdot r(x), ~ P(x) = - q(x) - \frac{r'(x)}{r(x)}. \]
		Получим:
		\[ u' = u^2 - P(x) \cdot u + Q(x). \]
		\item Ввести замену для повышения порядка уравнения:
		\[ u = -\frac{v'}{v}, ~ u' = \frac{v'^2}{v^2} - \frac{v''}{v}, ~ v = v(x). \]
		Подставляя ее в уравнение, получим:
		\[ \frac{v'^2}{v^2} - \frac{v''}{v} = \pares{-\frac{v'}{v}}^2 + P(x) \cdot \frac{v'}{v} + Q(x). \]
		Упрощая выражение путем домножения на $v$, и приведения подобных слагаемых, получим:
		\[ v'' + P(x) \cdot v' + Q(x) \cdot v = 0, \]
		что представляет собой приведенное однородное линейное уравнение второго порядка.
	\end{enumerate}

	\subsection{Примеры}

		Рассмотрим следующий пример:
		\[ 2y' + \frac{\tan{x}}{x} \cdot e^{-2y} = - \drecp{x}. \]
		Домножим уравнение на $e^{y}$:
		\[ 2y' e^{y} + \frac{\tan{x}}{x e^{y}} = - \frac{e^{y}}{x}. \]
		Можно видеть, что если сделать замену $u = e^{y}$, $u = u(x)$, $y = y(x)$, то $u' = y' e^{y}$, получим:
		\[ 2u' + \frac{\tan{x}}{ux} = - \frac{u}{x}. \]
		Если поменять порядок слагаемых, можно увидеть, что данное уравнение представляет собой уравнение Бернулли:
		\[ 2u' + \frac{u}{x} = -\frac{\tan{x}}{x} \cdot u^{-1}, ~ u^{-1} = \frac{1}{u}. \]
		Домножим уравнение на $u$, получим:
		\[ 2uu' + \frac{u^2}{x} = - \frac{\tan{x}}{x}. \]
		Сделаем замену $v = u^2$, $v = v(x)$, тогда $v' = 2uu'$. Подставим в уравнение, получим:
		\[ v' + \frac{v}{x} = - \frac{\tan{x}}{x}. \]
		Получили линейное уравнение относительно функции $v(x)$. Общее решение этого уравнения:
		\[ v = \frac{\ln{\cos{x}}}{x} + \frac{C}{x}. \]
		Возвращаемся к исходной замене:
		\[ v = u^2, ~ u = e^{y} \implies v = e^{2y}. \]
		Домножим уравнение на $x$, и подставим замену, получим общее решение:
		\[ xe^{2y} = C + \ln{\cos{x}}. \]

		Рассмотрим другой пример:
		\[ y' + 2 \pares{x - y}^2 = \frac{y}{x}. \]
		Представленное уравнение содержит квадратное выражение для $y$, соответственно уравнение не является линейным. Раскроем скобки, и перенесем всё, что не зависит от $y'$ в правую часть:
		\[ y' = -2x^2 + \pares{\frac{1}{x} + 4x} \cdot y - 2y^2. \]
		Представленное уравнение является уравнением Рикатти. Нетрудно заметить, что подстановка $y = x$ в первом выражении даст верное равенство:
		\[ y = x, ~ y' = 1, ~ 1 + 2 \pares{x - x}^2 = \frac{x}{x} \implies 1 = 1. \]
		Тогда $y = x$ -- частное решение уравнения Рикатти. Сведем уравнение к уравнению Бернулли. Сделаем следующую замену:
		\[ y = u + x, ~ y' = u' + 1, ~ u = u(x). \]
		Подставим:
		\[ u' + 1 + 2 u^2 = \frac{u}{x} + 1, \]
		или, упростив:
		\[ u' - \frac{u}{x} = -2u^2. \]
		Получили уравнение Бернулли. Разделим его на $-u^2$, и сделаем замену $v = \frac{1}{u}$, $v' = - \frac{u'}{u^2}$, получим:
		\[ v' + \frac{v}{x} = 2. \]
		Данное уравнение представляет собой классическое линейное уравнение. Его решение:
		\[ v = \frac{x^2 + C}{x}. \]
		Вернемся к исходным заменам:
		\[ v = \frac{1}{u}, ~ u = y - x \implies v = \frac{1}{y - x}. \]
		Подставим замены, получим:
		\[ \frac{1}{y - x} = \frac{x^2 + C}{x}, \]
		или, выражая $y$:
		\[ y = x + \frac{x}{x^2 + C}, \]
		что является общим решением исходного уравнения Рикатти.

		Построим для данного уравнения соответствующее линейное уравнение второго порядка. Для этого воспользуемся уже найденной классической формой записи самого уравнения:
		\[ y' = -2x^2 + \pares{\frac{1}{x} + 4x} \cdot y - 2y^2. \]
		Здесь $p(x) = -2x^2$, $q(x) = \frac{1}{x} + 4x$, $r(x) = -2$. Приведем квадратное выражение к каноническому. Сделаем замену $y = -\frac{u}{2}$, $y' = -\frac{u'}{2}$, получим:
		\[ -\frac{u'}{2} = - 2x^2 - \pares{\frac{1}{x} + 4x} \cdot \frac{u}{2} - \frac{u^2}{2}. \]
		Упростим:
		\[ u' = u^2 + \pares{\frac{1}{x} + 4x} \cdot u + 4x^2. \]
		Повысим порядок уравнения с помощью замены $u = -\frac{v'}{v}$, $u' = \frac{v'^2}{v^2} - \frac{v''}{v}$. Получим:
		\[ \frac{v'^2}{v^2} - \frac{v''}{v} = \pares{-\frac{v'}{v}}^2 - \pares{\frac{1}{x} + 4x} \cdot \frac{v'}{v} + 4x^2. \]
		Упростим, домножим на $v$, получим:
		\[ v'' - \pares{\frac{1}{x} + 4x} \cdot v' + 4x^2 \cdot v = 0. \]
		Данное уравнение представляет собой линейное однородное уравнение второго порядка относительно функции $v$.

	\pagebreak