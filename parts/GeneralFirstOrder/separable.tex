\section{Уравнения с разделяющимися переменными}

	Рассмотрим уравнения следующего вида:
	\[ y' = f \pares{x, y}, \]
	где функция $f\pares{x, y}$ представляет собой произведение двух функций разных переменных:
	\[ f\pares{x, y} = P \pares{x} \cdot Q \pares{y}. \]
	Такие уравнения называются уравнениями с разделяющимися переменными. Метод решения заключается в том, чтобы разделить переменные по разные стороны от знака равенства, учитывая, что $y' = \difft{y}{x}$:
	\[ \difft{y}{x} = P \pares{x} \cdot Q \pares{y}. \]
	Разделим данное уравнение на $Q \pares{y}$, и умножим на $dx$:
	\[ \frac{dy}{Q \pares{y}} = P \pares{x} ~ dx. \]
	Левая часть, как и правая -- выражения, зависимые каждая только от одной переменной. Такое уравнение носит название уравнения с разделенными переменными. Соответственно, обе части данного равенства можно проинтегрировать:
	\[ \int \frac{dy}{Q \pares{y}} = \int P \pares{x} ~ dx. \]
	Интегрируя, получим:
	\[ \varphi \pares{y} = \psi \pares{x} + C, \]
	где $\difft{}{y} \varphi \pares{y} = \dfrac{1}{Q \pares{y}}$, и $\difft{}{x} \psi \pares{x} = P \pares{x}$. Полученное выражение является решением исходного уравнения с разделяющимися переменными. Общий вид уравнений с разделяющимися переменными можно представить следующим образом:
	\[ M_1 \pares{x} \cdot N_1 \pares{y} ~ dx = M_2 \pares{x} \cdot N_2 \pares{y} ~ dy. \]
	Разделяя данное уравнение на функцию $N_1 \pares{y} \cdot M_2 \pares{x}$, получим уравнение с разделенными переменными, для построения решения которого необходимо будет проинтегрировать обе части.

	Начальным условием для дифференциального уравнения называется выражение вида:
	\[ y \pares{x_0} = y_0. \]
	Пара дифференциальное уравнение и начальное условие называется задачей Коши. Решением задачи Коши является частное решение исходного уравнения, удовлетворяющее начальному условию.

	\subsection{Примеры}

		Рассмотрим следующий пример:
		\[ y' \cos^2{x} = \tan{y} + \cot{y}, ~ y\pares{0} = \frac{\pi}{2}. \]
		Для начала построим общее решение уравнения. Само уравнение имеет следующий вид:
		\[ M_1 \pares{x} y' = N_2 \pares{y} \]
		Здесь $M_1 \pares{x} = \cos^2{x}$, $N_1 \pares{y} = 1$, $M_2 \pares{x} = 1$ и $N_2 \pares{y} = \tan{y} + \cot{y}$.
		Разделим все уравнение на произведение $M_1 \pares{x} \cdot N_2 \pares{y}$, и домножим на $dx$, учитывая, что $y' = \difft{y}{x}$:
		\[ \frac{dy}{\tan{y} + \cot{y}} = \frac{dx}{\cos^2{x}}. \]
		И проинтегрируем обе части. Интеграл в правой части -- табличный:
		\[ \int \frac{dx}{\cos^2{x}} = \tan{x} + C_1, \]
		в то время как в левой части мы для начала упростим выражение:
		\[ \frac{1}{\tan{y} + \cot{y}} = \frac{\sin{y} \cdot \cos{y}}{\sin^2{y} + \cos^2{y}} = \sin{y} \cdot \cos{y}. \]
		Тогда интеграл в левой части имеет следующий вид:
		\[ \int \sin{y} \cdot \cos{y} ~ dy = \frac{1}{2} \sin^2{y} + C_2. \]
		Запишем решение в общем виде:
		\[ \frac{1}{2} \sin^2{y} + C_2 = \tan{x} + C_1. \]
		Упростим данную запись, домножив на $2$ и переобозначив $C_1 - C_2 = \dfrac{C}{2}$:
		\[ \sin^2{y} = 2 \tan{x} + C, \]
		что является общим решением рассматриваемого уравнения.
		Теперь построим решение задачи Коши -- найдем значение $C$, для которого данное решение будет удовлетворять начальному условию. Подставим значения $x_0 = 0$ и $y_0 = \dfrac{\pi}{2}$ в общее решение:
		\[ \sin^2{\frac{\pi}{2}} = 2 \tan{0} + C \implies 1 = 0 + C \implies C = 1. \]
		Тогда решение задачи Коши имеет следующий вид:
		\[ \sin^2{y} = 2 \tan{x} + 1. \]
	
	\pagebreak
