\section{Однородные уравнения высших порядков}

	Рассматривать будем частные случаи следующих уравнений:
	\[ F\pares{x, y, y', y'', \dots, y^{(n)}} = 0. \]
	
	\begin{enumerate}
		\item Однородным дифференциальным уравнением по степеням $ y $ будем называть уравнение вида
			\[ F\pares{x, y, y', y'', \dots, y^{(n)}} = 0, \]
			где $F$ -- однородная функция только относительно $y$ и его производных:
			\[ F\pares{x, k \cdot y, k \cdot y', k \cdot y'', \dots, k \cdot y^{(n)}} = k^m \cdot F \pares{x, y, y', y'', \dots, y^{(n)}}. \]
			Здесь $k, m$ будем называть коэффициентом и показателем однородности соответственно. Такие уравнения обычно имеют следующий вид:
			\[ \sum_{j = 0}^{p-1} \bracks{f_j(x) \cdot \prod_{i=0}^{n} \pares{y^{(i)}}^{a_{ij}}} = 0. \]
			Здесь $p$ -- количество слагаемых в уравнении, $f_j$ -- некоторые произвольные функции переменной $x$. Показатель однородности для каждого слагаемого в таких уравнениях можно вычислить по следующей формуле:
			\[ m_j = \sum_{i = 0}^n a_{ij}. \]
			Если все показатели однородности совпадают ($m_0 = m_1 = \dots = m_p$), то данное уравнение является однородным по степеням $y$. Понизить порядок в таком уравнении можно с помощью следующей замены:
			\[ y' = u \cdot y, \]
			где $u = u(x)$ -- новая искомая функция.

		\item Однородным дифференциальным уравнением относительно $x, y$ и их производных будем называть уравнения вида
			\[ F\pares{x, y, y', y'', \dots, y^{(n)}} = 0, \]
			где $F$ -- полная однородная функция. То есть, она удовлетворяет следующему условию:
			\[ F\pares{k \cdot x_1, k \cdot x_2, \dots, k \cdot x_n} = k^m \cdot F\pares{x_1, x_2, \dots, x_n}, \]
			где $k, m$ -- некоторые вещественные числа. Как и в случае выше, такие уравнения обычно имеют следующий вид:
			\[ \sum_{j = 0}^{p-1} \bracks{b_j \cdot x^{q_j} \cdot \prod_{i=0}^{n} \pares{y^{(i)}}^{a_{ij}}} = 0. \]
			Здесь $b_j$ -- некоторые вещественные числа. Метод решения таких уравнений схож с методом решения обобщенных однородных уравнений первого порядка. Решение будем строить в виде следующей функции:
			\[ y = u \cdot x^m, \]
			где $m$ -- показатель однородности. Для нахождения показателя однородности необходимо составить соответствующее характеристическое уравнение. Оно строится по следующим правилам:
			\begin{itemize}
				\item Операции сложения, вычитания и равенства заменяются на равенства;
				\item Операция умножения заменяется на сложение, деление на вычитание;
				\item Каждый $x^p$ заменяется на $p$;
				\item Каждый $y^q$ заменяется на $m \cdot q$, где $m$ -- показатель однородности, на момент составления его значение неизвестно;
				\item Каждый $\pares{y^{(k)}}^{r}$ заменяется на $(m - k) \cdot r$.
			\end{itemize}
			Если из соответствующей системы удается найти значение $m$, в таком случае можно применить следующую замену:
			\[ \left\lbrace \begin{split} 
				x &= e^{t}, \\
				y &= u \cdot e^{mt},
			\end{split} \right. \]
			где $u = u(t)$. Данная замена схожа с заменой для уравнений первого порядка $y = u \cdot x^m$. Дифференцирование данного выражения также проводится с помощью правил дифференцирования сложных функций:
			\[ y' = \difft{y}{x} = \frac{~\difft{y}{t}~}{~\difft{x}{t}~} = \frac{\pares{\difft{u}{t} + m \cdot u} \cdot e^{mt}}{e^{t}}. \]

	\end{enumerate}

	\subsection{Примеры}

		\begin{enumerate}
			\item Рассмотрим следующий пример:
				\[ xyy'' - xy'^2 = yy'. \]
				Как можно видеть, во всём уравнении $y$ со своими производными для каждого слагаемого входит во второй степени. Сделаем замену: $y' = u \cdot y$, где $u = u(x)$. Тогда $y'' = u' \cdot y + u \cdot y' = u' \cdot y + u^2 \cdot y = \pares{u' + u^2} \cdot y$. Подставим в уравнение:
				\[ \pares{u' + u^2} \cdot x y^2 - u^2 \cdot x y^2 = u \cdot y^2. \]
				Можно видеть, что $y^2$ -- общий множитель для всего уравнения. Полагая $y \neq 0$, сократим всё уравнение на $y^2$. Раскроем скобки, получим следующее уравнение:
				\[ xu' = u. \]
				Получили уравнение с разделяющимися переменными. Его решение:
				\[ u = C_1 x. \]
				Возвращаясь к исходной замене -- $y' = u \cdot y$, получим:
				\[ \frac{y'}{y} = C_1 x. \]
				Снова получено уравнение с разделяющимися переменными. Тогда общее решение исходного уравнения имеет следующий вид:
				\[ y = C_2 \cdot e^{C_1 x^2}. \]

			\item Рассмотрим другой пример:
				\[ x^4 y'' + 4y \cdot \pares{x^2 - y + xy'} = x^2y' \cdot \pares{3x + y'} + x^4. \]
				Раскроем скобки:
				\[ x^4 y'' + 4x^2 y - 4y^2 + 4xyy' = 3x^3y' + x^2 y'^2 + x^4 \]
				Проверим данное уравнение на однородность по степеням $x$ и $y$. Для этого составим соответствующее характеристическое уравнение для каждого слагаемого:
				\[ m + 2 = m + 2 = 2m = 2m = m + 2 = 2m = 4. \]
				Нетрудно увидеть, что $m = 2$ удовлетворяет этому характеристическому уравнению. Тогда введем замену:
				\[ \left\lbrace \begin{split} 
					x &= e^{t}, \\
					y &= u \cdot e^{2t},
				\end{split} \right. \]
				Найдем все производные функции $y$:
				\[ y' = \difft{y}{x} = \frac{~\difft{y}{t}~}{~\difft{x}{t}~} = \frac{\pares{\difft{u}{t} + 2u} \cdot e^{2t}}{e^{t}} = \pares{\difft{u}{t} + 2u} \cdot e^{t}, \]
				\[ y'' = \difft{y'}{x} = \frac{~\difft{y'}{t}~}{~\difft{x}{t}~} = \frac{\pares{\diffn{u}{t}{2} + 3\difft{u}{t} + 2u} \cdot e^{t}}{e^{t}} = \diffn{u}{t}{2} + 3\difft{u}{t} + 2u. \]
				Подставим полученные выражения в исходное уравнение, вынося из каждого слагаемого общий множитель $e^{4t}$:
				\[ e^{4t} \cdot \pares{\diffn{u}{t}{2} + 3\difft{u}{t} + 2u} + 4e^{4t} \cdot \bracks{u - u^2 + u \cdot \pares{\difft{u}{t} + 2u}} = e^{4t} \cdot \pares{\difft{u}{t} + 2u} \cdot \pares{3 + \difft{u}{t} + 2u} + e^{4t}. \]
				Сократим на $e^{4t}$, раскроем скобки и упростим:
				\[ \diffn{u}{t}{2} = \pares{\difft{u}{t}}^2 + 1. \]
				Получено уравнение второго порядка, содержащее только старшую производную, и производную на один порядок ниже. Подобное уравнение уже было решено ранее, его решение имеет вид:
				\[ u = \ln{\sec\pares{t + C_1}} + C_2. \]
				Теперь вернемся к исходной замене:
				\[ \left\lbrace \begin{split} 
					x &= e^{t}, \\
					y &= u \cdot e^{2t}.
				\end{split} \right. \implies \left\lbrace \begin{split}
					t &= \ln{x}, \\
					u &= \frac{y}{x^2}.
				\end{split} \right. \]
				Тогда общее решение принимает вид:
				\[ y = x^2 \cdot \ln{\sec{\pares{\ln{x} + C_1}}} + C_2 x^2. \]
		
		\end{enumerate}

	\pagebreak