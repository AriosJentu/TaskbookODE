\section{Уравнения, допускающие понижение порядка}

	Рассматривать будем уравнения следующего вида:
	\[ F\pares{x, y, y', y'', \dots, y^{(n)}} = 0. \]
	Такие уравнения называются обыкновенными дифференциальными уравнениями $n$-го порядка. В данном разделе рассмотрим некоторые частные случаи уравнений, в которых возможно понизить порядок:
	\begin{enumerate}
		\item Уравнения вида
			\[ y^{(n)} = f(x) \]
			назовем $n$-интегрируемым уравнением. Решение такого уравнения можно получить путем интегрирования уравнения последовательно $n$-раз:
			\[ y = \int \cdots \int f(x) ~ d^n x + C_1 x^{n-1} + C_2 x^{n-2} + \dots + C_{n-1} x + C_n. \]

		\item Уравнения вида
			\[ F\pares{x, y^{(n)}} = 0 \]
			называются уравнениями, содержащими только аргумент и производную старшего порядка. Такие уравнения можно свести к уравнениям первого порядка с помощью замены $y^{(n-1)} = u$, где $u = u(x)$. Уравнения вида 
			\[ F(x, u') = 0 \] 
			были рассмотрены в разделах ранее.

		\item Уравнения вида
			\[ F\pares{x, y^{(k)}, y^{(k+1)}, \dots, y^{(n)}} = 0 \]
			называются уравнениями, не содержащими искомой функции до производной $k$-го порядка. Порядок можно понизить с помощью замены $y^{(k)} = u$, где $u = u(x)$. Тогда уравнение принимает вид:
			\[ F\pares{x, u, u', \dots, u^{(n-k)}}. \]

		\item Уравнения вида
			\[ F\pares{y^{(k)}, y^{(k+1)}, \dots, y^{(n)}} = 0, ~ k \ge 0 \]
			называются уравнениями, не содержащие аргумента и искомой функции до производной $k$-го порядка. В таких уравнениях самая младшая производная становится новым аргументом -- $y^{(k)} = t$, а производная на порядок старше -- новой искомой функцией -- $y^{(k+1)} = u$, где $u = u(t)$. Производные для функции $y$ таким образом вычисляются как производные сложной функции. Пример для производной $(k+2)$-го порядка:
			\[ y^{(k+2)} = \difft{y^{(k+1)}}{x} = \difft{u}{t} \cdot \difft{t}{x} = \difft{u}{t} \cdot \difft{y^{(k)}}{x} = \difft{u}{t} \cdot u. \]
			В результате получим новое уравнения, порядок которого ниже исходного на $k+1$:
			\[ G\pares{t, u, \difft{u}{t}, \diffn{u}{t}{2}, \dots, \diffn{u}{t}{n-k-1}} = 0, \]
			где $G$ -- новая функция, отвечающая за уравнение, полученная из $F$ путем замены переменных.

		\item Уравнения вида
			\[ F\pares{y^{(n-1)}, y^{(n)}} = 0 \]
			называются уравнениями, содержащими только производную старшего порядка и производную на порядок ниже. Такие уравнения можно свести к уравнениям первого порядка с помощью замены $y^{(n-1)} = u$, где $u = u(x)$. Уравнения вида 
			\[ F(u, u') = 0. \]
			были рассмотрены в разделах ранее.

		\item Уравнения вида
			\[ F\pares{y^{(n-2)}, y^{(n)}} = 0 \]
			называются уравнениями, содержащими только производную старшего порядка и производную на два порядка ниже. Такие уравнения можно свести к уравнениям второго порядка, не содержащего арумента, с помощью замены $y^{(n-2)} = u$, где $u = u(x)$:
			\[ F(u, u'') = 0. \]
			Если такое уравнение можно разрешить относительно производной, то домножая его на интегрирующий множитель $2u'$, можно еще раз понизить порядок уравнения. Также, в таких уравнениях можно воспользоваться введением нового аргумента и новой функции -- $y^{(n-2)} = t$, $y^{(n-1)} = u$, где $u = u(t)$. Тогда уравнение будет сведено до уравнения первого порядка:
			\[ G\pares{t, u, \difft{u}{t}} = 0, \] 
			где $G$ -- новая функция, отвечающая за уравнение, полученная из $F$ путем замены переменных.

		\item Уравнения, в которых левая часть является точной производной некоторой функции
			\[ F\pares{x, y, y', y'', \dots, y^{(n)}} = G'\pares{x, y, y', y'' \dots, y^{(n-1)}} \]
			называются уравнениями, допускающие полное интегрирование. Понизить порядок в таком уравнении можно с помощью интегрирования:
			\[ G\pares{x, y, y', y'', \dots, y^{(n-1)}} = C_1. \]

	\end{enumerate}

	\subsection{Примеры}

		Рассмотрим следующий пример:
		\[ y''' = 6 - \cos{x}. \]
		Данное уравнение представляет собой $n$-интегрируемое уравнение третьего порядка. Построим общее решение путем тройного интегрирования самого уравнения:
		\[ y = \iiint 6 - \cos{x} ~ dx = x^3 + \sin{x} + C_1x^2 + C_2x + C_3. \]

		Рассмотрим другой пример:
		\[ x = 6y'' + e^{y''}. \]
		Данное уравнение является уравнением второго порядка, содержащим только аргумент и производную второго порядка. Сделаем замену: $y' = u$, где $u = u(x)$. Тогда $y'' = u'$. Получим:
		\[ x = 6u' + e^{u'}. \]
		Данное уравнение является уравнением первого порядка, неразрешенным относительно производной. Его параметрическое решение имеет следующий вид:
		\[ \left\lbrace \begin{split} 
			x &= 6p + e^p, \\
			u &= 3p^2 + \pares{p-1} \cdot e^p + C_1.
		\end{split} \right. \]
		Вернемся к исходной замене, тогда:
		\[ y' = 3p^2 + \pares{p-1} \cdot e^p + C_1, \]
		где $y = y(x)$. Получено уравнение с разделяющимися переменными, учитывая, что $x = x(p)$. Запишем интеграл:
		\[ y = \int 3p^2 + \pares{p-1} \cdot e^p + C_1 ~ dx. \]
		Для того, чтобы проинтегрировать полученное выражение по $x$, необходимо найти зависимость между $dx$ и $dp$:
		\[ dx = d \pares{6p + e^p} = \pares{6 + e^{p}} ~ dp. \]
		Тогда интеграл можно записать в следующей форме:
		\[ y = \int \pares{3p^2 + \pares{p-1} \cdot e^p} \cdot \pares{6 + e^p} ~ dp + \int C_1 ~ dx. \]
		Интегрируя, общее решение можно записать в виде системы:
		\[ \left\lbrace \begin{split} 
			x &= 6p + e^{p}, \\
			y &= 6p^3 + 3 \pares{p^2 - 2} e^{p} + \frac{e^{2p}}{4} \pares{2p - 3} + C_1 x + C_2.
		\end{split} \right. \]

		Рассмотрим третий пример:
		\[ \pares{1 + x^2} \cdot y'' + 2xy' = 0. \]
		Данное уравнение представляет собой уравнение второго порядка, не содержащее искомой функции до производной первого порядка. Понизим порядок уравнения с помощью замены $y' = u$, где $u = u(x)$. Тогда $y'' = u'$. Получим следующее уравнение:
		\[ \pares{1 + x^2} \cdot u' + 2xu = 0. \]
		Данное уравнение является уравнением с разделяющимися переменными, его решение:
		\[ u = \frac{C_1}{x^2 + 1}. \]
		Возвращаясь к исходной замене:
		\[ y' = \frac{C_1}{x^2 + 1}. \]
		Снова получено уравнение с разделяющимися переменными, его общее решение можно представить в следующем виде:
		\[ y = C_1 \arctan{x} + C_2. \]

		Рассмотрим следующий пример:
		\[ 1 + y'^2 = 2yy''. \]
		Данное уравнение представляет собой уравнение второго порядка, не содержащее аргумента и искомой функции до производной $0$-го порядка. Тогда пусть $y = t$ -- новый аргумент, а $y' = u$, где $u = u(t)$ -- новая искомая фунция. Тогда:
		\[ y'' = \difft{y'}{x} = \difft{u}{t} \cdot \difft{t}{x} = u \cdot \difft{u}{t}. \]
		Подставим в уравнение:
		\[ 1 + u^2 = 2tu \cdot \difft{u}{t}. \]
		Получено уравнение первого порядка с разделяющимися переменными. Его решение:
		\[ u = \pm \sqrt{C_1 t - 1}. \]
		Возвращаемся к исходной замене:
		\[ y' = \pm \sqrt{C_1 y - 1}. \]
		Снова получено уравнение с разделяющимися переменными. Интегрируя его, получим общее решение исходного уравнения:
		\[ \sqrt{C_1 y - 1} = C_2 \pm \frac{C_1x}{2}. \]

		Рассмотрим еще один пример:
		\[ y'' = y'^2 + 1. \]
		Данное уравнение является уравнением второго порядка, содержащим только старшую производную, и производную на порядок ниже. Сделаем замену $y' = u$, где $u = u(x)$. Тогда $y'' = u'$. Получим следующее уравнение:
		\[ u' = u^2 + 1. \]
		Данное уравнение является уравнением с разделяющимися переменными. Его решение:
		\[ u = \tan\pares{x + C_1}. \]
		Возвращаемся к исходной замене:
		\[ y' = \tan\pares{x + C_1}. \]
		Снова получено уравнение с разделяющимися переменными. Интегрируя его, получим общее решение исходного уравнения:
		\[ y = \ln{\sec\pares{x + C_1}} + C_2. \]

		Рассмотрим другой пример:
		\[ y''' = \frac{1}{y'^3}. \]
		Данное уравнение представляет собой уравнение третьего порядка, содержащее в себе старшую производную, и производную на два порядка ниже. Решим это уравнение двумя способами:
		\begin{enumerate}
			\item Введем замену $y' = u$, где $u = u(x)$. Тогда $y'' = u'$ и $y''' = u''$. Получим следующее уравнение:
				\[ u'' = \frac{1}{u^3}. \]
				Здесь можно домножить уравнение на интегрирующий множитель $2u'$, или же продолжить понижение порядка до первого путем введения нового аргумента и новой функции: $u' = v$, где $v = v(u)$, здесь $u$ из искомой функции становится новым аргументом. Воспользуемся интегрирующим множителем:
				\[ 2u' u'' = \frac{2u'}{u^3}. \]
				Левая и правая части представляют собой полные производные. Понизим порядок путем интегрирования уравнения:
				\[ u'^2 = C_1^2 - \frac{1}{u^2}. \]
				Извлекая квадратный корень, получим уравнение с разделяющимися переменными. Его решение:
				\[ C_1 \sqrt{C_1^2 - u^2} = C_2 \pm x. \]
				Возвращаясь к исходной замене, и выражая $y'$ в явном виде, получим:
				\[ y' = \pm \frac{1}{C_1} \sqrt{C_1^4 - \pares{C_2 \pm x}^2}. \]
				Интегрируя, получаем общее решение:
				\[ y = \frac{C_1^3}{2} \arcsin\pares{\frac{C_2 \pm x}{C_1^2}} + \frac{C_2 \pm x}{2 C_1} \cdot \sqrt{C_1^4 + \pares{C_2 \pm x}^2} + C_3. \]

			\item Введем замену $y' = t$, ~ $y'' = u$, где $u = u(t)$. Тогда:
				\[ y''' = \difft{y''}{x} = \difft{u}{t} \cdot \difft{t}{x} = \difft{u}{t} \cdot u. \]
				Получим следующее уравнение:
				\[ u \cdot \difft{u}{t} = \frac{1}{t^3}. \]
				Данное уравнение является уравнением с разделяющимися переменными. Его решение:
				\[ u^2 = C_1^2 - \frac{1}{t^2}. \]
				Возвращаясь к исходной замене, получим уравнение второго порядка:
				\[ y'' = C_1^2 - \frac{1}{y'^2}. \]
				Теперь данное уравнение представляет собой уравнение второго порядка, содержащее только старшую производную, и производную на один порядок ниже. Если ввести замену $y' = v$, где $v = v(x)$, тогда $y'' = v'$, получим в точности уравнение, которое представлено в способе выше:
				\[ v'^2 = C_1^2 - \frac{1}{v^2}. \]
				Дальнейшее решение аналогично методу выше.

		\end{enumerate}

		Рассмотрим последний пример:
		\[ yy'' + y'^2 + 2y^2 y'^2 = \frac{yy'}{x}. \]
		Разделим уравнение на $yy'$:
		\[ \frac{y''}{y'} + \frac{y'}{y} + 2yy' = \frac{1}{x}. \]
		Каждое слагаемое в данном уравнении представляет собой полную производную. Проинтегрируем:
		\[ \ln{y'} + \ln{y} + y^2 = \ln{x} + C_1. \]
		Спотенцируем:
		\[ yy' e^{y^2} = C_1 x. \]
		Получено уравнение с разделяющимися переменными. Его общее решение:
		\[ e^{y^2} = C_1 x^2 + C_2. \]
