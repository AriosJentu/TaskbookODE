\section{Уравнения, допускающие полное интегрирование}

	В данном разделе рассмотрим некоторые частные случаи интегрируемых уравнений второго порядка. В общем виде интегрируемые уравнения имеют следующий вид:
	\[ a_1 \pares{x, y, y'} + a_2 \pares{x, y, y'} \cdot y' + a_3 \pares{x, y, y'} \cdot y'' = 0, \]
	где левая часть представляет собой полную производную некоторой функции $v\pares{x, y, y'}$:
	\[ v'\pares{x, y, y'} = a_1 \pares{x, y, y'} + a_2 \pares{x, y, y'} \cdot y' + a_3 \pares{x, y, y'} \cdot y''. \]
	Здесь \( a_1 \pares{x, y, y'} = \dpart{v}{x}, ~ a_2 \pares{x, y, y'} = \dpart{v}{y}, ~ a_3 \pares{x, y, y'} = \dpart{v}{y'} \). Понизить порядок в таком уравнении возможно путем интегрирования всей функции:
	\[ v \pares{x, y, y'} = C_1. \]
	Общий случай таких уравнений рассмотрен в разделах выше, здесь же рассмотрим другой случай нелинейных уравнений второго порядка:
	\[ P(x, y) + 2 Q(x, y) \cdot y' + R(x, y) \cdot y'^2 + S(x, y) \cdot y'' = 0, \]
	где функции $P, Q, R, S$ -- непрерывны и дифференцируемы по своим переменным в некоторой области $\Omega$. Такое уравнение схоже с интегрируемым уравнением второго порядка, где \( a_1 \pares{x, y, y'} = P(x, y), ~ a_2 \pares{x, y, y'} = 2Q(x, y) + R(x, y) \cdot y', ~ a_3 \pares{x, y, y'} = S(x, y) \). Будем полагать, что левая часть такого уравнения представляет собой вторую полную производную некоторой функции $u(x, y)$:
	\[ u''(x, y) = P(x, y) + 2 Q(x, y) \cdot y' + R(x, y) \cdot y'^2 + S(x, y) \cdot y'', \]
	где
	\[ u''(x, y) = \dpartn{u}{x}{2} + 2 \dpartmix{u}{x}{y} \cdot y' + \dpartn{u}{y}{2} \cdot y'^2 + \dpart{u}{y} \cdot y''. \]
	Тогда такие уравнения называются уравнениями, допускающими полное интегрирование дважды. Для того, чтобы определить, является ли таким уравнением исходное, необходимо воспользоваться критерием Коши (критерием интегрируемости): 
	\[ \dpartmix{v}{x}{y} = \dpartmix{v}{y}{x}. \]
	В таком случае он будет состоять из трех условий -- двух основных, и одного связующего:
	\begin{enumerate}
		\item \( \dpart{P}{y} \equiv \dpart{Q}{x} := \frac{\partial^3 u}{\partial x^2 \partial y} \);
		\item \( \dpart{Q}{y} \equiv \dpart{R}{x} := \frac{\partial^3 u}{\partial x \partial y^2} \);
		\item \( \dpart{S}{y} \equiv R(x, y) := \dpartn{u}{y}{2} \) -- связующее условие между функциями $R(x, y)$ и $S(x, y)$. 

		Если в уравнении отсутствует функция $R(x, y)$, то связующее условие можно переписать через функцию $Q(x, y)$:
		\( \dpart{S}{x} \equiv Q(x, y) := \dpartmix{u}{x}{y}. \)
	\end{enumerate}

	Для построения функции $u$ найдем вспомогательные функции, отвечающие за различия между функциями $P, Q, R$ и $S$:
	\[ \vp(x) = P(x, y) - \int \dpart{Q}{x} ~ \partial y, \]
	\[ \psi(y) = R(x, y) - \int \dpart{Q}{y} ~ \partial x, \]
	\[ \alpha = S(x, y) - \int R(x, y) ~ \partial y = S(x, y) - \int Q(x, y) ~ \partial x. \]
	Здесь для простоты и целостности вычислений будем полагать, что неопределенные интегралы не содержат в себе произвольных функций и произвольных постоянных, тем самым выражая единственную первообразную. Тогда решение уравнения можно записать в следующем виде:
	\[ u(x, y) = \iint Q(x, y) ~ \partial x ~ \partial y + \iint \vp(x) ~ d^2 x + \iint \psi(y) ~ d^2 y + \alpha y, \]
	но так как функция $u(x, y)$ введена таким образом, что $u'' = 0$, то с другой стороны
	\[ u(x, y) = C_1 x + C_2. \]
	Тогда общее решение исходного уравнения можно записать в следующем виде:
	\[ \iint Q(x, y) ~ \partial x ~ \partial y + \iint \vp(x) ~ d^2 x + \iint \psi(y) ~ d^2 y + \alpha y = C_1 x + C_2. \]

	Частным случаем дважды интегрируемых уравнений являются линейные уравнения второго порядка:
	\[ P_1(x) + P_2(x) \cdot y + 2 Q(x) \cdot y' + S(x) \cdot y'' = 0. \]
	Критерий интегрируемости таких уравнений тогда принимает вид следующей системы:
	\begin{enumerate}
		\item \( \difft{Q}{x} \equiv P_2(x) \);
		\item \( \difft{S}{x} \equiv Q(x) \).
	\end{enumerate}
	Вычисляя функции $\vp(x), \psi(y)$ и постоянную $\alpha$, получим:
	\[ \vp(x) = P_1(x) + P_2(x) \cdot y - \int \difft{Q}{x} ~ \partial y = P_1(x) + P_2(x) \cdot y - P_2(x) \cdot y = P_1(x), \]
	\[ \psi(y) = 0 - \int \dpart{}{y} Q(x) ~ dx = 0 - 0 = 0, \]
	\[ \alpha = S(x) - \int Q(x) ~ dx. \]
	Тогда общее решение можно построить в следующем виде:
	\[ \iint Q(x) ~ dx ~ dy + \iint P_1(x) ~ d^2 x + \alpha y = C_1 x + C_2, \]
	или, вычислив первый интеграл, объединяя его с $\alpha y$, получим:
	\[ S(x) + \iint P_1(x) ~ d^2 x = C_1 x + C_2. \]
	Выражая $y$ в явном виде, получим:
	\[ y = \frac{1}{S(x)} \pares{C_1 x + C_2 - \iint P_1(x) ~ d^2 x}. \]

	\subsection{Примеры}

		\begin{enumerate}

			\item Рассмотрим следующий пример:
				\[ x^3 y'' \cos{2y} + 6 x^2 y' \cos{2y} + 3x \sin{2y} + y'' = 2 x^3 y'^2 \sin{2y}. \]
				Перепишем уравнение так, чтобы оно удовлетворяло виду уравнений допускающих интегрирование дважды:
				\[ 3x \sin{2y} + 6 x^2 \cos{2y} \cdot y' - 2 x^3 \sin{2y} \cdot y'^2 + \pares{x^3 \cos{2y} + 1} \cdot y'' = 0. \]
				Здесь $P(x, y) = 3x \sin{2y}$, $Q(x, y) = 3x^2 \cos{2y}$ (где $3 = \frac{6}{2}$), $R(x, y) = -2 x^3 \sin{2y}$ и $S(x, y) = x^3 \cos{2y} + 1$. Проверим критерий интегрируемости:
				\begin{enumerate}
					\item \( \dpart{P}{y} \equiv \dpart{Q}{x} := 6x \cos{2y} \);
					\item \( \dpart{Q}{y} \equiv \dpart{R}{x} := -6x^2 \sin{2y} \);
					\item \( \dpart{S}{y} \equiv R(x, y) := -2 x^3 \sin{2y} \).
				\end{enumerate}
				Критерий интегрируемости выполняется, соответственно представленное уравнение является уравнением, допускающим интегрирование дважды. Тогда выпишем вспомогательные функции:
				\[ \vp(x) = P(x, y) - \int \dpart{Q}{x} ~ \partial y = 3x \sin{2y} - \int 6x \cos{2y} ~ \partial y = 3x \sin{2y} - 3x \sin{2y} = 0; \]
				\[ \psi(y) = R(x, y) - \int \dpart{Q}{y} ~ \partial x = -2 x^3 \sin{2y} - \int -6x^2 \sin{2y} ~ \partial x = -2 x^3 \sin{2y} + 2 x^3 \sin{2y} = 0; \]
				\[ \alpha = S(x, y) - \int R(x, y) ~ \partial y = x^3 \cos{2y} + 1 + \int 2 x^3 \sin{2y} ~ dy = x^3 \cos{2y} + 1 - x^3 \cos{2y} = 1. \]
				Тогда общее решение уравнения можно записать в виде следующего выражения:
				\[ \iint Q(x, y) ~ \partial x ~ \partial y + y = C_1 x + C_2. \]
				Подставляя значения, получим такой вид:
				\[ \iint 3x^2 \cos{2y} ~ \partial x ~ \partial y + y = C_1 x + C_2. \]
				Интегрируя, получаем общее решение исходного уравнения:
				\[ \frac{1}{2} x^3 \sin{2y} + y = C_1 x + C_2, \]
				или, домножая на $2$, в силу произвольности постоянных:
				\[ x^3 \sin{2y} + 2y = C_1 x + C_2. \]

			\item Рассмотрим другой пример:
				\[ y'' \pares{\cos{x} - 3} - 2y' \sin{x} - y \cos{x} = \pares{x - 2} \cdot e^{-x}. \]
				Данное уравнение является линейным неоднородным уравнением второго порядка. Приведем его к виду уравнения допускающего интегрирование дважды:
				\[ \pares{x - 2} e^{-x} + y \cdot \cos{x} + 2y' \cdot \sin{x} - y'' \cdot \pares{\cos{x} - 3} = 0. \]
				Здесь $P_1(x) = \pares{x - 2} e^{-x}$, $P_2(x) = \cos{x}$, $Q(x) = \sin{x}$, $S(x) = 3 - \cos{x}$. Проверим критерий интегрируемости:
				\begin{enumerate}
					\item \( \difft{Q}{x} \equiv P_2(x) := \cos{x} \);
					\item \( \difft{S}{x} \equiv Q(x) := \sin{x} \).
				\end{enumerate}
				Представленное уравнение является уравнением, допускающим интегрирование дважды. Тогда общее решение запишем в виде следующего интеграла:
				\[ y = \frac{1}{S(x)} \pares{C_1 x + C_2 - \iint P_1(x) ~ d^2 x}. \]
				Подставим известные значения:
				\[ y = \frac{1}{3 - \cos{x}} \pares{C_1 x + C_2 - \iint \pares{x - 2} e^{-x} ~ d^2 x}. \]
				Интегрируя и упрощая, получаем следующее общее решение:
				\[ y = \frac{C_1x + C_2 + xe^{-x}}{\cos{x} - 3}. \]
		
		\end{enumerate}
