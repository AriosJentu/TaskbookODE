\section{Обобщенное уравнение Клеро}

	Рассмотрим следующий вид двумерного нелинейного уравнения с частными производными первого порядка:
	\[ u = px + qy + \varphi \pares{p, q}. \]

	Такое уравнение называется двумерным обобщенным уравнением Клеро. Построим его решение методом Лагранжа-Шарпи. Для этого найдем вспомогательное уравнение $G \pares{u, x, y, p, q}$. Положим
	\[ F = u - px - qy - \varphi \pares{p, q}. \]
	Найдем соответствующие производные функции $F$:
	\[ \dpart{F}{u} = 1 , ~ \dpart{F}{x} = -p, ~ \dpart{F}{y} = -q, ~ \dpart{F}{p} = - x - \dpart{\varphi}{p}, ~ \dpart{F}{q} = - y - \dpart{\varphi}{q}. \]

	Теперь построим вспомогательное квазилинейное уравнение для функции $G$. Найдем соответствующие коэффициенты уравнения:
	\[ \begin{split}
		U &= q \cdot \pares{y + \dpart{\varphi}{q}} + p \cdot \pares{x + \dpart{\varphi}{p}}, \\
		X &= x + \dpart{\varphi}{p}, ~ Y = y + \dpart{\varphi}{q}, \\
		P &= - p + p = 0, ~ Q = - q + q = 0.
	\end{split} \] 
	Получим уравнение:
	\[ \pares{px + qy + p\dpart{\varphi}{p} + q \dpart{\varphi}{q}} \cdot \dpart{G}{u} + \pares{x + \dpart{\varphi}{p}} \cdot \dpart{G}{x} + \pares{y + \dpart{\varphi}{q}} \cdot \dpart{G}{y} = 0. \]
	В этом уравнении отсутствуют производные по переменным $p$ и $q$, при этому сами переменные в уравнении присутствуют. Из этого следует, что $p$ и $q$ -- параметры уравнения, или, что то же -- какие-либо постоянные, от которых зависит общее решение в явном виде. Наша задача состоит в том, чтобы построить хотя бы одно частное решение. Нетрудно показать, что для этой системы выполняется условие вполне интегрируемости:
	\[ du = p ~ dx + q ~ dy. \]
	Так как $p$ и $q$ относительно уравнения для функции $G$ -- постоянные, то первый интеграл системы имеет вид:
	\[ u = px + qy + C. \]
	Убедимся, что постоянные $p$ и $q$ являются произвольными. Для этого в соответствующей симметрической характеристической системе рассмотрим следующее выражение:
	\[ \frac{dp}{0} = \frac{dq}{0}. \]
	Это равенство верно согласно принципу равенства дробей при условии, что $p$ и $q$ -- какие-либо постоянные ($dp = dq = 0$ одновременно), что в данном случае выполняется. Здесь или $p = C_1$ или $q = C_1$, где $C_1$ -- произвольная постоянная. Положим $q = C_1$, таким образом вполне интегрируемая система примет вид:
	\[ \left\lbrace \begin{split}
		u &= px + qy + \varphi \pares{p, q}, \\
		q &= C_1.
	\end{split} \right. \]
	Подставим второе уравнение в первое, получим:
	\[ u = p x + C_1 y + \varphi \pares{p, C_1}. \]
	Положим $C_1, y$ параметрами уравнения, перепишем его в следующем виде:
	\[ u = px + \psi \pares{p, C_1, y}. \]
	В данном уравнении присутствует помимо функции $u$, ее производная по переменной $x$, домноженная как раз соответствующую ей переменную. Так же функция $\psi$ в общем случае зависит только от производной функции $u$, в рамках переменной $x$, параметры $C_1$ и $y$ могут интерпретироваться как фиксированные постоянные. Таким образом получено классическое одномерное параметрическое уравнение Клеро, в ходе решения которого получим, что $p = C_2$, где $C_2$ -- вторая произвольная постоянная, отличная от $C_1$. Производя переобозначения индексов произвольных постоянных, получим общее решение:
	\[ u = C_1 x + C_2 y + \varphi \pares{C_1, C_2}. \]
	Аналогичный результат можно получить, положив $p = C_1$ из решения соответствующей симметрической характеристической системы. Из этой же системы, очевидно, вытекает следствие, что $p$ и $q$ должны являться произвольными постоянными одновременно.

	Таким образом, общее решение двумерного обобщенного уравнения Клеро имеет вид:
	\[ u = C_1 x + C_2 y + \varphi \pares{C_1, C_2}. \]

	Эти следствия можно расширить в общем случае для больших размерностей. Обобщенное уравнение Клеро имеет вид:
	\[ u = \sum_{k = 1}^{n} x_k \cdot \dpart{u}{x_k} + \varphi \pares{\dpart{u}{x_1}, \dots, \dpart{u}{x_n}}, \]
	а его общее решение, соответственно:
	\[ u = \sum_{k = 1}^{n} C_k \cdot x_k + \varphi \pares{C_1, \dots, C_n}. \]

	% \pagebreak
	\subsubsection*{Замечание}
		Стоит упомянуть, что в системе
		\[ \frac{dp}{0} = \frac{dq}{0} \]
		можно построить сразу два первых интеграла, и, в силу того, что эта система была основана на исходном уравнении, то, согласно методу Лагранжа-Шарпи, можно построить систему из двух уравнений сразу в явном виде:
		\[ \left\lbrace \begin{split} p &= C_1, \\ q &= C_2. \end{split} \right. \]
		При подстановке в уравнение Пфаффа, получили бы сразу полный интеграл
		\[ u = C_1 x + C_2 y + C, ~ \forall C. \]
		Далее, методом соответствия уравнения его решению получим, что $C = \varphi \pares{C_1, C_2}$, так как $p$ и $q$ -- произвольные постоянные. Соответственно, общее решение принимает следующий вид:
		\[ u = C_1 x + C_2 y + \varphi \pares{C_1, C_2}. \]


	% \pagebreak
	\subsection{Примеры}

		Рассмотрим следующий пример:
		\[ u = px + qy + pq. \]
		Используя полученные ранее выкладки, из симметрической характеристической системы получим:
		\[ \frac{dp}{0} = \frac{dq}{0}. \]
		На основе этой системы получим сразу два первых интеграла, удовлетворяющих самому уравнению:
		\[ p = C_1, ~ q = C_2. \]
		Подставим в уравнение Пфаффа, и проинтегрируем:
		\[ u = C_1 x + C_2 y + C, ~ \forall C. \]
		Методом соответствия сравним этот результат с исходным уравнением, полагая, что $p$ и $q$ -- произвольные постоянные, получим, что
		\[ C = C_1 C_2, \]
		следовательно, общее решение принимает вид:
		\[ u = C_1 x + C_2 y + C_1 C_2. \]

		\vspace{30pt}

		Рассмотрим другой пример:
		\[ 2x \cdot \pares{u - qy} = pq + px^2. \]
		Данное уравнение не является уравнением Клеро в явном виде, но его можно к нему привести. Положим, например $t = x^2$, тогда
		\[ u = u \pares{x, y} = u \pares{t, y}. \]
		Найдем соответствующие производные:
		\[ p = \dpart{u}{x} = \dpart{u}{t} \cdot \difft{t}{x} = 2x \cdot \dpart{u}{t}. \]
		Положим
		\[ p_1 = \dpart{u}{t} ~ \implies p = 2x \cdot p_1. \]
		Переведем уравнение от переменных $u, x, y$ в переменные $u, t, y$, оставив $2x$ как общий множитель:
		\[ 2 x \cdot \pares{u - qy} = 2 x \cdot p_1 q + 2 x \cdot p_1 t. \]
		Сокращая на $2 x$, полагая $x \neq 0$, и упрощая, получим:
		\[ u = p_1 t + qy + p_1 q. \]
		Данное уравнение является уравнением Клеро для $u \pares{t, y}$. Тогда общее решение принимает вид:
		\[ u = C_1 t + C_2 y + C_1 C_2, \]
		или, подставляя $t = x^2$, получим:
		\[ u = C_1 x^2 + C_2 y + C_1 C_2. \]

		\vspace{30pt}

		Помимо замен независимых переменных $x$ и $y$, в уравнении могут присутствовать и замены функции $u$, включая и комбинации таких замен. Рассмотрим следующий пример:
		\[ 2y u = y^2 q + \pares{q + 2y} \cdot \pares{p x + u}. \]
		Здесь можно обратить внимание, что в левой части функция $u$ стоит с коэффициентом $2y$, а в правой части первым слагаемым $q$ стоит с коэффициентом $y^2$. Так что стоит попробовать сделать замену $t = y^2$, что, возможно, упростит дальнейшее вычисление:
		\[ t = y^2, ~ u = u \pares{x, y} = u \pares{x, t}. \]
		Найдем соответствующие производные:
		\[ q = \dpart{u}{y} = \dpart{u}{t} \cdot \difft{t}{y} = 2y \cdot \dpart{u}{t} = 2y \cdot q_1. \]
		Подставим в уравнение, оставляя $2y$ как общий множитель, а $y^2$ заменяя на $t$:
		\[ 2y u = 2y t q_1 + 2y \cdot \pares{q_1 + 1} \cdot \pares{px + u}. \]
		Сократим на $2y$, полагая $y \neq 0$, раскроем скобки, приведем подобные слагаемые:
		\[ t q_1 + p q_1 x + u q_1 + px = 0. \]
		Данное уравнение все еще не является обобщенным уравнением Клеро, поэтому попробуем провести ещё какие-либо замены. Стоит обратить внимание, что в исходном уравнении присутствует конструкция $px + u$. В дифференциальной форме она имеет следующий вид:
		\[ px + u = x \dpart{u}{x} + u = \dpart{}{x} \pares{ux}. \]
		В таком случае, попробуем сделать замену $v = ux$. Найдем соответствующие производные:
		\[ p_2 = \dpart{v}{x} = px + u, \quad q_2 = \dpart{v}{t} = x q_1. \]
		Выразим конструкции $u$, $p$ и $q_1$:
		\[ u = \frac{v}{x}, \quad p = \frac{p_2 - u}{x} = \frac{p_2x - v}{x^2}, \quad q_1 = \frac{q_2}{x}. \]
		Подставим в полученное уравнение:
		\[ \frac{t}{x} \cdot q_2 + \frac{p_2x - v}{x} \cdot \frac{q_2}{x} + \frac{v}{x} \cdot \frac{q_2}{x} + \frac{p_2x - v}{x} = 0. \]
		Аналогично, упростим, приведем подобные, получим:
		\[ v = p_2 x + q_2 t + p_2 q_2. \]
		В результате было получено обобщенное уравнение Клеро, его общее решение в переменных $v \pares{x, t}$ имеет вид:
		\[ v = C_1 x + C_2 t + C_1 C_2. \]
		Проводя обратные замены, получим:
		\[ ux = C_1 x + C_2 y^2 + C_1 C_2, \]
		что является общим решением поставленного уравнения.

		\vspace{20pt}
		\subsubsection*{Замечание}
			Стоит заметить, что в общем случае проводить замены независимых переменных проще в комбинации, даже если замена является незначительной. К примеру, уравнение может являться уравнением Клеро для переменных $x + y$ и $y$, в таком случае стоит проводить замену совместно следующим образом: 
			\[ t = x + y, ~ s = y. \]

		% \pagebreak
		\subsubsection*{Возможные замены}
			\begin{enumerate}
				\item Если присутствует $n \cdot px$, $n \in \mathbb{R}$, то возможна замена \( t = x^{\frac{1}{n}} \);
				\item Если присутствует $n \cdot qy$, $n \in \mathbb{R}$ то возможна замена \( t = y^{\frac{1}{n}} \);
				\item Если присутствует $- p^2$ или $- q^2$, то возможна замена: \( t = xy \);
				\item Если присутствует $+ p^2$, то возможна замена: \( t = \frac{y}{x} \);
				\item Если присутствует $+ q^2$, то возможна замена: \( t = \frac{x}{y} \);
				\item Если присутствует $u + px$, то возможна замена: \( v = ux \);
				\item Если присутствует $u + qy$, то возможна замена: \( v = uy \);
				\item Если присутствует $u - px$, то возможна замена: \( v = \frac{u}{x} \);
				\item Если присутствует $u - qy$, то возможна замена: \( v = \frac{u}{y} \);
			\end{enumerate}

	% \pagebreak
