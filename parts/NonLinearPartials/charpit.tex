\section{Метод Лагранжа-Шарпи нахождения полного интеграла}

	% Рассмотрим частный случай уравнений с частными производными первого порядка. Положим двумерный случай
	Будем рассматривать двумерный случай уравнения с частными производными первого порядка:
	\[ F \pares{u, x, y, \dpart{u}{x}, \dpart{u}{y}} = 0, ~ u = u(x, y). \]

	В дальнейшем, для упрощения записи, будем использовать обозначения Монжа для частных производных:
	\[ F \pares{u, x, y, p, q} = 0, \]
	где $p = \dpart{u}{x}$ и $q = \dpart{u}{y}$.

	Так же, введем следующее обозначение для Якобиана:
	\[ \begin{vmatrix}
		\dpart{F_1}{x_1} & \dots & \dpart{F_1}{x_n} \\
		\vdots & \ddots & \vdots \\
		\dpart{F_n}{x_1} & \dots & \dpart{F_n}{x_n}
	\end{vmatrix} = \jac{F_1 & \dots & F_n}{x_1 & \dots & x_n} \]

	Метод Лагранжа-Шарпи заключается в нахождении сопряженного уравнения, содержащего первую произвольную постоянную:
	\[ G \pares{u, x, y, p, q} = C_1, \]
	такого, что система для $F$ и $G$ удовлетворяла условию вполне интегрируемости. Для этого необходимо, чтобы систему возможно было разрешить относительно переменных $p$ и $q$, и при этом
	\[ \jac{F & G}{p & q} \neq 0. \]

	Сама система для $F$ и $G$ принимает следующий вид:
	\[ \left\lbrace \begin{split} F \pares{u, x, y, p, q} &= 0, \\ G \pares{u, x, y, p, q} &= C_1. \end{split} \right. \]

	Предположим, что условия выполнены, тогда
	\[ p = A \pares{u, x, y}, \quad q = B \pares{u, x, y}. \]
	Чтобы система для $F$ и $G$ имела общее, совместное решение, необходимо, чтобы выполнялось условие:
	\[ \dpart{p}{y} = \dpart{q}{x} \equiv \dpartmix{u}{x}{y}. \]
	Тогда, полагая, что $p$ и $q$ известны, найдем соответствующие частные производные:
	\[ \left\lbrace \begin{split} 
		\dpart{p}{y} &= \dpart{A}{y} + \dpart{A}{u} \cdot \dpart{u}{y} = \dpart{A}{y} + \dpart{A}{u} \cdot B; \\
		\dpart{q}{x} &= \dpart{B}{x} + \dpart{B}{u} \cdot \dpart{u}{x} = \dpart{B}{x} + \dpart{B}{u} \cdot A.
	\end{split} \right. \]
	Запишем условие на совместное решение, полагая $A = p$ и $B = q$:
	\[ \dpart{p}{y} + \dpart{p}{u} \cdot q = \dpart{q}{x} + \dpart{q}{u} \cdot p. \]
	Это условие называется критерием интегрируемости. Если оно выполняется, то возможно построить общее решение системы для $F$ и $G$. Для нахождения вспомогательного уравнения $G$, построим ему соответствующее линейное дифференциальное уравнение с частными производными. Для этого найдем, соответственно $\dpart{p}{u}$, $\dpart{q}{u}$, $\dpart{p}{y}$ и $\dpart{q}{x}$ из системы для $F$ и $G$. Продифференцируем систему по переменным $u$, $x$ и $y$ соответственно, полагая, что $p$ и $q$ зависимы от них, из которых выразим необходимые производные. Рассмотрим случай для $u$, остальные -- аналогичны: 
	\[ \left\lbrace \begin{split} 
		\dpart{F}{u} + \dpart{F}{p} \cdot \dpart{p}{u} + \dpart{F}{q} \cdot \dpart{q}{u} &= 0; \\
		\dpart{G}{u} + \dpart{G}{p} \cdot \dpart{p}{u} + \dpart{G}{q} \cdot \dpart{q}{u} &= 0.
	\end{split} \right. \]
	Решаем эту систему относительно $\dpart{p}{u}$ и $\dpart{q}{u}$, получим:
	\[ \dpart{p}{u} = -\frac{\jac{F & G}{u & q}}{\jac{F & G}{p & q}}, \quad \dpart{q}{u} = -\frac{\jac{F & G}{p & u}}{\jac{F & G}{p & q}}. \]
	Применяя аналогичный метод для переменных $x$ и $y$, нетрудно показать, что:
	\[ \dpart{p}{y} = -\frac{\jac{F & G}{y & q}}{\jac{F & G}{p & q}}, \quad \dpart{q}{x} = -\frac{\jac{F & G}{p & x}}{\jac{F & G}{p & q}}. \]

	Подставим найденные производные в критерий интегрируемости, полагая, что функция $G$ -- неизвестна. Сократим знаменатели (общие, по условию не равные нулю), упростим выражение, получим:
	\[ \jac{F & G}{y & q} + \jac{F & G}{u & q} \cdot q = \jac{F & G}{p & x} + \jac{F & G}{p & u} \cdot p. \]
	Раскрывая Якобианы, перенося слагаемые в одну сторону, получим:
	\[ \begin{split}
		&\dpart{F}{y} \dpart{G}{q} - \dpart{F}{q} \dpart{G}{y} 
		+ \pares{ \dpart{F}{u} \dpart{G}{q} - \dpart{F}{q} \dpart{G}{u} } \cdot q - \\
		- &\dpart{F}{p} \dpart{G}{x} + \dpart{F}{x} \dpart{G}{p} 
		- \pares{ \dpart{F}{p} \dpart{G}{u} - \dpart{F}{u} \dpart{G}{p} } \cdot p = 0.
	\end{split} \]

	Приведем это уравнение к каноническому линейному уравнению с частными производными относительно неизвестной функции $G$. Произведем следующие переобозначения, полагая в общем случае все функции аргументов $\pares{u, x, y, p, q}$:
	\[ \begin{split}
		U &:= -q \cdot \dpart{F}{q} - p \cdot \dpart{F}{p}, ~ X := -\dpart{F}{p}, ~ Y := -\dpart{F}{q}, \\ 
		P &:= \dpart{F}{x} + p \cdot \dpart{F}{u}, ~ Q := \dpart{F}{y} + q \cdot \dpart{F}{u}.
	\end{split} \]

	Применяя эти переобозначения, получим следующее линейное уравнение относительно функции $G$:
	\[ U \cdot \dpart{G}{u} + X \cdot \dpart{G}{x} + Y \cdot \dpart{G}{y} + P \cdot \dpart{G}{p} + Q \cdot \dpart{G}{q} = 0. \]
	Для этого уравнения достаточно найти первый интеграл вида:
	\[ G \pares{u, x, y, p, q} = C_1, \]
	который удовлетворяет условию разрешимости вместе с $F$ относительно $p$ и $q$. Производя разрешение, получим, что
	\[ p = \vp_1 \pares{u, x, y, C_1}, ~ q = \vp_2 \pares{u, x, y, C_1}. \]
	% Замечение, что в силу однородности этого уравнения, интегрируемую комбинацию можно выбрать и при том условии, что в симметрической системе присутствует уравнение 
	% \[ \cdots = \frac{dG}{0}. \]
	Подставим разрешенные конструкции в уравнение Пфаффа:
	\[ du = p ~ dx + q ~ dy, \]
	получим вполне интегрируемое уравнение. В результате его интегрирования, получим общее решение, которое запишем в форме:
	\[ \Phi \pares{u, x, y, C_1, C_2} = 0. \]

	В дополнение стоит сказать, что этот метод является двумерным расширением метода общей параметризации для нелинейного обыкновенного дифференциального уравнения.

	% \pagebreak
	\subsection{Примеры}

		В качестве примера рассмотрим линейное неоднородное уравнение с частными производными, для которого построим общее решение методом характеристик, и сравним его с методом Лагранжа-Шарпи. В общем случае, конечно, данный метод решает нелинейные уравнения. Дано уравнение:
		\[ \frac{p}{y} - \frac{q}{x} = \frac{1}{x} + \frac{1}{y}. \]
		Упростим запись уравнения путем домножения на $xy$:
		\[ xp - yq = x + y. \]
		Соответствующая симметрическая характеристическая система имеет вид:
		\[ \frac{dx}{x} = \frac{dy}{-y} = \frac{du}{x+y}. \]
		Выделяя интегрируемые комбинации, получим общее решение:
		\[ \Phi \pares{xy, u - x + y} = 0, \]
		или, если выразить $u$ в явном виде:
		\[ u = x - y + \Phi \pares{xy}. \]

		Теперь построим решение методом Лагранжа-Шарпи. Для этого будем искать вид вспомогательного уравнения $G \pares{u, x, y, p, q} = C_1$. Запишем вид функции $F$:
		\[ F \pares{u, x, y, p, q} = xp - yq - x - y. \]
		Для дальнейших преобразований найдем соответствующие производные функции $F$:
		\[ \dpart{F}{u} = 0, ~ \dpart{F}{x} = p - 1, ~ \dpart{F}{y} = -q - 1, ~ \dpart{F}{p} = x, ~ \dpart{F}{q} = -y. \]
		С этого этапа можно сразу строить соответствующее линейное уравнение для функции $G$, находя коэффициенты по выведенным выражениям ранее. Здесь же рассмотрим вывод этого уравнения. Так как система уравнений для $F$ и $G$ должна быть совместна, она должна удовлетворять критерию интегрируемости:
		\[ \dpart{p}{y} = \dpart{q}{x}. \]
		Полагая, что $p, q$ зависят не только от $x, y$, но еще и от $u$, критерий интегрируемости принимает вид:
		\[ \dpart{p}{y} + \dpart{p}{u} \cdot q = \dpart{q}{x} + \dpart{q}{u} \cdot p. \]
		Найдем соответствующие производные функций $p, q$ по переменным $u, x$ и $y$. Для этого продифференцируем уравнения системы $F, G$ по этим переменным. Так как уравнения будут похожи для каждой переменной, положим, что $\xi$ будет отвечать за конкретную переменную из этого набора. Система принимает вид:
		\[ \left\lbrace \begin{split} 
			\dpart{F}{\xi} + \dpart{F}{p} \cdot \dpart{p}{\xi} + \dpart{F}{q} \cdot \dpart{q}{\xi} &= 0; \\
			\dpart{G}{\xi} + \dpart{G}{p} \cdot \dpart{p}{\xi} + \dpart{G}{q} \cdot \dpart{q}{\xi} &= 0. 
		\end{split} \right. \]
		Решаем эту систему относительно переменных $\dpart{p}{\xi}$ и $\dpart{q}{\xi}$. Получим:
		\[ 
			\dpart{p}{\xi} = - \frac{\jac{F & G}{\xi & q}}{\jac{F & G}{p & q}}, \quad
			\dpart{q}{\xi} = - \frac{\jac{F & G}{p & \xi}}{\jac{F & G}{p & q}}. 
		\]

		% \pagebreak
		Для подстановки в критерий интегрируемости достаточно посчитать числители этих выражений. Просчитаем индивидуально для необходимых переменных:
		\[ \jac{F & G}{u & q} = y \dpart{G}{u}, ~ \jac{F & G}{p & u} = x \dpart{G}{u}, \]
		\[ \jac{F & G}{y & q} = -\pares{q + 1} \cdot \dpart{G}{q} + y \dpart{G}{y}, \]
		\[ \jac{F & G}{p & x} = x \dpart{G}{x} - \pares{p - 1} \cdot \dpart{G}{p}. \]

		Подставим найденные производные в критерий интегрируемости, и приведем сразу к линейному уравнению с частными производными:
		\[ \pares{yq - px} \dpart{G}{u} - x \dpart{G}{x} + y \dpart{G}{y} + \pares{p - 1} \dpart{G}{p} - \pares{q + 1} \dpart{G}{q} = 0. \] 
		Соответствующая симметрическая характеристическая система принимает вид:
		\[ \frac{dx}{-x} = \frac{dy}{y} = \frac{du}{yq - px} = \frac{dp}{p-1} = \frac{dq}{-\pares{q + 1}}. \]
		Нетрудно заметить, что если домножить первый элемент на $p$, второй на $q$, сложить и приравнять к третьему, после упрощений получится вполне интегрируемое уравнение Пфаффа:
		\[ du = p ~ dx + q ~ dy. \]
		Второе уравнение системы же получим из первого интеграла на основе двух последних элементов этой симметрической системы:
		\[ \frac{dp}{p-1} + \frac{dq}{q+1} = 0, \]
		или, интегрируя, получим:
		\[ \pares{p-1} \cdot \pares{q+1} = \tilde{C}_1. \]
		Получим систему из двух уравнений:
		\[ \left\lbrace \begin{split} 
			&xp - yq = x + y; \\
			&\pares{p-1} \cdot \pares{q+1} = \tilde{C}_1.
		\end{split} \right. \]
		Из последнего уравнения выразим $p$, переобозначая для удобства $\tilde{C}_1 := C_1^2$:
		\[ p = \frac{C_1^2}{q+1} + 1. \]
		Подставим это выражение в первое уравнение системы:
		\[ \frac{C_1^2 x}{q + 1} + x - yq = x + y. \]
		Упростим, и выразим $q$:
		\[ q = C_1 \sqrt{\frac{x}{y}} - 1. \]
		Подставляя это в $p$, получим:
		\[ p = C_1 \sqrt{\frac{y}{x}} + 1. \]
		Подставляя это выражение в уравнение Пфаффа, получим:
		\[ du = C_1 \pares{\sqrt{\frac{y}{x}} ~ dx + \sqrt{\frac{x}{y}} ~ dy } + dx - dy. \]
		Данное уравнение является вполне интегрируемым, общее решение примет вид:
		\[ u = 2C_1 \sqrt{xy} + x - y + C_2, \]
		что является частным случаем решения методом характеристик, где
		\[ \Phi \pares{\vp_1} = 2C_1 \sqrt{\vp_1} + C_2. \]

		Аналогичные частные случаи решения линейного уравнения можно также получить, выбрав другие комбинации для первого интеграла в сопряженном уравнении, например $\frac{dp}{p-1} = \frac{dy}{y}$, из чего вытекает система $p = C_1 y + 1$, $q = C_1 x - 1$, и общий интеграл принимает вид:
		\[ u = C_1 xy + x - y + C_2, \]
		что так же удовлетворяет частному решению общего, полученного методом характеристик.

	% \pagebreak