\section{Уравнения Лагранжа и Клеро}

	Рассматривать будем частные случаи уравнений, неразрешенных относительно производной:
	\[ y = x \cdot \vp(y') + \psi(y'). \]
	Здесь $\vp(y')$ и $\psi(y')$ -- некоторые функции, зависимые только от производной неизвестной функции. Такие уравнения называются уравнениями Лагранжа. Воспользуемся методом общей параметризации для построения общего решения. Введем параметризацию $y' = p$, и вспомогательную функцию $G(x, y, p)$:
	\[ \left\lbrace \begin{split}
		&y = x \cdot \vp(p) + \psi(p), \\
		&G(x, y, p) = C.
	\end{split} \right. \]
	Согласно методу общей параметризации, найдем полный дифференциал первого уравнения:
	\[ dy = x \cdot \vp'(p) ~ dp + \vp(p) ~ dx + \psi'(p) ~ dp. \]
	Воспользуемся параметризацией $dy = p ~ dx$, тем самым исключим из уравнения зависимость от $y$:
	\[ p ~ dx = x \cdot \vp'(p) ~ dp + \vp(p) ~ dx + \psi'(p) ~ dp. \]
	Упростим выражение:
	\[ \bpares{p - \vp(p)} ~ dx = \bpares{x \cdot \vp'(p) + \psi'(p)} ~ dp. \]
	Если разделить всё уравнение на $dp$, полагая $x = x(p)$, получим:
	\[ \bpares{p - \vp(p)} \cdot \difft{x}{p} = \vp'(p) \cdot x + \psi'(p). \]
	Получено линейное неоднородное уравнение для функции $x(p)$. Его решение представим в виде:
	\[ x = C \cdot x_h(p) + x_{nh}(p). \]
	Тогда общее решение уравнений Лагранжа можно представить в виде следующей параметрической системы:
	\[ \left\lbrace \begin{split} 
		x &= C \cdot x_{h}(p) + x_{nh}(p), \\ 
		y &= x \cdot \vp(p) + \psi(p).
	\end{split} \right. \]

	Особенностью уравнений Лагранжа является то, что они являются линейными относительно $x$ и $y$, и при этом вспомогательное уравнение так-же является линейным уравнением.

	Частным случаем уравнений Лагранжа являются уравнения Клеро:
	\[ y = xy' + \psi(y'). \]
	Их отличием является то, что $\vp(y') = y'$. Проводя те же шаги, что и для уравнений Лагранжа, можно прийти к следующему уравнению для функции $x(p)$:
	\[ \pares{p - p} ~ dx = \bpares{x + \psi'(p)} ~ dp, \]
	или, что то же:
	\[ \bpares{x + \psi'(p)} ~ dp = 0. \]
	У полученного уравнения существует два дальнейших исхода решения:
	\begin{enumerate}
		\item $dp = 0$. Тогда $p = C$, и параметрическая система для общего решения принимает вид:
			\[ \left\lbrace \begin{split} &y = xp + \psi(p), \\ &p = C. \end{split} \right. \]
			Так как $p$ теперь извесно в явном виде, его можно подставить в первое уравнение, тем самым избавиться от системы:
			\[ y = Cx + \psi(C). \]
			Это выражение представляет собой общее решение уравнения Клеро.
		\item $x + \psi'(p) = 0$. Тогда $x = - \psi'(p)$, и параметрическая система для решения принимает следующий вид:
			\[ \left\lbrace \begin{split} &y = xp + \psi(p), \\ &x = - \psi'(p). \end{split} \right. \]
			Данная система будет являться особым решением уравнения Клеро.
	\end{enumerate}

	\subsection{Примеры}

		\begin{enumerate}
			\item Рассмотрим следующее уравнение:
				\[ yy' + x = 2 y' \sqrt{y'^2 + 1}^3. \]
				В данном уравнении $x$ и $y$ входят линейно, соответственно данное уравнение является уравнением Лагранжа. Приведем его к общему виду:
				\[ y = - \frac{x}{y'} + 2 \sqrt{y'^2 + 1}^3. \]
				Построим его решение методом общей параметризации. Введем параметр $y' = p$:
				\[ y = - \frac{x}{p} + 2 \sqrt{p^2 + 1}^3. \]
				Найдем полный дифференциал уравнения:
				\[ dy = -\frac{dx}{p} + \frac{x}{p^2} ~ dp + 6p \sqrt{p^2 + 1} ~ dp. \]
				Воспользуемся параметризацией $dy = p ~ dx$, исключая из уравнения $y$:
				\[ p ~ dx = - \frac{dx}{p} + \frac{x}{p^2} ~ dp + 6p \sqrt{p^2 + 1} ~ dp. \]
				Упростим:
				\[ \pares{p + \frac{1}{p}} ~ \difft{x}{p} = \frac{x}{p^2} + 6p \sqrt{p^2 + 1}. \]
				Сведем уравнение к приведенному линейному неоднородному уравнению первого порядка относительно $x(p)$:
				\[ \difft{x}{p} - \frac{x}{p \cdot \pares{p^2 + 1}} = \frac{6p^2 }{\sqrt{p^2 + 1}}. \]
				Общее решение данного уравнения имеет следующий вид:
				\[ x = \frac{Cp + 3p^3}{\sqrt{p^2 + 1}}. \]
				Тогда общее решение исходного уравнения Лагранжа имеет вид:
				\[ \left\lbrace \begin{split} 
					x &= \frac{Cp + 3p^3}{\sqrt{p^2 + 1}}, \\
					y &= - \frac{x}{p} + 2 \sqrt{p^2 + 1}^3.
				\end{split} \right. \]

			\item Рассмотрим другое уравнение:
				\[ y = xy' + \sin{y'}. \]
				Здесь так же, $x$ и $y$ входят в уравнение линейно, при этом при $x$ стоит $\vp(y') = y'$. Представленное уравнение является уравнением Клеро. Построим общее и особое решения. Введем параметризацию $y' = p$:
				\[ y = xp + \sin{p}. \]
				Найдем полный дифференциал уравнения:
				\[ dy = x ~ dp + p ~ dx + \cos{p} ~ dp. \]
				Воспользуемся параметризацией $dy = p ~ dx$, исключая из уравнения $y$:
				\[ p ~ dx = x ~ dp + p ~ dx + \cos{p} ~ dp. \]
				Упростим:
				\[ \pares{x + \cos{p}} ~ dp = 0. \]
				Построим общее решение. Для этого рассмотрим случай, когда $dp = 0$:
				\[ p = C \implies y = Cx + \sin{C}. \]
				Теперь построим особое решение:
				\[ \left\lbrace \begin{split} x &= - \cos{p}, \\ y &= xp + \sin{p}. \end{split} \right. \]

		\end{enumerate}
