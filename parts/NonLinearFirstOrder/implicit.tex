\section{Уравнения не разрешенные относительно производной}

	Уравнения вида
	\[ F(x, y, y') = 0, \]
	в которых невозможно в явном виде выразить производную искомой функции так, чтобы получить уравнение вида
	\[ y' = f(x, y) \]
	называются уравнениями, неразрешенными относительно производной. Для таких уравнений решение можно найти или в явном виде, или в параметрическом. Рассмотрим метод общей параметризации. Согласно методу, введем параметр $y' = p$, относительно которого будем строить решение. Тогда уравнение принимает вид:
	\[ F(x, y, p) = 0. \]
	Для построения общего решения, введем вспомогательную функцию $G(x, y, p)$ такую, чтобы:
	\[ \left\lbrace \begin{split} F(x, y, p) &= 0, \\ G(x, y, p) &= C. \end{split} \right. \]
	Здесь $y' = p$ считается вспомогательным условием совместности системы, которое необходимо удовлетворить. Найдем полную производную всей системы по переменной $x$ (где $y = y(x)$):
	\[ \left\lbrace \begin{split} 
		\dpart{F}{x} + p \cdot \dpart{F}{y} + \dpart{F}{p} \cdot \difft{p}{x} &= 0, \\ 
		\dpart{G}{x} + p \cdot \dpart{G}{y} + \dpart{G}{p} \cdot \difft{p}{x} &= C. 
	\end{split} \right. \]
	Выразим $\difft{p}{x}$ из первого уравнения системы:
	\[ \difft{p}{x} = -\frac{\dpart{F}{x} + p \cdot \dpart{F}{y}}{\dpart{F}{p}}, \]
	и подставим его во второе уравнение:
	\[ \dpart{G}{x} + p \dpart{G}{y} + \dpart{G}{p} \cdot \pares{ -\frac{\dpart{F}{x} + p \dpart{F}{y}}{\dpart{F}{p}} } = 0. \]
	Домножим на $\dpart{F}{p}$:
	\[ \dpart{F}{p} \cdot \dpart{G}{x} + p \dpart{F}{p} \cdot \dpart{G}{y} - \pares{ \dpart{F}{x} + p \dpart{F}{y} } \cdot \dpart{G}{p} = 0. \]
	Получено уравнение для вспомогательной функции $G$, где $F$ -- известная функция. В общем случае решение таких уравнений описано в разделах ниже. Но если удается получить решение, то система:
	\[ \left\lbrace \begin{split} F(x, y, p) &= 0, \\ G(x, y, p) &= C \end{split} \right. \]
	является параметрическим решением исходного уравнения. Мы рассмотрим некоторые более простые частные случаи метода общей параметризации:
	\begin{enumerate}
		\item Случай, когда в уравнении присутствует только производная:
			\[ F(y') = 0. \]
			Данное выражение представляет собой задачу на поиск таких значений аргумента функции $F$, при которых само выражение является верным. Положим некоторый набор постоянных $\{ p_k: ~ k \in \mathbb{Z} \}$, для которых выполняется следующее условие:
			\[ F(p_k) = 0. \]
			Тогда, согласно исходному уравнению, задача сводится к нахождению таких функций $y$, для которых выполняется следующее условие:
			\[ y' = p_k. \]
			Интегрируя данное выражение, получим семейство линейных функций $y$, удовлетворяющих исходному уравнению:
			\[ y = p_k \cdot x + C. \]
			Выражая из данного выражения $p_k$, и подставляя в исходное уравнение, получим общее решение исходного уравнения:
			\[ p_k = \frac{y + C}{x}, ~ F\pares{\frac{y + C}{x}} = 0. \]

		\item Случай, когда в уравнении отсутствует искомая функция:
			\[ F(x, y') = 0. \]
			Если в данном уравнении возможно выразить $y'$, тогда уравнение сразу сводится к уравнению с разделяющимися переменными. Если же в уравнении возможно выразить $x$ в явном виде:
			\[ x = f(y'), \]
			то уравнение можно решить с помощью метода общей параметризации. Введем параметр $y' = p$ (или, что то же -- $dy = p ~ dx$) и вспомогательную функцию $G(x, y, p)$:
			\[ \left\lbrace \begin{split} &x = f(p), \\ &G(x, y, p) = C. \end{split} \right. \]
			Первое уравнение уже представляет собой параметрическую запись функции $x(p)$. Теперь необходимо найти параметричекую функцию, содержащую в себе $y(p)$. Этой функцией как раз и будет выступать вспомогательная функция $G(x, y, p)$.
			Построим полный дифференциал первого уравнения:
			\[ dx = f'(p) ~ dp. \]
			Если домножить данное уравнение на $p$, то левая часть будет иметь конструкцию $p ~ dx$, что, согласно параметризации, равно $dy$:
			\[ dy = p \cdot f'(p) ~ dp. \]
			Данное уравнение будем считать вспомогательным уравнением для функции $G(x, y, p)$. Интегрируя, получим в явном виде параметрическое выражение для искомой функции $y(p)$:
			\[ y = \int p \cdot f'(p) ~ dp + C. \]
			Подставляя данное выражение вместо функции $G(x, y, p)$ в исходную систему, получаем общее параметрическое решение уравнения:
			\[ \left\lbrace \begin{split} x &= f(p), \\ y &= \int p \cdot f'(p) ~ dp + C. \end{split} \right. \]

		\item Случай, когда в уравнении отсутствует аргумент в явном виде:
			\[ F(y, y') = 0. \]
			Ситуация аналогична предыдущему случаю. Если в данном уравнении возможно выразить $y'$, то уравнение сводится к уравнению с разделяющимися переменными. Если же можно выразить $y$ в явном виде:
			\[ y = f(y'), \]
			то решение можно построить аналогично предыдущему случаю. Введем параметр $y' = p$:
			\[ y = f(p). \]
			Данное выражение представляет первую функцию параметрической системы решений - функция $y(p)$ в явном виде. Найдем теперь функцию для $x(p)$. Для этого построим полный дифференциал этого выражения:
			\[ dy = f'(p) ~ dp. \]
			Из параметризации следует, что $dy = p ~ dx$. Тогда, разделив данное уравнение на $p$, получим:
			\[ dx = \frac{f'(p)}{p} ~ dp. \]
			Данное уравнение представляет собой уравнение с разделяющимися переменными для функции $x(p)$. Проинтегрируем, и запишем общее решение в виде системы:
			\[ \left\lbrace \begin{split} x &= \int \frac{f(p)}{p} ~ dp + C, \\ y &= f(p). \end{split} \right. \]

		\item Случай, где можно выразить $x$ или $y$:
			\[ F(x, y, y') = 0, \quad y = f(x, y') \quad \text{или} \quad x = f(y, y'). \]
			Эти случаи похожи между собой. Решение построим методом общей параметризации. Введем параметр $y' = p$, и будем искать вспомогательную функцию $G(x, y, p)$:
			\[ \left\lbrace \begin{split} &x = f(y, p), \\ &G(x, y, p) = C. \end{split} \right. \]
			Первое уравнение уже известно, осталось найти зависимость между $x$ и $p$, или между $y$ и $p$ для второго уравнения. Построим полный дифференциал первого уравнения:
			\[ dx = \dpart{f}{y} ~ dy + \dpart{f}{p} ~ dp. \]
			В данном уравнении возможно исключить $x$ путем домножения всего уравнения на $p$, при $dy = p ~ dx$:
			\[ dy = p \cdot \dpart{f}{y} ~ dy + p \cdot \dpart{f}{p} ~ dp. \]
			Перенесем всё в одну часть, получим:
			\[ \pares{p \cdot \dpart{f}{y} - 1} ~ dy + p \cdot \dpart{f}{p} ~ dp = 0. \]
			Переобозначим функции:
			\[ M(y, p) ~ dy + N(y, p) ~ dp = 0. \]
			Получено уравнение первого порядка, разрешенное относительно производной. Решением этого уравнения будет некоторая функция $G(y, p) = C$, которая и будет являться вспомогательной функцией в параметрической системе решения. Тогда общее решение принимает вид:
			\[ \left\lbrace \begin{split} &x = f(y, p), \\ &G(y, p) = C. \end{split} \right. \]
			Аналогично выводится система, если в исходном уравнении возможно выразить $y$.

	\end{enumerate}

	\subsection{Примеры}

		Рассмотрим следующий пример:
		\[ y'^3 - 1 = 0. \]
		Данное уравнение представляет собой уравнение вида $F(y') = 0$. Здесь $y' = p_k$, где $p_k$ -- нули функции $F$. Искать их не обязательно, вид общего решения выведен выше. Общее решение имеет следующий вид:
		\[ \pares{\frac{y + C}{x}}^3 - 1 = 0, \]
		или, упрощая:
		\[ \pares{y + C}^3 - x^3 = 0. \]

		Рассмотрим другой пример:
		\[ x = y' + e^{y'}. \]
		Уравнение представляет собой уравнение вида $x = f(y')$. Построим решение с помощью метода общей параметризации. Введем параметр $y' = p$:
		\[ x = p + e^p. \]
		Получен готовый вид фукнции $x(p)$, теперь необходимо найти функцию $y(p)$. Найдем полный дифференциал этого уравнения:
		\[ dx = \pares{1 + e^{p}} ~ dp. \]
		Согласно параметризации, для того, чтобы найти уравнение для $y$, необходимо домножить уравнение на $p$, и воспользоваться самой параметризацией -- $dy = p ~ dx$:
		\[ dy = \pares{p + pe^{p}} ~ dp. \]
		Получено уравнение с разделяющимися переменными. Достаточно проинтегрировать, чтобы получить функцию $y(p)$ в явном виде:
		\[ y = \frac{p^2}{2} + \pares{p - 1} e^{p} + C. \]
		Тогда общее решение принимает вид:
		\[ \left\lbrace \begin{split} x &= p + e^{p}, \\ y &= \frac{p^2}{2} + \pares{p-1} e^{p} + C. \end{split} \right. \]

		Рассмотрим еще один пример:
		\[ y = x + y' + \ln{y'}. \]
		Уравнение представляет собой уравнение вида $y = f(x, y')$. Построим его решение с помощью метода общей параметризации. Введем параметр $y' = p$:
		\[ y = x + p + \ln{p}. \]
		Найдем вспомогательное уравнение. Для этого построим полный дифференциал этого выражения:
		\[ dy = dx + dp + \frac{dp}{p}. \]
		Здесь можно или домножить на $p$ для выражения $p ~ dx = dy$, или разделить на $p$ для выражения $dx = \frac{dy}{p}$. Домножим на $p$, получим:
		\[ p ~ dy = dy + p ~ dp + dp. \]
		Упростим:
		\[ \pares{1 - p} ~ dy + \pares{p + 1} ~ dp = 0. \]
		Получено уравнение с разделяющимися переменными. Разделим их:
		\[ dy = \frac{p + 1}{p - 1} ~ dp. \]
		Проинтегрируем:
		\[ y = p + 2\ln\abs{p - 1} + C. \]
		Тогда общее решение можно записать в виде следующей параметрической системы:
		\[ \left\lbrace \begin{split} y &= x + p + \ln{p}, \\ y &= p + 2 \ln\abs{p-1} + C. \end{split} \right. \]
		Из первого уравнения можно выразить $x$ как функцию переменных $y, p$, и затем подставить в него значение $y$, чтобы получить $x(p)$ и $y(p)$ в явном виде, но и полученная система также является решением, заданном в неявном виде.

	\pagebreak